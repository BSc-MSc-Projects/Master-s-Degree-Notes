\documentclass[16px]{article}
\usepackage{graphicx}
\usepackage[utf8]{inputenc}

\begin{document}
\tableofcontents
\section{Introduzione-warm up example}
Il problema spesso è che una buona crittografia è applicata male alla risoluzione di un problema.esempio: paper che discute di una tecnica di sicurezza ed in cui viene matematicamente provata la sicurezza.\\ Basata sul meccanismo del One Time Pad: ho il mio plain text e voglio criptarlo in modo che non si capisca cosa ci sia scritto. Genero una sequenza random di bit, di lunghezza pari alla lunghezza del testo. Una volta ottenuta la chiave, computo lo XOR fra la chiave ed il plaintext ed ottengo il mio ciphertext.\\ Il seguente meccanismo è il migliore possibile per fare enrcyption.\\ Per decriptare applico il procedimento inverso, facendo sempre l'XOR, infatti: b $\oplus$ 1 = $\overline{b}$ $\oplus$ 1 = b.\\ Devo però fare delle assunzioni:
\begin{enumerate}
\item Per ogni nuovo messaggi, devo usare una diversa chiave. Questo perché, se ripetessi la chiave avrei il peggior meccanismo di enrcyption: se ho due messaggi M$_{1}$ ed M$_{2}$, ed ottengo C$_{1}$ = M$_{1}$ $\oplus$ K e C$_{2}$ = M$_{2}$ $\oplus$ K, ora facendo C$_{1}$ $\oplus$ C$_{2}$ = (M$_{1}$ $\oplus$ K) $\oplus$ (M$_{2}$ $\oplus$ K) = M$_{1}$ $\oplus$ M$_{2}$ (in quanto K $\oplus$ K si elide).Conoscendo uno dei due messaggi posso ricavare l'altro.
\item La chiave deve essere lunga quanto il plaintext
\item La chiave deve essere veramente random, e non pseudorandom.
\end{enumerate}
L'algoritmo introdotto sopra è chiamato Vernam Cipher, ed è il miglior meccanismo di enrcyption possibile.\\ Il problema nel metterlo in pratica è che la chiave deve essere nota sia a chi produce il messaggio, sia a chi lo riceve, quindi va trasmessa su un canale sicuro. Se le dimensioni della chiave cominciano a diventare considerevoli, ad esempio per 2GB di plaintext devo avere 2GB di chiave, il costo d'invio al receiver diventa oneroso.\\ Quando si parla di sicurezza, non bisogna chiedersi come rendere il sistema sicuro, ma da cosa devo difendermi, di cosa è capace l'attaccante.\\ OTP protegge la confidenzialità, ma non garantisce l'integrità.\\ Le 3 proprietà che posso/voglio garantire sono:
\begin{itemize}
\item Confidenzialità: proteggo i dati da persone esterne,che non possono leggere il contenuto senza una chiave
\item Integrità: voglio che i miei dati rimangano inalterati
\item Aviability: mi proteggo, ad esempio da DDoS
\end{itemize}
Nel caso di OTP, l'integrità non è garantita: se un avversario prende il mio messaggio e lo cambia,ad esempio flippando alcuni byte (Man in the middle attack) non riesco a rendermene conto; l'avversario ha agito sul testo cifrato, senza interessarsi del,contenuto.
\section{Encryption}
Servizio di sicurezza che vuole proteggere la confidenzialità dei dati, non protegge però l'integrità. Si parte dal plaintext $\longrightarrow$ encryption $\longrightarrow$ cipher text $\longrightarrow$ invio $\longrightarrow$ decryption $\longrightarrow$ plaintext.\\ Servono delle chiavi per potere de/criptare, per ora mi concentro sul meccanismo della symmetric key: sia sender che receiver usano la stessa chiave. Il cipher sarà il mio algoritmo per criptare e decriptare:\\
C = ENC(K,P) D = DEC(K,C).
\subsection{Esempi storici di cipher}
Un primo esempio può essere quello di sostituire le lettere del plaintext con altre lettere, in maniera reversibile ovvero se a $\longrightarrow$ b, non posso avere c $\longrightarrow$ b.\\ Posso decriptare in maniera veloce? Vedo la frequenza delle lettere di una lingua, ad esempio l'italiano, e saprò quali lettere compaiono più spesso in un plaintext, inoltre posso avere delle parole con delle ripetizioni interne.\\ Procedo per tentativi, nel momento in cui deduco una lettera, la associo ad una più o meno probabile.\\ Posso inoltre ricavare la chiave usata per criptare.\\ Il metodo è storico (me pare addirittura lo usavano i romani sotto Cesare),  quindi fortemente sconsigliato.
\subsection{RFID mutual authentication}
Vernam cipher è il miglior meccanismo, ma ha delle forti implicazioni. Considero una situazione reale:\\ ho un TAG, ed un reader presso cui devo autenticarmi. Il TAG ha lo scopo di provare al reader che l'utente è davvero reale, ma anche il reader dovrebbe dimostrare di essere sicuro; voglio quindi che l'autenticazione sia mutua.\\ Ho un segreto S, scritto ad esempio in una mia carta d'autenticazione e quando mi approccio al reader mi devo identificare. Ho due probelmi:
\begin{itemize}
\item La trasmissione avviene su un canale wireless, se poi la trasmissione è in chiaro un attacker può captare e rubare S
\item Se il reader è falso, ora conosce il mio segreto S
\end{itemize}
Vorrei poter autenticarmi senza mostrare il segreto S esplicitamente, uso uno schema:\\ il TAG ha un segreto S,statico, ed una chiave temporanea k. Invece di trasmettere S, invio k $\oplus$ S al reader, che avrà un database in cui ha il segreto salvato e la chiave k. Il reader riesegue quindi lo XOR e vede se il risultato coincide con quello che gli ho inviato io. La prossima chiave sarà generata dal reader a partire da k, quindi con un meccanismo pseudorandom. Provo ad usare un former analyzer, che mi garantisce che il sistema è sicuro (software che prova a crackare il meccanismo di encryption). Sono realmente sicuro?\\ Ho l'operazione k $\oplus$ S, k è pseudorandom e posso avere due situazioni:
\begin{enumerate}
\item S è la chiave: sto violando la proprietà 1, in quanto riuso S più volte per messaggi diversi
\item k è la chiave: se faccio (S $\oplus$ k$_{i}$) $\oplus$ (S $\oplus$ k$_{i+1}$) = k$_{i}$ $\oplus$ k$_{i+1}$. Non ho violato il sistema, ma ho la combinazione delle due chiavi, che sono pseudorandom: ho x $\oplus$ f(x) (PNRG(x)), x dipende da f(x), quindi posso fare un ciclo fino al valore massimo, controllo se z$_{i}$ = x$_{i}$ $\oplus$ PNRG(x$_{i}$), alla fine ricaverò k$_{i}$.
\end{enumerate}
Il meccanismo non può essere risolto in alcun modo, le assunzioni erano errate, quindi:
\begin{itemize}
\item Former analyzers non sono una certezza, bisogna comunque verificare che l'assunzione è corretta
\item Random e pseudorandom sono completamente diversi
\end{itemize}
\subsection{Sicurezza di un cipher}
Un cipher è sicuro quando:
\begin{itemize}
\item protegge la confidenzialità
\item nasconde i messaggi
\item non può essere violato
\end{itemize}
Ma questa definizione è una supercazzola (serio, così ha detto il prof a lezione e così scrivo io negli appunti), e anche altre definizioni sono brutte e sbagliate.
Un cipher può essere sicuro per un determinato attacco che vuole svelare il contenuto, ma non sicuro per un altro che vuole vedere solo parte delle coppie plaintext-ciphertext.\\ Ad esempio un chosen plaintext attack permette di vedere sia plaintext che ciphertext, voglio essere robusto quantomeno a questo tipo di attacco.\\ Definizione di semantically secure o IND-CPA, ovvero Indistinguishability Under Chosen Plaintext Attack. \\ esempio: ho due messaggi, M$_{0}$ ed M$_{1}$, suppongo di poter criptare solo uno dei due. \\ Permetto all attacker di mandarmi i due messaggi ed io scelgo a caso quale dei due criptare con un coinflip. Mando indietro il messaggio cifrato all'attaccante: in condizioni normali l'attacante può facilmente decriptare il messaggio, se usassi un cipher non semantically secure, ma ora entra in gioco IND-CPA $\Rightarrow$ l'attaccante ha una probabilità del 50\% di ottenere il messaggio corretto, ovvero deve scegliere a caso fra i due.I sistema sarà semantically secure se l'avversario non può risolvere questa situazione: ha a disposizione un oracolo, che gli fornisce l'encryption dei due messagi, quindi se uso un meccanismo di encryption sostitutivo (vedi esempio de Giulio Cesare) $\Rightarrow$ GAME OVER. Ora uso una chiave random (esempio Vernam Cipher): allo stesso plaintext corrispondono ciphertext diversi, quindi l'oracolo non può fornire il risultato esatto all'attaccante. L'unico modo che ha per vincere è di tirare ad indovinare, quindi con un coinflip.\\ L'encryption deve essere random, perché se una sotto stringa si ripete non deve corrispondere allo stesso ciphertext. Lo XOR è random: \\
bit segreto $\oplus$ bit random\\
bit segreto: 0 = p, 1 = 1-p\\ 
bit random: 0 = $\frac{1}{2}$, 1 = $\frac{1}{2}$\\
Avrò quindi: \\
\begin{table}
\begin{tabular}{||c c c c||}
\hline\hline
bit segreto & bit random & XOR & probabilità \\
\hline
0 & 0 & 0 & $\frac{p}{2}$\\
\hline
0 & 1 & 1 & $\frac{p}{2}$\\
\hline
1 & 0 & 1 & $\frac{(1-p)}{2}$\\
\hline
1 & 1 & 0 & $\frac{(1-p)}{2}$\\
\end{tabular}
\end{table}
Quindi il Vernam ciper è perfettamente random: l'avversario vede solo il cipher text, quindi può indovinare 0 o 1 con probabilità: $\frac{p}{2}$ + $\frac{(1-p)}{2}$ = $\frac{1}{2}$.
Vernam cipher è però teorico e nella pratica si usano altri cipher, divisi in categorie:
\begin{itemize}
\item stream cipher: un mimic di Vernam cipher, usa un algoritmo pseudorandom usando lo XOR, il più famoso era RC4, oggi si usano Salsa20 e ChaCha20.
\item Block cipher: il più usato è AES, usano una tecnica diversa
\item Block cipher in stream mode: AES-CTR, il block cipher genera una chiave pseudorandom e poi usa uno stream cipher.
\end{itemize}
\subsection{Stream cipher}
L'obbiettivo è quello di approssimare One Time Pad: invece di usare una chiave random, uso una chiave di 128 bit come seed per uno stream di bit pseudorandom, che sarà il keystream.\\ Usa poi lo XOR, la chiave è più corta e viene incrementata con il keystream: l'algoritmo pseudorandom è progettato ad hoc, non è il classico pseudorandom. La differenza cruciale con OTP è che la chiave è generata a partire da una chiave k piccola, quindi posso trasmettere k al receiver facilmente. Ma se k è sempre la stessa ho un problema, ovvero encrypto sempre con la stessa key di base. Se una sottostringa si ripete, avrò ciphertext diverso (la periodicità del sistema pseudorandom deve essere molto lunga), ma per lo stesso messaggio ho lo stesso ciphertext, in quanto l'algoritmo pseudorandom  deterministico. Vorrei comunicare la chiave una volta per tutte senza doverla cambiare (come avviene in Wi Fi access point), ho una chiave k piccola ed un keystream lungo, ma non sono IND-CPA.
\subsubsection{Initialization vector}
Ho un plaintext che voglio cifrare, mando un messaggio alla mia NIC in modo che lo encrypti con un algoritmo di tipo stream cipher. La NIC ha una chiave k a lungo termine e quando riceve il messaggio genera una quantità dinamica,che è l'initialization vector (IV); questa quantità può essere truly random. Il seed sarà generato giustapponendo la chiave k all'IV, che mi fornirà il keystream, ovviamente l'IV deve essere diverso per ogni messaggio. Come comunico all'altro end l'IV? Lo mando in chiaro con il messaggio, se lo strem cipher è buono non posso determinare il messaggio a partire dall'initialization vector. Ora il receiver può riprodurre il keystream: fa lo XOR e decripta il messaggio ricevuto; l'ipotesi fondamentale è che il PRNG sia buono.\\ Ho la prova di essere semantically secure se l'IV non si ripete.
\subsubsection{Case study: WEP 802.11}
Wired Equivalent Privacy, standardizzato nel 1997-1999 dagli stessi progettisti di Wi-Fi. Aveva 3 obbiettivi:
\begin{itemize}
\item confidenzialità: proteggere i pacchetti da qualcuno di esterno alla rete, uso dell'algoritmo stream cipher RC4 (poi scoperto vulnerabile, ma è n'altra storia).
\item integrità: il pacchetto non doveva essere modificato lungo il tragitto.
\item: autenticazione: voglio che qualcuno possa entrare nella rete solo tramite delle credenziali.
\end{itemize}
\subsubsection{RC4}
Algoritmo PRNG specifico, usato per generare il keystream. Oggi è considerato debole, ma comunque WEP avrebbe avuto gli stessi problemi anche se fosse stato buono.\\ ENC(KEY,MSG) = MSG $\oplus$ RC4(KEY,IV)\\ L'IV va generato per ogni frame e deve essere diverso per ognuno di essi, inoltre lo stream cipher deve essere sincronizzato in un canale che ha perdita. L'IV viene trasmesso in chiaro, se lo stream cipher è buono è buono non ho problemi. WEP è sicuro se l'IV non si ripete, altrimenti userei la stessa chiave e non avrei semantic security.\\ In Wi-Fi è "semplice" attaccare con Chosen Plaintext Attack o Known Plaintext Attack, anche se non conosco i messaggi ma li vedo in XOR posso ricavare qualcosa, l'IV è quindi cruciale e in WEP furono commessi due errori:
\begin{itemize}
\item La taglia era di 24 bit, molto piccola: circa 16.7 milioni di encryption diversi, se assumo 1500 byte di trama, con 7 Mbps di thoughput $\Rightarrow$ riciclo dopo appena 8 ore.
\item L'implementazione fu lasciata libera $\Rightarrow$ COSA DA NON FARE MAI, MAIIIII M A I (MAI PIÙÙÙÙÙÙÙÙÙÙ NON NOMINARE MIA MADRE CIT*), potrebbero metterci tutti 00..0 se non leggono la specifica.
\end{itemize}
Inoltre, conviene generare l'IV random o in maniera sequenziale? Se lo genero random, ho il 50\% di probabilità di avere un duplicato dopo circa 4000 frame (birthday paradox).Meglio quindi sceglierli in serie, però sono suscettibile ad un attacco: se il router viene spento e riacceso, la sequenza riparte da 0. L'attacker può catturare i messaggi, rebootare di nuovo e fare un Chosen Plaintext Attack, ricreando la sequenza degli IV.\\ Il reboot dovrebbe prevedere un seed sempre diverso, ma qui il generatore è PRNG.\\ L'attacker può quindi creare un dizionario:\\ per ogni IV avrà il corrispondente keystream = RC4(IV,K), così da poter usare la coppia per attaccare (manda un contenuto noto ed una volta ricevuta la risposta ricava MSG $\oplus$ keystream $\oplus$ MSG ed ottiene il keystream).\\ Se RC4 è buono, non deve essere possibile ricavare una entry del dizionario avendo tutte le restanti. Un altro attacco può consistere nell'aspettare che l'IV si ripeta.
\subsubsection{User authentication}
Autenticazione: mostrare davvero chi sei. Non va confusa con l'identificazione, con cui fornisco nome cognome etc..., l'autenticazione è la prova che controllo la mia identità digitale.\\ Non è semplice definire l'autenticazione, molti siti difatti permettono di creare ad esempio mail che non mostrano il mio nome e cognome e quindi questo non mi identifica, ma voglio comunque che l'account sia usato da una sola persona.\\ Metodi di autenticazione:
\begin{itemize}
\item Metodo "base": una password, un pin, chiave segreta etc...
\item device fisici: smart card, token digitali, hardware non clonabile.
\item biometrics: impronta digitale, retina etc...
\item behavioural authentication: registrazione vocale, hand writing etc...
\end{itemize}
In WEP non fu prevista l'autenticazione di ogni singolo utente. L'obiettivo era quello di riuscire ad autenticare un gruppo di persone che potessero entrare nella rete. L'idea: chi sta nella stessa rete può essere visto dagli altri, quindi usa los etsso meccanismo di enrcyption.\\ Il grant di accesso era dato solo a chi aveva una password comune, pre-distribuita. Come provare l'autenticazione: non posso inviarla in chiaro (sono in Wi-Fi), quindi in WEP venne introdotto un meccanismo che prevedeva di effettuare delle operazioni sulla password; il risultato non doveva dare informazioni sulla password.\\ Meccanismo: conosco k, l'access point mi manda una challenge ed io gli fornisco un encryption della challenge e della password.Per ogni nuovo utente devo usare una challenge diversa, può seere una stringa in plaintext, in WEP era di 128 bit. Una volta ricevuta la risposta, l'AP decriptava e se il risultato era la k dava l'accesso.Tecnica symmetric key, buona? Sì, trovo in un libro scritto da gente top nel settore che mi descrive esattamente questa tecnica, se la challenge è random e senza ripetizioni sono al sicuro.\\ In WEP non è così,anzi l'autenticazione aiuta a violare la confidenzialità: come detto sopra, posso effettuare un Known Plaintext Attack per creare un dizionario IV | keystream = RC4(K,IV). Quello che vedo nel messaggio è chipertext = plaintext $\oplus$ keystream, devo conoscere il plaintext. WEP fornisce la possibilità di un KPA con l'autenticazione: CT $\oplus$ challenge = RC4(k, IV) = keystream. L'approccio è corretto, ma viene riusata la stessa chiave per cifrare la challenge ed i messaggi. La challenge inoltre è in plaintext $\longrightarrow$ nota $\longrightarrow$ Known Plaintext Attack.\\ Attacker si finge l'access point ed inviando challenge false costruisce il dizionario, una volta ottenuto il keystream (user mi manda challenge $\oplus$ keystream, io ho la challenge, faccio $\oplus$ ed ottengo il keystream) posso usarlo per criptare la challenge successiva e ottenere l'accesso.\\ L'autenticazione era certificata come robusta, ma l'implementazione non lo era, inoltre l'IV era lasciato all'implementatore $\Rightarrow$ MAI FARLO.\\ Il fix fu di far scegliere sia la challenge che l'IV all'acces point, ma in ogni caso essendo l'IV corto si sarebbe ripetuto.
\subsection{Integrità}
Per l'integrità, l'idea fu quella di utilizzare CRC-32, il controllo a lvl2, come integrity check. Non è certo però che funzioni, ma l'attacker vedrà solo il ciphertext, quindi anche se il CRC-32 non è buono è protetto dall'encryption: assunzione errata. Confidenzialità non garantisce integrità. CRC-32 è lineare rispetto allo XOR: se faccio CRC32(A) e CRC(32) di B (con A e B due messaggi diversi) fare CRC32(A $\oplus$ B) = CRC32(A) $\oplus$ CRC32(B). Inoltre, lo XOR era proprio usato per l'encryption $\Rightarrow$ deadly. Ogni messaggio può subire modifiche o injection: \\
ho un plaintext M di, cui l'attacker vuole flippare 3 bit, ho CRC32(M), ed ho M $\oplus$ RC4(K,IV). Produco un messaggio $\delta$ che è uguale ad M, ma con i 3 bit che voglio flippare pari ad 1, computo CRC32($\delta$), prendo il precedente ciphertext e ne faccio lo XOR con il mio:\\
keystream $\oplus$ M $\oplus$ $\delta$ | keystream$_{2}$ $\oplus$ CRC(M $\oplus$ $\delta$) (per linearità dello XOR).\\ Ho un nuovo messaggio valido (nelle ipotesi che $\delta$ sia pari ad M), quindi posso eseguire un Man in the middle attack.\\
Dopo WEP ci fu 802.11 in cui il protocollo è WPA (anche WPA2 con AES), venne inoltre eseguita una patch firmware a RC4:
\begin{itemize}
\item IV a 48 bit
\item protezione dell'IV
\item etc...
\end{itemize}
Morale: rivolgersi ad un esperto di crittografia.
\section{Autenticazione in generale}
Le password sono deboli, faccio una panoramica per capire se una password è hard o no. Autenticazione: provo che ho una password, che per ora ritengo analoga ad un segreto (in pratica: un segreto è una stringa random). Se ho 4 bit, ho $2^4$ possibilità, quindi la probabilità di indovinare al primo tentativo è $\frac{1}{2^4}$.\\ Una password è una stringa con meno entropia: se ho una password di N bit, la probabilità di indovinare al primo tentativo è $>>$ di $\frac{1}{2^N}$.\\ Ho 4 problemi maggiori:
\begin{itemize}
\item password overload: gli utenti tendono a riutilizzare le password su più siti
\item restricted charset: 1 byte = 8 bit, quindi 256 possibili combinazioni, ma da tastiera ne ho circa 100.
\item low entropy: la password non è del tutto random, in quanto va comunque memorizzata.
\item predictability: spesso le password sono associate alla vita reale
\end{itemize}
\subsection{Password overload}
Nel 2018, in USA, uno studio ha rivelato che ogni persona ha circa 130 account nel web: il 38\% degli utenti riutilizza la stessa password su più siti. Se scopro una password di un account, posso usarla per accedere su altri siti $\Rightarrow$ cross site break.\\ Il 21\% degli utenti modifica la propria password, ma le modifiche sono predicibili, inoltre il 46.5\% delle password si cracka con 100 tentativi.
\subsection{Restricted charset}
Se ho un segreto di 8 byte, quindi 64 bit, ogni byte ha 256 diverse possibilità, quindi la probabilità di guess al primo tentativo è $\frac{1}{256^8}$. È un numero elevato? \\ Una macchina "ordinaria" può effettuare 66 milioni guess/secondo, quindi il tempo medio per crackare la password è di circa 4431 anni: 1.8x$10^{19}$ tentativi totali, divido per il numero di guess/secondo e divido per 2 (per fare una media), converto in anni.
Le password però non hanno 256 possibilità per ogni byte, inoltre alcune usano solo lettere lower case, o al più numeri. Anche se vengono introdotti numeri e lettere upper case/simboli, spesso vengono messi in posizioni predicibili (es: all'inizio, alla fine, nel mezzo). \\ Sto considerando un attack brute force offline, in quanto proteggere un web server sarebbe possibile, ade sempio bloccando l'accesso dopo il 3° attempt fallito.
\subsection{Low entropy}
Ci sono dei tool dell'information theory che misurano la randomness. Le password non sono quasi mai random. Come misuro la randomness: Shannon entropy:
Entropia H(X) = $-\sum\limits_{i}p_{i}log_{2}(p_{i})$, considero p $>$ 0, inoltre il segno meno serve perché essendo p $\leq$ 1, il log mi da un valore negativo.\\ La quantità viene misurata in bit. esempio: un coinflip di una moneta equiprobabile ha H(X) = $-2 \cdot (\frac{1}{2} \cdot log_{2}(\frac{1}{2}))$ = 1.\\ Per il dado ho $-6 \cdot (\frac{1}{6} \cdot log_{2}(\frac{1}{6}))$ = 2.58. Per un random byte ho $-256 \cdot (\frac{1}{256} \cdot log_{2}(\frac{1}{256}))$ = $log_{2}(256)$ = 8, ma questo solo se i bit sono davvero random, altrimenti ho un valore minore di 8.\\ L'information value di x$_{i}$ dipende da quanto x$_{i}$ è inatteso: minore è la probabilità di un certo evento e più sono sorpreso, l'information content è quindi = $\frac{1}{p_{i}}$.\\ $log_{2}(\frac{1}{p_{i}})$ è una traslazione della probabilità in bit, ad esempio $\frac{1}{4}$ diventa 2 bit. L'information content è misurata quindi come $log_{2}(\frac{1}{p_{i}})$.\\ Definisco l'entropia come l'average dell'information content degli x$_{i}$: H(X) = E[IC(X)] = $\sum\limits_{i}p_{i}IC_{i}$ = $-\sum\limits_{i}p_{i}log_{2}(p_{i})$.\\ Entropia: misura quantitativa per vedere quanto un evento random è predicibile, se pari ad 8 ho un byte perfettamente random, se è 0 è deterministico. Se N = $2^b$ possibili outcome allora b = $log_{2}(N)$, se l'entropia è pari a b, non posso predire. esempio: una moneta truccata con $\frac{1}{4}$ $\frac{3}{4}$ ho entropia pari a 0.81 $<$ 1, quindi è predicibile.\\ Conseguenze: quando trasmetto un bit, trasferisco una quantità minore di informazione, posso comprimere di (1-quantità)\% un file.\\
esempio: genero 3 bit random, ho entropia pari a 3, ma se ci sono dipendenze? Ad esempio se solo il primo è un coinflip e gli altri due sono deterministici, ad esempio prendono il valore del primo ho entropia = 1. Non conta quindi la lunghezza della stringa, bensì la randomness.\\ Nel 1950, Shannon misurò l'entropia di un testo (in inglese), mostrando che il linguaggio naturale è molto predicibile:\\
le lettere che comparivano nel testo non erano equiprobabili, quindi l'information content della singola lettera non è 4.71 (ovvero non ho probabilità di $\frac{1}{26}$ $\Rightarrow$ $-log_{2}(\frac{1}{26}$). Nota la prima,l'entropia della seconda etc... sono in un certo modo predicibili, ogni lettera inglese ha nella migliore condizione 1.3 di information content e 0.6 nella peggiore. Ogni lettera ha un contributo $\simeq$ 2, e quindi generando una password avrò un entropia di circa 2 bit invece di 8.\\ Se ho 10 lettere random:\\
tempo di crack se puramente random = $2^{4,7 \cdot 10}$ = $2^{47}$ attempts, mentre nel caso di password "umana" ho $2^{2 \cdot 10}$ = $2^{20}$ attempts; perdo un fattore $2^27$ $\simeq$ 134 milioni, quindi molto meno robusta.
\subsection{Predicibilità-Dictonary attack}
In realtà, non serve nemmeno fare un brute force attack, ma si possono usare parole note. Faccio un dictionary attack: scelgo una serie di parole comuni in una lingua e faccio try su queste parole.\\ Se riesco a recuperare un set pubblico di password dal web quello brutto e cattivo costruisco il mio dizionario, che può anche essere mirato al singolo individuo (so nomi di familiari, date di nascita, gusti etc...).Gli attacchi funzionano sia online che offline, dove la forza dipende dalla potenza dell'hardware e dalla randomness della password.\\ Alcune statistiche:
\begin{itemize}
\item 25\% delle password è del tipo 123456..., posso pensare anche ad un password sparying: prendo una passowrd e la provo su più account di diverse persone, in verticale (può essere molto efficace).
\item 26\% delle password sono di 6 byte, ne vanno usati almeno 16.
\end{itemize}
\section{Autenticazione password-based vs autenticazione challenge-handshake}
Dopo aver esaminato le password, vorrei un protocollo che mi permetta di usarle per autenticarmi. Ricordo che l'autenticazione è la prova di conoscere un segreto, senza dover per forza rivelarlo. Ho alcune alternative:
\begin{itemize}
\item PAP: mostro la password in chiaro
\item CHAP: alcune informazioni leakate
\item ZPK: nessuna informazione leakate (crypto forte), molto complessi e poco usati nella realtà 
\end{itemize}
\subsection{PAP}
Il protocollo di autenticazione più semplice possibile: mando la password in chiaro, one way authentication. L'utente manda la sua password (insieme allo user id)ad un autenticatore, che fa un check nel DB in cui per vedere se ha una entry user id | password.\\ La password è pre-shared, ma la sto mandando in chiaro e se vengo intercettato è GAME OVER (se il canale permette eavesdropping, se è cablato sono leggermente più sl sicuro).\\ Inoltre non ho nessuna protezione da reply attack: se vengo intercettato, subito dopo l'attaccante può fingersi me, se non enrypto con un algoritmo semantically secure e se non ci sono limiti nel poter ripetere l'autenticazione; inoltre non permetto mutual authentication.\\ Il messaggio PAP è suddiviso in campi specifici a seconda del server a cui mi devo autenticare, se ad esempio ho server PPP: i campi sono espressi in ASCII ed ogni campo ha una semantica, me la studio, prendo il  pacchetto e scopro tutte le info.
\subsection{CHAP}
L'authenticator mi manda una challenge, a cui rispondo con un messaggio contente il mio user id + hash(challenge,password,etc).Proof of knowledge: computazione di un segreto/password, mando una f(password) per dimostrare che la conosco. La funzione deve avere due proprietà:
\begin{itemize}
\item la computazione non deve rivelare il segreto, quindi non devo poter computare $f^{-1}$
\item la funzione f non deve poter essere replicata da un attaccante.
\end{itemize}
L'authenticator mi manda una challenge ogni volta nuova, ovver una nonce. Lo user risponde con userID ed una funzione di challenge, key, etc...(parametri opzionali). La funzione deve rispettare le due proprietà, l'authenticator la ricalcola, dopo aver preso dal db la password corrispondente allo userID ricevuto; la funzione deve quindi essere deterministica.\\ Se la challenge è fresh non sono suscettibile a reply attack, inoltre la password non è inviata in chiaro.\\ La funzione può essere una hash function crittografica.\\ In CHAP è l'autenticatore che controlla tutto il processo: potrebbe accadere che un'attacker potrebbe intercettare la mia sessione kickarmi, sostituendosi a me. Per prevenire ciò, in CHAP è possibile far si che l'authenticator rimandi la challenge periodicamente, per accertare l'autenticità dell'utente. Tutto ciò in PAP non è possibile, ma il grande svantaggio di CHAP è che le password devono essere salvate in chiaro nel db.
\subsection{Hash function}
Funzioni crittografiche di base. Prende qualcosa in input e la riduce in polvere in maniera che sia incomprensibile ed irreversibile. Se ho un messaggio di lunghezza X, Y=H(X) è detto digest ed ha una taglia fissa; H(X) dovrebbe essere abbastanza semplice da poter essere computata su ogni X.\\ Non è sempre detto che le funzioni hash sia crittografiche, alcuni esempi di funzioni non crittografiche:
\begin{itemize}
\item 4 bit parity vector checksum: prendob blocchi da 4 bit e metto un bit di parità sui blocchi. La size del digest (ottenuto giustapponendo i bit di parità) è sempre pari a 4, e la funzione non è invertibile, in quanto l'inversa non è unica
\item modula checksum: spezzo in chunk di interi (valori $\in$ [0,..9]) il mio messaggio, li sommo e ne faccio il mod1000.
\item call center control: devo autenticarmi con username e password, mi richiedono un pin ma non lo mando tutto, bensì solo specifiche cifre.
\end{itemize}
Ogni hash function, anche non crittografica, non è invertibile.\\ Un hash function crittografica prende il testo e lo comprime in un digest di dimensione fissa, ma ha un importante proprietà: anche piccoli cambiamenti producono digest completamente diversi.Deve cercare di approssimare al meglio la generazione di una stringa random.Nel caso di funzioni hash non crypto, un cambiamento minimo è abbastanza prevedibile.\\ Un attaccante non deve in alcun modo ricreare l'hash digest: nel caso di non-crypto hash, cambiando i bit posso ottenere un messaggio diverso che mi fornisce lo stesso digest $\Rightarrow$ collisione. L'attacker non dovrebbe essere in grado di poter ricreare o modificare il messaggio così da ottenere il digest originale. Devono valere 3 proprietà:
\begin{enumerate}
\item Perimage resistance (one-way property): dato y=digest, deve essere computazionalmente difficile trovare X tale che H(X)=Y. Proprietà più forte del non invertibile, la computazione non deve poter essere ricavabile, anche se ho infiniti emssaggi che generano lo stesso digest.\\ Corollario: per essere one-way la lunghezza del digest deve essere grande, non devo potervi fare brute force attacko crypto-analysis.
\item Weak collision resistance: dato X, è computazionalmente difficile trovare un X', che sia diverso da X, e tale per cui H(X) = H(X').esempio: sono un giudice di un tribunale, ho un hard disk su cui ci sono delle prove, lo do ad un esperto per analizzarlo. Come posso essere sicuro che le prove non siano inquinate? Computo l'hash dell'hard disk e lo metto ar pizzo (lo scrivo su un pizzino magari), così che se qualcuno inquina le prove gli do la sedia elettrica, perché vale questa proprietà e non può produrre modifiche tali per cui l'hash è lo stesso.
\item  Strong collision resistance: ci sono funzioni che sono solide per la proprietà 1 ma non per la 2? Sì, ad esempio se considero Y= f(x) = $g^xmodp$: g è dato, p è un numero primo molto grande. Se ad esempio so che 321475 = $3^x$, riesco a ricavare x? Sì, ho che x = log$_{3}$321475, ma se aggiungo il $modp$ non posso più farlo, non è facile computare l'inversa sotto determinate condizioni. Ma non rispetto la proprietà 2: se ho un X, mi basta sommare k$\cdot$(p-1) per trovare lo stesso risultato; la funzione sembra difficile, ma non rispetta le proprietà.\\ Con la strong collision resistance voglio che sia impossibile trovare una qualunque coppia X$_{1}$, X$_{2}$ che collida.
\subsection{Paradosso del compleanno e dimensione del digest}
Voglio vedere come rispettare la strong collision resistance. Considero il birthday paradox: ho k=23 persone in una stanza, voglio associare a ciascuno un hash fatto sul loro giorno+mese di nascita. Probabilità che non ci siano collisioni tra uno dei k e gli altri k-1: $(\frac{364}{365})^{22}$ = 94.1\%. Ma qual'è la probabilità che non ci siano collisioni tra tutti i k: 1 $\cdot$ (1-$\frac{1}{365}$)..... $\cdot$ (1-$\frac{22}{365}$) $\simeq$ 49.3\%. Quindi la probabilità di collidere è il complementare, ovvero 1-0.493 = 0.507 = 50.7\%. esempio: ho n bit di digest, N = $2^n$ diversi risultati. Considero k messaggi:\\ P(no collisioni) = 1-p = $\frac{\frac{N!}{(N-k+1)!}}{N^k}$ = $\frac{N}{N}$ $\cdot$ $\frac{N-1}{N}$ $\cdot$ $\frac{N-2}{N}$ ..... $\cdot$ $\frac{N-k+1}{N}$ $\simeq$ (1 - $\frac{1}{N}$) $\cdot$ (1 - $\frac{2}{N}$) ....... $\cdot$ (1 - $\frac{k-1}{N}$) = $\prod\limits_{i=1}^{k-1}(1 - \frac{i}{N})$.\\ Nelle ipotesi di N grande, 1-i $\simeq$ -i $\Rightarrow$ $\frac{-i}{N}$ $\simeq$ $e^{-\frac{i}{N}}$, quindi ho $\simeq$ $\prod\limits_{i=1}^{k-1}e^{-\frac{i}{N}}$, ma questa è uguale alla somma degli esponenti $\Rightarrow$ $e^{-\sum\limits_{i=1}^{k-1}\frac{i}{N}}$. Ho inoltre che $\sum\limits_{i=1}^{k-1} i$ è la somma di Gauss = $\frac{k \cdot (k-1)}{2}$ e quindi ho $e^{-\frac{k \cdot (k-1)}{2}}$ $\simeq$ $e^{-\frac{k^2}{2N}}$, approssimando k-1 a k. Questa è la probabilità di non avere collisioni: 1-p = $e^{-\frac{k^2}{2N}}$ da cui ln(1-p) = $-\frac{k^2}{2N}$ $\Rightarrow$ k = $\sqrt[2]{-2N \cdot ln(1-p)}$ $\Rightarrow$ k = $\sqrt[2]{2N \cdot ln(\frac{1}{1-p})}$. Quindi all'aumentare del numero di messaggi k, aumenterà la probabilità di collidere. L'obbiettivo è capire quanti messaggi devo raccogliere per avere il 50\% di probabilità di collidere:\\
$\sqrt[2]{N}$ $\cdot$ $\sqrt[2]{ln(\frac{1}{1-\frac{1}{2}})}$ = $\sqrt[2]{N}$ $\cdot$ $\sqrt[2]{2}$ $\sqrt[2]{ln2}$ $\simeq$ 1.177$\sqrt[2]{N}$ $\simeq$ $\sqrt[2]{N}$. Siccome N = $2^n$, avrò k = 1.117 $\cdot$ $2^{\frac{n}{2}}$ $\simeq$ $2^{\frac{n}{2}}$. Se la RAND fosse una perfetta hash function: con 32 bit avrei 4.5 miliardi possibili output, e devo raccoglierne solo 60k per avere una collisione.\\ Per md5, con k = 1.8 $\cdot$ $10^{19}$ = $2^{64}$ oggi è considerato weak, mentre per SHA256 ho 3.4 $\cdot$ $10^{38}$.
\end{enumerate}
\subsection{PAP vs CHAP}
Posso chiedermi quale fra i due è il più robusto. Bisogna comunque avere chiaro l'adversary model:
\begin{itemize}
\item eavesdropping attack: ascolto chi trasmette
\item rubo i dati dal db
\end{itemize}
In PAP, se qualcuno ascolta il canale è finita, perché la password viene trasmessa in chiaro, quindi c'è la necessità di proteggere il canale di comunicazione (SSL, TLS, EAP/TTLS), ma nel caso in cui ci sia lo steal del DB PAP è molto più robusto, in quanto posso salvare le password non in chiaro. CHAP è invece meglio nel caso del 1° attacco, ma nel secondo no: non posso salvare l'hashing delle password, nel caso di PAP può essere effettuato brute force e la riuscita dipende dall'entropia delle password.\\ Perché in CHAP devo per forza salvare la password in chiaro: in CHAP la challenge è sempre diversa, non posso salvare H(psw, challenge) e se salvo solo H(psw),  non posso ricavare la password perché la funzione non è invertibile.\\ Provo a modificare CHAP: faccio l'hash della password on the fly, ovvero faccio hash(hash(psw),challenge) e mando all'autenticatore, che ora può salvare l'hash della password. Ma in questo modo, se il db viene crackato, non devo nemmeno fare sforzi: userò l'hash della password per rispondere alla challenge e mi autenticherò.\\ In conclusione: 
\begin{itemize}
\item Se l'attacco è sul canale di comunicazione, è meglio usare CHAP
\item Se il canale è robusto ma il BD no, meglio usare PAP.
\end{itemize}
Quando valuto la sicurezza di un sistema devo capire bene cosa l'attaccante può fare e come posso difendermi.\\ Un modo per poter mitigare CHAP è aggiungere del "sale": authenticator mi manda la challenge più del salt, io prendo il salt e lo combino alla password e ne faccio l'hash, che uso per fare hash con la challenge.Cambiando il salt, anche il risultato cambia, quindi posso creare un DB con UID e l'hash di (psw,salt).Rimane comunque il problema in caso di db stealing, ma risolvo buttando via il db e ricostruendolo; sono inoltre soggetto a brute force e dictionary attack. Ho bisogno di un DB aggiuntivo,che proteggo in maniera più forte, in cui salvare le password in chiaro per poterlo ricostruire in caso di steal.
\subsection{Hash Chain}
One time password: voglio una password diversa per ogni tentativo di autenticazione, così da essere al sicuro da reply attack. Sembra una cosa triviale: creo uno userDB con una lista di password random, per ognuna metto un flag che mi indica se è stata già usata o no, quando ne ricevo una la segno.\\ Su na scala reale: se ho molti utenti, il numero di password totali è considerevole, il problema non è tanto nella taglia del DB quanto nel dover cambiare la struttura del DB. L'idea è quindi di generare un numero random/pseudorandom di password da un seed. Uso una hash function crypto, come SHA256, a cui passo il seme e computo il digest. Parto da un seme P[0] e genero P[1] = H(p[0]) e così via, devo quindi salvare solo P[0] per poterle computare tutte. Uso P[0], poi P[1] etc..., ma il modo è errato: se riesco a leggere P[0] poi posso generare tutte le successive, quindi faccio il contrario: mando P[n], poi P[n-1]..., l'attacker non andare al ritroso nel calcolo (sto usando una crypto hash function), non è possibile invertire la funzione.\\ L'authenticator computerà da P[0] a P[n], ma su larga scala questo è oneroso: quello che viene fatto è computare offline, ade sempio durante la registrazione, fino a P[n+1] e salva solo questa. Quando riceverà P[n], farà P[n+1] = H(P[n]) e lo confronterà con il valore di P[n+1] che ha salvato. Il numero di password è finito, quindi dopo un po' sarà necessario rigenerare la chiave, per creare una nuova hash chain. \\ Posso avere un problema nel momento in cui ricevo l'ok per l'autenticazione, mando il valore successivo e non accade nulla: posso tentare le altre password, se si perde la sincronizzazione tra client e server, so qual'è ultima password correttamente ricevuta lato server (perché ho salvato P[n+1]) e quindi definisco una finestra di tolleranza per cui provo a fare hash per vedere se mi torna il risultato (la finestra va a salire), ovviamente il valore deve essere limitato.\\ Benefit della OTP:
\begin{itemize}
\item Invio in chiaro
\item Rilasso la server security: l'authenticator salva solo la password che si aspetta di computare, quindi in caso di db steal non ho informazioni sulla password esatta.
\item minore complessità del db
\end{itemize}
Problemi:
\begin{itemize}
\item Dimensionare bene n
\item client side è vulnerabile, in caso di key steal è finita.
\end{itemize}
\subsection{2-factor authentication}
Non posso fidarmi della password dell'utente, quindi spesso viene inviato anche un codice (via sms, mail etc...), in modo che l'attacker deve trovare entrambe per poter avere successo. Il codice è un one-time authentication token generato su un device differente e ricevuto su un canale differente.\\ Deve essere human friendly: un codice da 6 a 8 cifre, possibile generarlo con un hash su cui poi viene effettuato un troncamento...\\ Se assumo che sia client che server sono sicuri, non ho bisogno di usare un hash chain: ho due protocolli possibili, HOTP e TOTP
\subsubsection{HOTP-Hash One Time Password}
Ho client e server sicuri, c'è il segreto shared su entrambe, non computo P[n] = H(P[n-1]), bensì P[n] = H(segreto, n); ad esempio P[35] = H(secret, 35) e così via.\\ Anche se ottengo uno dei P[i] non posso ricavare gli altri se la funzione hash è crypto. Inoltre k può essere "infinito", parto da un counter e non devo precomputare nulla. Uso SHA256(k,n) che è un HMAC, e prendo il troncamento del risultato (6-8) digits.
\subsubsection{TOPT-Time-based One time password}
Sistema più sicuro, in cui non devo generare una OTP, am ha come assunzione che il tempo sia sicuro: ho un time sicuro, parto da un timestamp TS iniziale. TOPT = HOTP(k,T), dove T = $\frac{TS}{X}$, dove X è il range nel quale penso non avvenga un reply attack (solitamente 30 secondi). Questo TOTP ha valore per un certo lasso di tempo, computo il valore ad esempio alle 10:46 ed il server riceve alle 10:46, ma se c'è dello sfasamento il valore non torna. Per incrementare la robustezza posso provare ad usare valori precedenti, tolleranza ad esempio di uno slot temporale.\\ Se il tempo non è sicuro, un attacker può cambiare il tempo e riusare una OTP. Ad esempio: server NTP ha precisione di O(10 ms), quindi non è sicuro.
\subsection{Mutual authentication con CHAP}
Assumo che CHAP sia un protocollo sicuro, però è pensato per single authentication. Provo ad adattarlo per mutua autenticazione. Pro tip: non fare mai quello che segue, non si fa. Esce fuori un casino:
\begin{center}
\includegraphics[scale=0.5]{immagini/chap_mutual.png}
\end{center}
Non è una buona idea: uso la stessa chiave per rispondere alla challenge sia dello user che dell'authenticator, ma invece di essere l'utente il primo ad autenticarsi, chiede all'authenticator di farlo (posso farlo se il canale è full duplex). Se però la challenge è la stessa, non è detto che l'authenticator si renda conto che non è cambiata: quando ricevo la risposta posso compiere un reply attack, perché la challenge può essere uguale. Cambio il protocollo in un unico mutuo protocollo: l'authenticator manda il suo nome e la sua challenge, l'utente genera una nuova challenge, manda all'authenticator la nuova challenge e la risposta; ora l'authenticator può verificare che la risposta sia diversa.\\ Reflection attack: mando la challenge C$_{1}$, l'utente mi manda C$_{2}$ + hash per C$_{1}$. Fingo una perdita di connessione, sospendo la sessione e mando come challenge C$_{2}$, l'utente vede C$_{1}$ e C$_{2}$, ma non sa che deve verificare le due sessioni, quindi mi manda la risposta a C$_{2}$ e la nuova challenge C$_{2}$. Ora fingo che la sessione è tornata up e mando la risposta a C$_{2}$.\\ Ulteriore patch: ogni nuova sessione rende invalida la precedente, posso comunque compiere un man in the middle/intertwining attack. L'attacker prende il messaggio, lo manda all'authenticator e riceve l'autenticazione dal server, quindi fa credere all'utente che è connesso col server.\\ Per fixare: richiedo che venga effettuata una computazione sulla challenge dell'utente e dell'authenticator, ovvero di effettuare crypto binding sulle due challenge. Lo user fa crypto binding su C$_{1}$,C$_{2}$ e si fa mandare dall'authenticator l'hash di C$_{2}$,C$_{3}$, ma così ci sono troppe challenge: l'attaccante invia C$_{1}$, mi faccio mandare l'hash di C$_{1}$,C$_{2}$ posso iniziare una nuova sessione con C$_{2}$.\\ La soluzione prevede che sia l'utente che l'authenticator usino le stesse due challenge, ma scambiando l'ordine con cui viene effettuato l'hashing,ma in questo modo ho progettato un protocollo diverso da CHAP.\\ Avevo due sessioni indipendenti, l'unico modo per renderle dipendenti è usare crypto dependency sulle due challenge.
\section{Nonce}
Una nonce è un valore sempre fresco, ovvero ogni volta diverso.\\ Sono di 3 tipi:
\begin{enumerate}
\item Random challenge: $\sqrt[2]{n}$ in termini di randomness
\item \# seq : più robusto in termini di predicibilità, n.
\item timestamp: è predicibile, ma devo avere garanzia sul tempo (es: GPS e Galileo, GPS può essere spoofato).
\end{enumerate}
La nonce è un superset di possibili challenge, può prevede l'utilizzo di più metodi
\section{Challenge-response authetication in 2G/3G/4G(5G)}
Architettura semplificata di un sistema cellulare:
\begin{center}
\includegraphics[scale=0.4]{immagini/cell_net.png}
\end{center}
La rete cellulare deve garantire almeno queste 3 parti, la serving network offre un servizio di roaming agli altri operatori(non è detto che sia lo stesso operatore dell'utente). l'autenticazione serve perché in questo modo il dispositivo può accedere alla rete e l'operatore sa chi sta accedendo. Usando un protocollo come PAP, la serving netowork vede la mia password e se sono in roaming in paesi molto ad est non so se esistono regole sulla privacy.\\ Non posso fidarmi della serving network, quindi uso un protocollo CHAP-like: la rete è complessa, la parte che gestisce l'autenticazione comunica con la home network.\\ In 3G, il mio device dice alla rete dove sono e chi sono, la serving network contatta la mia home network per ottenere i parametri di configurazione e a questo punto posso autenticarmi usando le credenziali con procedure crypto. In realtà, viene derivata anche la chiave per l'encryption: AKA = Authentication and key agreement, userò anche una chiave successivamente per criptare i messaggi.
\subsubsection{Autenticazione in 2G}
L'autenticazione è unilaterale: la SN manda alla HN l'IMSI che gli comunica il mio device (l'IMSI è il codice della sim) ed una challenge, ovvero chiede alla HN quale risposta deve aspettarsi dall'autenticazione dell'utente. La HN possiede l'id dell'utente e la password, riceve la challenge e ne fa l'hash usando il segreto, producendo l'SRES, fatta da un authenticator trusted: la HN è il ground truth.\\ Inoltre, fornisce la K$_{c}$, ovvero la chiave temporanea che verrà usata dopo l'autenticazione per criptare i messaggi. La SN mi manda la challenge (128 bit random challenge in 2G), io (la mobile station) usa la funzione A$_{3}$ (simile ad un hash function) a cui passa il segreto e la challenge. La risposta è di 32 bit e viene mandata alla SN, che controlla se è uguale alla SRES ottenuta in precedenza dalla HN. Sono protetto, perché la SN non conosce il segreto k$_{i}$ (identity key) dell'utente.C'è anche la fase di key agreement: (k$_{i}$, rand challenge) $\longrightarrow$ k$_{c}$ usata per criptare, schema di symmetric encryption.\\ In wi-fi: dispositivi che si collegano allo stesso AP condividono la stessa chiave di accesso, la protezione è solo dall'esterno, per la protezione interna servono ulteriori meccanismi (TLS,SSL,...) Nel modello appena descritto ogni utente ha una chiave diversa e questa chiave cambia per ogni nuova sessione.\\ La challenge è una nonce, con cui computo k$_{c}$ mediante la funzione A$_{8}$ (anch'essa simile ad un hash function).\\ Se l'utente si collega da più celle, devo ogni volta ripetere il processo descritto sopra, quindi questo è molto lento. L'idea è quella di fornire alla SN un vettore di $<$ challenge, response, k$_{c}$ $>$xN, in modo che la SN abbia N triple e questo porta a diversi vantaggi:
\begin{itemize}
\item viene contattata una sola volta l'HN, quindi pagherò il delay della connessione una volta sola
\item la challenge random è una nonce che deve essere generata propriamente, se la genera la HN e non la SN sono più al sicuro.
\end{itemize}
Quindi lo schema di challenge-response:\\
challenge: RAND | secret: K$_{i}$ | hash: algoritmo A$_{3}$.
Gli algoritmi A$_{3}$ ed A$_{8}$ prendono 128 bit di K$_{i}$ + 128 bit di RAND e restituiscono rispettivamente 32 bit di SRES e 64 di K$_{c}$ (anche detto il segreto di Pulcinella). $2^{64}$ non è computabile, ma $2^{54}$ sì: la chiave K$_{c}$ era di 54 bit, estesi a 64 con 10 zeri alla fine.
\subsubsection{Scelta degli algoritmi}
A$_{3}$ ed A$_{8}$ girano nel chip della sim. Ma io non ho accesso al chip della sim, quindi non conosco l'implementazione di A$_{3}$, ma non ne ho bisogno. Nemmeno la SN deve conoscerla, il risultato sta già nella tripla, è proprio la response, che è l'SRES, quindi ogni operatore può scegliere gli algoritmi che preferisce. Gli operatori scelsero di usare un algoritmo noto, COMP128 che non era open source, ma veniva tenuto nascosto $\Rightarrow$ security by obscurity. CI fu un leak del codice, e questo venne analizzato e violato da crypto guys molt forti in O(min), rendendo vulnerabili milioni di sim.\\ Morale:
\begin{itemize}
\item Security by obscurity non funziona, perché è difficile che se scoprono una vulnerabilità non la dicano.
\item Algoritmo pubblico viene validato dai crypto shark che cercano di romperlo per il clout.
\item Preferisci sempre l'open source al codice chiuso per la sicurezza
\end{itemize}
\subsection{Autenticazione in 3G/4G}
Il problema del 2G sta nel fatto che l'algoritmo per computare K$_{c}$ era vulnerabile, ma non c'era nemmeno mutual authentication: suscettibile ad over the air attack, ovvero una finta base station a cui l'utente si collega. La base station non prova mai la sua autenticità, e non voglio questo. Uso un algoritmo open source per la mutua autenticazione:
la mobile station comunica il suo IMSI alla SN, che lo manda alla HN e riceve una 5-pla $<$ rand, XRES, C$_{k}$, IK, AUTN $>$xN.\\ L'IK è l'integrity key, usata per garantire l'integrità del messaggio. Questo perché l'encryption non la garantisce, quindi serve una chiave diversa dalla K$_{c}$ per garantirla.L'AUTN serve invece per provare l'autenticità della rete all'utente, è il Network Authentication Token.\\ La SN mi manda l'AUTN e la rand, provandomi di essere trusted, ma l'AUTN è stato prodotto dalla HN, quindi voglio che sia il device a produrre la challenge che la SN dovrà risolvere. In questo caso non è così, dovrei avere due nonces, uno dall'MS alla SN ed uno dalla SN alla HN.\\ Funzioni usate:\\
f$_{2}$(K, RAND) = 32 bit di RES\\
f$_{3}$(A$_{8}$ equivalent)(K, RAND) = 128 bit di C$_{k}$\\
f$_{4}$(K, RAND) = 128 bit di IK\\
Tutte dipendono solo dal segreto dell'utente e dalla random nonce, le implementazioni sono note.\\ Sarebbe possibile autenticarsi con un solo messaggi, usando il timestamp:\\
se la mobile station e la SN usano ad esempio GPS, sotto l'assunzione che il tempo sia sicuro, conoscono perfettamente la nonce usando TS.\\ Farò H(TS,K), non posso avere reply attack etc... e con questa forma di nonce evito l'uso della challenge.
\\ La mutual authentication in 3G+ (simile anche in 4/5G) prevede che la MS invii una nonce, invece di usare una quantità random, usa un \#seq number: in questo modo posso sapere quante volte ho fatto un accesso e qualcuno nella rete mi dice che mi sto autenticando per la \#seq + 1 volta. Se mi arriva un messaggio con un \#seq vecchio, scopro che è un reply attack. Se il vece il \#seq è più grande anche solo di 1, va bene (definisco anche in questo caso una tollerance window): i numeri di sequenza della MS e della HN devono essere sincronizzati, perché la SN si autentica alla HN per ottenere il vettore di credenziali.Quindi, quando mi autentico, uso il \#seq come challenge implicita, risparmiando un messaggio: il messaggio che mi fornisce la SN è un'operazione della nonce + k, usando le credenziali fornite dalla HN; in questo modo so che la SN ha le mie credenziali. È un 2-way exchange con 2 messaggi, il problema è che la MS e la HN devono essere sempre approssimativamente sincronizzate; se si perde la sincronia, servono dei meccanismo per ripristinarla.\\ Format dell'AUTN:\\
N° sequenza a 48 bit | AMF: 16 bit di auth and key management | 64 bit di MAC-A, derivato come segue: MAC-A = f$_{1}$(k, SQN, AMF, RAND), la coppia SQN + RAND corrisponde ad un crypto binding, K ed SQN sono la proof of knowledge di k, l'AMF sono informazioni aggiuntive.\\ Una volta connesso alla service network, questa mi manda un TMSI, ovvero un identificatore temporaneo allocato dinamicamente quando mi collego; nelle prossime autenticazioni userò questo TMSI (GUTI in 4G). Il TMSI è meglio per la privacy, in quanto se mi collego usando sempre l'IMSI, posso essere tracciato, invece qui uso una quantità che via via cambia.\\ L'unica vulnerabilità è che la prima volta l'IMSI viene mandato in chiaro, bisogna cercare di usarlo il meno possibile.\\ Mi autentico ed ogni volta che mi ricollego cambia il TMSI, occorre fare molto lavoro aggiuntivo per evidare che i TMSI siano legati fra loro. Devo inoltre proteggere il sequence number SQN: potrei non trasmetterli ed usarli davvero come informazioni implicite, oppure produrre un encryption con la chiave, ma prima del collegamento non la ho: se però sono autentico ed anche la SN lo è, allora vuol dire che entambi conosciamo k, quindi possiamo derivare qualcosa da k. Faccio l'encryption del SQN con un pattern AK apparentemente random: rendo AK computabile per entrambe gli end, quindi poi la base station potrà risolvere:
\begin{itemize}
\item leggi random
\item computa f$_{5}$(k, random) = AK di 48 bit
\item de-anonimizza \#seq : \#seq $\oplus$ AK $\oplus$ AK = \#seq
\item  leggi il resto del pacchetto
\item computa f$_{4}$(\#seq , k, rand, AMF) = MAC-A
\item controlla il MAC-A 
\end{itemize}
\section{Pro-tip: come generare password a partire da un segreto master}
Ho il seguente problema: devo generare chiavi infinite per infiniti utenti: posso usare un PRNG e mettere i record userID | password in un database e proteggere il DB. Pro tip: uso un master secret S, che tengo al sicuro, e quando devo generare una nuova chiave per l'utente uso un identificativo univoco per l'utente (es: il codice fiscale) e faccio hash(S, UID).
\section{Message authentication-Integrity}
In 3G e oltre ho l'IK per l'integrità del messaggio, faccio quindi messsage authentication. Ricordo che la confidenzialità è diversa dall'integrità: con la prima,voglio nascondere il contenuto del messaggio mentre con la seconda voglio che il messaggio non sia modificato durante la trasmissione, solo la sorgente legittima dovrebbe poterlo forgiare.\\ Ho bisogno di un algoritmo che garantisca l'integrità.
\subsection{Message Authentication con symmetric key}
Il sender ed il receiver che hanno una stessa chiave k, ad ogni messaggio aggiungo un TAG = gen\_tag(k, message). Il messaggio ha una size arbitraria, ma il tag deve avere size fissa, quindi userò qualcosa di simile ad una funzione hash. Il receiver riceverà il pacchetto e ricomputerà il tag, confrontandolo con il tag ricevuto. \\ esempio: message authentication code (MAC) è più debole della firma digitale\\
La firma digitale non può essere contraffatta, nessuno può modificare un messaggio firmato da me. Nel caso del tag sender e receiver possono modificare il messaggio, chiunque possiede k può farlo. La digital signature ha la non repudiation property o source authentication. Ad esempio, in una chat multicast si usa la stessa IK, quindi chiunque può produrre un messaggio spaccandosi per me, con la firma digitale questo non può accadere.
\subsection{Message Authentication con hash functions}
Ho mai detto che una hash function può essere usata per msg auth?\\
Nuovo problema: come usare hash function per msg auth e ci saranno problemi (perché non è quello il purpose per cui è pensata).\\
Encrytpion non garantisce integrità a meno che si usi un authenticated enrcyption $\Rightarrow$ AEAD algorithms, Authenticated Encryption Associeted data ovvero un algortimo che fa sia encryption e authentication.\\ In TLS 1.3 (la nuova) hanno proibito di usare algoritmi che non abbiamo authentication, quindi sono AEAD.\\
AES-128 o AES-256 è enrcyption only, AES-GSM o AES-CCM sono auth enrcyption.\\
\subsection{Message authentication with simmectric key}
Prendi msg m, computa con una chiave k nota ad entrambi gli end(nota $\Rightarrow $che è pre-shared)
C'è una funzione che è usata per generare il tag: riceve una size arbitraria e produce un tag di size fissa e possibilmente piccola (non troppo, per birthday paradox).
Trasmetto msg + il tag, message authentication code aggiunge bytes al msg, c'è dell'overhead in quanto non deve aver infromazione(il msg in se è al massimo livello di entropia).
Recevier verifica il tag usando la stessa funzione, nota e condivisa, generando il nuovo tag dal msg + chiave k e lo confronta con quello ricevuto.\\
\subsection{Definizione di sicurezza per Message Authentication Code}
IND-CPA model definiva la confidenzialità, posso trovarne uno analogo per msg auth?\\
Sicurezza in integrity vuol dire che l'atk non può essere in grado di creare un nuovo msg o poter modificarne uno; anche se il msg modificato perde di senso è considerata una violazione.\\
Formalmente, faccio un "gioco" contro l'attaccante:
\begin{itemize} 
\item attacker model può essere Known Message Attack o Chosen Message Attack, ovvero attacker può chiedere qualsiasi coppia (msg,tag) precedente
\item può essere adattivo ovvero che il msg è scelto dopo una analisi della situazione.
\end{itemize}
Ora, se l'attacker scegli uno nuovo messaggio m, diverso da quelli del passato per cui ha i tag, non deve poter forgiare il nuovo tag per il msg m.\\
Formalmente, la probabilità di forgiare una coppia valida deve essere un $\epsilon$(prob dell'ordine di $2^{-100}$):
\begin{itemize} 
\item Devo escludere che il tag sia di 1 byte
\item Non può tirare a caso il tag del msg con scelta puramente random.
Se fosse di 1 byte $\Rightarrow$ avrei $\frac{1}{256}$, che non va bene, il minimo è almeno 96 bit di tag (meglio 256).
\end{itemize}
Differenza cruciale nella scurezza: IND-CPA l'attacker poteva scegliere se il msg era A o B e aveva esattamente $\frac{1}{2}$ possibilità.\\
Message integrity protegge dal man in the middle? Sì, genero il msg m, produco il tag=F(K,m). L'atk intercetta il mesaggio e vuole cambiarlo: se F è cryptographically strong:\\
\begin{itemize}
\item K non può essere computata dal msg e dal tag.
\item Non posso cambiare il tag in un nuovo tag senza conoscere k, non posso computare tag*=G(K,m*)
\item Non posso cambiare m in m' così che F(K,m)=G(K,m')$\Rightarrow$ anticollision property.
\end{itemize}
Se lo schema è sicuro, allora potrò sempre intercettare un mitm atk.Mitm ha due aspetti:
\begin{itemize}
\item networking class: triviale farlo,ARP poisoning, DNS spoofing.\\
\item L'attacco è efficace se posso modificare il msg, non solo cambiare il flow dei msg.
\end{itemize}
Un buon algoritmo deve anche proteggere dalla creazione di un nuovo msg: message spoofing$\Rightarrow$ creo un nuovo messaggio in cui metto un ip fake facendoti credere che è quello con cui vuoi comunicare.\\
Posso risolverlo con un auth mechanism:\\
Se ogni msg è autenticato: DNS è autenticato in plaintext, come faccio a sapere che è proprio, es. Google.com?\\
Devo aggiungere qualcosa che mi garantisce che sia Google ad esempio con un tag.(la versione di DNSSec dovrebbe proteggere da questo, ma questo aggiunge complessità alla rete quindi si continua ad usare DNS.)Posso spoofare msg, ma devo conoscere il tag$\Rightarrow$se algoritmo è buono probabilità è un $\epsilon$.\\
Questo schema NON protegge da un reply attack:\\
voglio mandare due messaggi, es due transazioni. Produco due msg identici, ma la F si applica alla chiave$\Rightarrow$la F deve essere deterministica (va ricomputata all'altro end)e quindi i tag saranno gli stessi,MAC non è abbastanza.Ma se i messaggi hanno un contenuto diverso: timestamp, num. seq etc.. potrei dire che non ci sono problemi.Ma non è così: l'applicazione dovrebbe essere disegnata senza avere in mente problemi di sicurezza.Il protocollo deve essere sicuro, non mi deve importare dell'applicazione.\\
Prevengo reply atk: uso le nonces, devo garantire a livello di protocollo che tutti i messaggi siano diversi, aggiungo una nonce ai msg. Computo il tag sul msg+nonce, posso mandare la nonce in chiaro.\\
\begin{itemize}
\item Se uso seq num: come gestisco i reboot?Devo prestare attenzione.
Parto da 0 e vado avanti, però perdo alcuni n° sequenza, come faccio a dire che i pacchetti ricevuti con alcuni buchi in mezzo sono ok? Devo tenere in mente l'utlimo correttamente ricevuto per discriminare reply atk.
\item Random number, se truly random sono meglio.Non ho probelma del reboot, ma come controllo se pkt è nuovo? Ho un certo n° bit,quindi dovrei tenere tutt la lista de msg precedentemente ricevuti, costo di memoria  e di computazione per il controllo.
\item Timestamp migliore possibile, ma il tempo deve essere garantito o ho problemi.
\end{itemize}
Settare una nonce sembra facile ma non lo è, la maggior parte dei probelmi implementativi è qui.\\
1° ingrediente:\\
Hash functions: molto veloci, sono anticollsione se cryptographic. Buoni prodotti:
\begin{itemize}
\item SHA-2 family(SHA256, SHA224,SHA384$\Rightarrow$SHA512 troncato, SHA512).
Nel passsato SHA1 e MD5, MD5 la più comune e famosa funzione hash,oggi tutte e due rotte.
\item Next: SHA-3 family, sempre gli stessi digest ma con approcci differenti.
\end{itemize}
es: in TLS e Ipsec, SSH non troppo serio si usa SHA256, sha256sum fa hashing di file su Linux.\\
2° ingrediente:\\
Includo il segreto nell'hashing del messaggio messaggio.
Facile? Ma dove metto l'auth key nel msg?Lo metto dopo il messaggio: H(M$||$K), o faccio il contrario?\\
O in altri modi? Ad esempio metterlo sia all'inizio che alla fine etc..Perché me ne preoccupo?\\
Una funzione hash teorica è una black box, c'è anche definizione per la perfect hash function:\\
Random Oracle: black box, che preso input X, H(X) = valore truly random, ma che si ripete se X è lo stesso. Ma le due cose non possono coesistere, H(X) deve essere computabile. Nella teoria questo è il modello ideale (come per one time pad) che vorrei avere, ma non posso implementarlo.\\
Devo vedere nel box:tutte le hash functions (tranne SHA-3, oggi non usate) sono costruite con la costruzione iterative Merkle-Damgard:
è difficile trovare f tale che: f(any size)$\longmapsto$ fixed size. Ma è possibile trovare f t.c:\\
f(fixed size)$\longmapsto$ samller fixed size otuput. Compression function, che possono essere molto buone.\\
es: sha256\\
prendo msg di k bit, paddo il messaggio in modo che il risultato(compreso i 64bit di lunghezza del messaggio) sia multiplo di 512 bit: se ad esempio la size del mio file è 1025 bit, metto un bit ad 1 seguito da vari zeri, alla fine del msg mette la size del msg come lunghezza modulo $2^{64}$, sono 64 bit (faccio il modulo nel caso in cui lunghezza sia maggiore di $2^{64}$, così che sia di size fissa).
Ora taglio il msg in chunks di 512 bit: parto con un initialization vector(non crypto) che è noto e fisso, deve poter essere ripetuto. SHA256 prende IV 256bit,diviso in 8 gruppi da 32, è una costante.L'IV fa sì che la funzione di compressione prenda 512(il chunk) + 256(l'IV) = 768 bit di input.Questo perché SHA256 usa aritmetica $\bmod$32 o 64 a seconda dell'architettura.La compression function comprime i 768 bit in 256 bit che è l'hash summary del chunk 1.\\
Ma ora, se uso questi 256 bit come input per un secondo blocco di compressione, che comprime il chunks 2? SHA256 reitera la stessa funzione di compressione.
La F è il cuore dell'hash function, theorem di Merkle-Damgard dimostra che se F è resistente, ovvero soddisfa le 3 proprietà di una funzione hash$\Rightarrow$l'intera costruzione è sicura.(la F non deve essere lineare)\\La chiave è trovare una buona compression function, questa prende un input fisso e ridà un output fisso, a questo punto posso usarla iterativamente; l'ultima iterazione mi darà i 256 bit finali.\\
In che posizione metto il segreto nell'argomento dell'hash fucntion? Prima del messaggio, o dopo, o in altri modi?
La posizione del segreto conta ed è importantissima:\\
es: msg di 1GB, segreto 128bit, poi ho pad+length.Messaggio è noto, vedo il tag = hash(msg,k), vado da 0 a $2^{128}$ e faccio H(msg,k$_{x}$)=?tag. Brute force atk devo computare fino al massimo $2^{128}$ hash functions.\\
SHA256 è white box, so che è costruita iterativamente, il msg è sempre lo stesso: computo i primi blocchi che contengono il messaggio e prenderò l'output precomputato (i 256 bit risultanti), ed ora dovrò computare solo l'ultimo pezzo a partire dal precomputato.Non quindi computare N$\cdot$ $2^{128}$ blocchi, bensì $2^{128}$ $\Rightarrow$ riduco la complessità d un fattore N.\\Se metto il secret all'inizio, posso rompere la forgiability?Posso forgaire un tag valido per un m\textsuperscript{'} scelto da me, partendo da M,tag=H(s,m). Sì è possibile:\\
triviale forgiare un messaggio autenticato valido
m' != m.
Estendo msg, che può anche essere insensato, con una parte di plaintext.\\
Non posso modificare il msg originale ma non è un problema, inoltre lo faccio senza conoscere il segreto: es. aggiungo una transazione alla fine del messaggio.Aggiungo extra chunks, partendo dal MAC di prima e genero un MAC extended valido.\\
Questo è un problema $\Rightarrow$ ho una funzione forte, ma la costruzione rompe tutto (errore tipico della crypto), quindi non si usa mai una funzione non pensata per quel purpose, anche se i purpose sono simili.\\
La posizione del segreto CONTA TANTISSIMO.\\
Come fixare il problema:\\
Hash Based Message Authentication Code (HMAC), che è stata dimostrata essere sicura come l'hash sottostante.\\
Ho imparato che una secure hash non basta, quindi HMAC aggiunge un modo sicuro di aggiungere segreto nell'hash, non patcho l' hash in se quindi non dipende da come è fatta l'hash.\\
1996, paper di Bellare, Canetti e Krawczyk con due versioni: crypto e IETR RFC 2104.\\
Pluggable hash e usando l'HMAC non aumenti il costo computazionale di molto:\\
HMAC$_{k}$(M)=H(K\textsuperscript{+}$\oplus$opad $|$H(K\textsuperscript{+}$\oplus$ipad$|$M))\\
Il primo pezzo contiene la chiave, i secondo il messaggio. Sembra che sto facendo come prima, ma in realtà sto usando hash del messaggio tra message e secret alla fine. Quindi faccio hash della chiave seguita da hash di message e chiave,come fare 2 volte hash del msg.\\
Se il segreto K è $<$ della lunghezza di un blocco fai si che sia di pari lunghezza, paddo con zeri, ottengo così K\textsuperscript{+}.Questo è il primo chunk di SHA256.\\Per la sicurezza della costruzione, i due segreti che uso negli hash devono essere diversi: miglior costruzione è la nested MAC construction : H(secret$_{1}$ $||$ H(secret$_{2}$ $||$ msg)). Ma chiedere di usare due segreti sarebbe stato un disatro, quindi per praticità non era conveniente lasciare all'implementatore la scelta dei due segreti.\\Soluzione è che produco due segreti dviersi a partire dallo stesso: in entrambi i due risultati flippo bit diversi riseptto all'originale, sembrano quindi due segreti indipendenti(ma non lo sono) ed hanno una distanza larga in termini di bit.\\
es : k = 01010101, inner: 01010101$\oplus$01011100=K$_{i}$, outer: 01010101$\oplus$00110110=K$_{o}$ (entrambe le sequenze ripetute come serve).\\
Parto dal msg, aggiungo all'inizio (prefix) K$_{i}$, runno hash function: parto da IV e lo unisco a K$_{i}$ ed ottengo un secret IV.Hash sugli altri chunks, ed ottengo l'inner hash: ho un classico MAC secret prefix, devo mettergli una pezza: prendo l'oueter key K$_{o}$ e faccio hash del singolo blocco (inner hash + pad).\\Ottengo quindi HMAC, che è dimostrato essere una costruzione sicura.\\
storiella: 2005 md5 broken, tutti gli HMAC tags usavano md5. Thm ti dice che la costruzione è sicura quanto l'hash sottostante: se l'hash è unsecure $\Rightarrow$dovrebbe rompersi anche il meccanismo di HMAC. 2006: non era ancora stato trovato un atck pratico ad HMAC di md5.\\
Assunzioni: modello math dell'hash function:
\begin{itemize}
\item pseudorandom output
\item anticollision property.
\end{itemize}
Entrando nei dettagli, Bellare si rende conto che non usa mai la proprietà 2 e capisce che HMAC è più sicuro dell'hash fuction, finché la properietà 1 non è violata.\\
Paper del 2006 su collision resistance NON necessaria.\\
Messaggi importanti:\\
\begin{itemize}
\item Confidentiality != integrity
\item Message authentication with symmetric key
\item Reply atck: MAC non è abbastanza, servono nonces e vanno gestite bene.
\item Crypto hash functions
\item Come includere key nell'hash fuction? Non è triviale, usa HMAC.
\end{itemize}
\section{Gestione dell'accesso remoto: RADIUS}
Tool usato nel backend, Remote Authentication Dial In User Service, obsoleto:\\
oggi migliori protocolli(DIAMETER) ma ci sono un sacco di problemi quindi è utile studiarlo.\\
Posso accedere alla rete usando diverse tecnologie, tutte eterogenee fra loro e largamente distribuite. Gestire la rete con tutte queste tecnologie ed access point: uso server centralizzato, RADIUS server che è incaricato di gestire username e password dell'utente, così da evitare la distribuzione all'interno della rete.\\Anche una questione di sicurezza:(di solito in AP: Linux machine con db interni), non lascio username e psw distribuite in giro per la rete.\\Devo cambiare l'authentication model: faccio auth con local technology, RADIUS client che counica con l'utene e col server contatta quest'ultimo ed il server ,manda RADIUS response con un si o no a seconda se l'utente può accdere o meno$\Rightarrow$parte più importante.Il client dice quindi all'utente se può entrare o no(l'utente non sa che sta usando RADIUS).\\
\subsection{RADIUS: AAA protocol}
3 servizi di solito eseguiti insieme:
\begin{itemize}
\item Authentication
\item Authorization: da non confondere con Authentication, qui voglio sapere che l'utente ha il permesso di usare il servizio (perché ha pagato o per altri motivi).
Posso avere 
\begin{itemize}
\item authentication senza authorization
\item authorization without authentication ed avrei un servizio privacy preserving.\\Letteratura scientifica è ricca di tecniche per farlo, ma nel mondo pratico non molto usato.
\item l'intersezione fra i due.
\end{itemize}
Serve per management
\item Accounting: transmitted bytes (quanti GB sto consumando), billing, minuti di telefonate spese etc..\\Segno cosa stai facendo in termini di una risorsa che stai usando.
\end{itemize}
\subsubsection{RADIUS è client-server protocol}
Richiesta parte dal client, non confondere il RADIUS client con l'end user, ovvero il NAS: ho end user - NAS- RADIUS client- Server.\\Basato su UDP/IP porta 1812, client port è ephemeral. Sistema centralizzato, logicamente centralizzato: in teoria ho un singolo server ma in pratica è ridondato (sennò è single point of failure)\\In RADIUS si può usare roaming: se cambio città rispetto a dove sta il server, es. della mia università, dovrei cambiare account, ma quello che accade è che la mia richiesta viene presa dal RADIUS server della città e la inoltra al RADIUS server della mia univeristà, agendo da proxy server.\\Architettura complessa, diversi blocchi:
\begin{itemize}
\item Server application
\item User db: per ogni username ho almeno authentication info, authentication method e authorization attributes
\item Client db: clients che possono comunicare col server.
\item  Accounting db: se RADIUS usato per accounting, deve essere aggiornato in real time, per questo separato dal db utente.Non necessario se si fa solo authentication.
\end{itemize}
\subsubsection{RADIUS security features}
Due feature, 1° è
per packet authenticated reply: NAS non ha le mie credenziali, le manda al server, atck interecetta il messaggio e risponde con un "sì", il NAS ora vede che l'utente è autenticato.Non devo pter spoofare il msg$\Rightarrow$deve essere autenticato, ed è quello che è stato fatto: si usa shared secret, approccio CHAP-like,ma:
\begin{itemize}
\item solo la reply è autenticata
\item l'autenticazione è hash based e non HMAC-based
\item funzione hash specifica MD5, quando uso una hash function deve essere future-proof, se metto uno specifico crypto algo in un protocollo è male: qualcuno prima o poi lo romperà. Non è semplice andare poi a modificarlo. Il protocollo è una cosa, l'algoritmo di encryption deve essere messo a parte, così da cambiarlo in caso venga violato.
\item Secret non truly random, ma low-entropy shared key
\end{itemize}
2° servizio: user password enrcyptata: se uso PAP, ho la psw in chiaro. Standardizzazione di un meccanismo. Problemi:
\begin{itemize}
\item Custom mechanism, non inventare algoritmi per quanto possibile, ma usane uno già esistente. (Non era rotto, però devo considerarlo come possibile vulnerabilità).
\item Shared secret key usata anche per l'authentication $\Rightarrow$NUN SE FA, anche se non è exploitabile è errato, perché se rompi la chiave rompi più di un servizio.
\end{itemize}
\subsubsection{RADIUS authenticated reply concept}
End user credentials $\Longrightarrow$ manda le credenziali al NAS, RADIUS client e server hanno uno shared secret che è $\neq$ dalle credenziali del utente. NAS parsa le informazioni e le traduce nel RADIUS language, include le credenziali in un pacchetto RADIUS che è un pacchetto UDP/IP che ha:
\begin{itemize}
\item ID field: mi permette di matchare una richiesta con la risposta.
\item Authentication field: nonce di forma strana, è una nonce che mando al server così che il server possa creare un reply message (sì, no go-on se servono più informazioni) e possa autenticare il pacchetto, ovvero il pacchetto deve avere un authentication tag. In message authentication includevo il TAG (che era HMAC di K + content), qui ho una cosa analoga: ho la risposta, il tag si costruisce combinando l'ID, il valore random usato come nonce ed il segreto pre-shared.MAC=H(ID,nonce,secret).\\ Il reply può anche avere authorization, esempio poter permettere accesso per un tempo limitato.
\end{itemize}
NAS si tiene in un local db l'associazione ID-nonce(authentication). Faccio un check e se mi torna $\Rightarrow$ sono sicuro che il messaggio mi è arrivato dal server e so che non può essere replicato perché l'auth è fresh per ogni nuovo handshake.Ora NAS passa l'informazione all'end user. È una sorta di challenge-response:
\begin{itemize}
\item la challenge è il request authenticator
\item la risposta include anche,una volta validata, il messaggio di risposta.
\end{itemize}
Formato del pacchetto:\\
IP header $|$ UDP header $|$ RADIUS packet: 
\begin{itemize} 
\item byte di codice:
\begin{itemize}
\item 1) sì
\item 2) no
\item . . .
\item 3)access challenge: sta per go-on, non inteso come la classica challenge.
\end{itemize}
\item 1 byte di identifier
\item 2 byte di length per il pacchetto
\item 16 byte di authenticator che deve essere non replyable $\Rightarrow$ unique. Sono 128 bit $\Rightarrow$ $2^{128}$ possibili authenticator, se fosse realmente truly random, avrei avuto probabilità di collidere proporzionale al birthday paradox (ordine $2^{64}$). 
\item Attributi sono triplette di (type,length,value), ogni tipo corrisponde ad un determinato tipo (username, password, framed-MTU, Callback-number)
\end{itemize}
Authenticator field: la parte più importante per la sicurezza.Dovrebbe essere unico ed unpredictable per evitare reply attack. Ha due scopi: nell'access request server per authentication mechanism, nella response è sempre di 16 byte ma viene usato per il TAG. TAG è MD5(Code$|$ID$|$Length$|$RequestAuthenticator$|$Attributes$|$Secret): qui code è codice di risposta, length è la lunghezza del pacchetto di risposta, attributes sono le triplette. Request Authentication si ottiene dall'access request.\\ Access-request di solito contiene 2 classi di informazioni, uno dell'utente ed uno dell'access service device:
\begin{itemize}
\item Username: NAS ha le credenziali, deve mandarle al server
\item Password dell'utente
\item Identificatore del RADIUS client, NAS-IP o NAS-identifier
\item Identificatore della porta a cui l'utente sta accedendo, la NAS-port (se il NAS ha una porta)
\end{itemize}
Access-reject: o ho fallito l'autenticazione oppure non ho l'autorizzazione (esempio: non ho pagato)\\
Access challenge è un go-on message: usato quando server vuole che venga fatto altro: ci sono altri protocolli di autenticazione (esempio: EAP-TLS, EAP-TTLS) in cui devo fare più operazioni, che richiedono più messaggi
\subsubsection{PPP CHAP support with RADIUS}
In una situazione normale di challenge handshake ho user, server: server mi da challenge,rispondo e lui mi dice si o no.\\ Nel caso di user $|$ NAS $|$ server:\\
potrei generare un processo simile, ma se faccio questo devo anche mandare il segnale fisico per far capire che l'utente è attivo: overhead grande, devo "svegliare" l'utente, il NAS deve chiedere la challenge al server e così via.\\ L'utente si sveglia, il NAS genera la quantità random (mi dovrei fidare dell'access point): utente risponde con hash della password e della challenge usando CHAP. Il NAS crea Access-request RADIUS con Username$|$Risposta della challenge$|$ Chalenge$|$Servizio....\\ Ora il server può verificare se il client è autentico e decidere se dargli accesso o no, manda RADIUS Access accept. Nel caso di protocollo CHAP non uso access-challenge message, uso solo Access-request.\\ Vulnerabilità: messaggio del NAS non è autenticato, l'Access Accept non contiene la tripletta di username o psw, è anche vero che la challenge cambia sempre.NAS non può verificare che la challenge era quella vera. Attacco:\\ prendo il NAS, mando una challenge "1234" e user manda reply"$\alpha\beta\gamma$" NAS manda il pacchetto al server ed ottiene Access Accept.\\L'attacker si finge me: prende il pacchetto che ha generato fingendosi me e sostituisce ai campi dell'auth che il NAS gli ha mandato e la sua risposta alla challenge (che è random,tanto non è importante che sia corretta), a quel punto lo invia al server e non è detto che il server faccia un check per vedere se la challenge che il NAS mi ha dato è fresh o no. Attacco al payload del messaggio: rispondo con una coppia di valori precedenti validi. \\ Dal 1998 anche le richieste diventano autenticate, ma non era una cosa necessaria.\\ Problema: posso fixarlo? Potrei pensare di autenticare reply e request, ad esempio fare HASH(request, reply).
\subsubsection{Password encryption}
Se user manda username e psw con PAP: NAS dovrebbe mandare nel RADIUS packet, ma sono in chiaro. La rete in generale, tra il NAS ed il server RADIUS non può essere trustata $\Rightarrow$ tecnica per criptare la password: predo la psw nativa, la paddo per riempire un blocco da 16 byte, faccio MD5(secret$|$RequestAuth), risultato sembra una string pseudoranom, quindi uso una tecnica simile allo stream cipher: il keystream non è prodotto da uno PNRG, ma da un hash function. Psw è messa in XOR con il keystream ottenuto dall'MD5(secret$|$RequestAuth). Se la psw è più di 16 caratteri: posso dividerla in due blocchi e paddarla con lo stesso valore, ma il segreto è lo stesso e c'è una sola nonce $\Rightarrow$ padderei due volte con lo stesso keystream. Computo un keystream differente, usando il cipertext precedente e faccio XOR con i due blocchi in cui divido la password.
\subsection{RADIUS Security Weakness}
Vulnerabile al message sniffing e modification. Access request non è autenticato, il testo è mandato in chiaro quindi ho problemi di privacy.\\ Soluzione non c'è, devi coprire con un altro protocollo (ad esempio TLS), per l'autenticazione usato EAP (Exstensible Authentication Protocol): protocollo che permette di scambiare messaggi di autenticazione, difatti nella specifica si trova EAP-TLS, EAP-AKA, ovvero usi un protocollo di autenticazione con dentro un AEP packet exchange.\\ Message authenticator: ho un pacchetto di richiesta: contiene code$|$ID$|$Length$|$RandomAuth$|$triple(T,L,V) devo aggiungere un TAG, l'idea è di creare una nuova tripla T,L,V in cui il tipo fosse sepcifico per l'autenticazione. Il valore ora è computato con HMAC-MD5, type è 80 e lunghezza 18.\\ Reply atck: evito reply attack al pacchetto RADIUS di base, ma posso fare reply di una richiesta. Il pacchetto contiene una nonce e l'auth TAG, ma questo è un pacchetto valido, quindi posso replicarlo. Per evitare reply attack di request message, il server deve verificare che la nonce sia fresh. Può o non essere un problema: separa la practical explanation dalla vulnerabilità, deciderà chi implementa se questo è un problema o no.
\subsubsection{Dictionary attack to shared secret}
Problema grande di RADIUS, cos'è lo shared secret? Segreto che sceglie il network manager, problema è che è difficile che venga inserita una stringa truly random, ma una stringa a low entropy, ricorda inoltre restricted charset. Spesso un singolo segreto è usato per tutta la rete esempio: Fonera, aggregazione di AP a cui si ci può connettere in roaming. Stesso segreto per 100k+ device ed era triviale (tipo Salute! in spagnolo).\\ Quindi, usare uno shared secret per ogni client (metodo del segreto unico, che conosco solo io e ne faccio HMAC con un identificatore univoco dell'AP).\\ Possibile fare dictonary attack offline:\\ intercetto una coppia (request,response), ho tutte le info per fare brute force o dictionary atck e posso fare precomputation perchè il segreto è alla fine nell'MD5 del response packet.\\ Se richiesta e risposta su due canali diversi, e posso accedere ad esempio solo la request: non serve la coppia. Siccome il segreto è lo stesso mi basta generare uno userID ed una psw arbitrari, so che avrò una risposta con i campi definiti. Prendo la nonce dal request packet, fare nonce con la psw che ho scelto (Chosen Plaintext atck), quindi ho un keystream e posso fare bruteforce con dictionary atk.\\ Attacco alla password dell'utente: parto dl nome della vittima che voglio, metto psw arbitraria, ottengo unser password attribute e la psw encryptata che è scelta da me, pulisco encrypted password ed ho un keystream valido. Suppongo di voler fare brute force di un utente: faccio trial di psw, ma è lento e verrei bloccato dopo un certo n° attempts. Ma così ho trovato il modo di bypassare il server: mi metto dietro il NAS e spoofo tutta una serie di request (non è detto che il server faccia check che la nonce sia diversa) e faccio dictonary attack.
\subsubsection{Poor PRNG implementations}
PRNG è una delle parti più importanti in security.Security in RADIUS richiede un Request Authenticator unico e fresh: ho un req auth di 128bit, 16 byte.\\ esempio, prendo un random generator, lo chiamo rand(), genera 4 byte:
\begin{itemize}
\item Lo chiamo 4 volte.
\item Lo chiamo una volta e riempio di 0
\item Faccio md5() del risultato della chiamata.
\end{itemize}
Quale meccanismo uso? PRNG ha un periodo, rand() ha un periodo di $2^{32}$: periodo è la lunghezza del ciclo affinché non si ripeta lo stesso pattern (in molti casi, per non crypto PRNG il periodo è $2^{n°_bit}$). Mando il RADIUS packet e l'auth req. non dovrebbe ripetersi. Se faccio merge di più pacchetti: il periodo si accorcia, perché abbiamo preso più valori. In termini di entropia, 2 e 3 sono quasi equivalenti: sono sicuro se la nonce non si ripete, l'approccio 1 ha $2^{30}$ come periodo, mentre 2-3 avrebbero la stessa sicurezza.\\ La maggiora parte delle implementazioni di RADIUS usano non crypto PRNG. Se uso PRNG che è buono dal punto di vista statistico, ma può non essere dal punto di vista della sicurezza:
\begin{enumerate}
\item Predictability: non devo poter predirre quale sarà il prossimo valore 
\item Periodo: prima o poi il generatore si ripete, non voglio short cycles.
\item Random generator garantisce valori unici o ripetuti: ci sono alcuni random generator in cui è possibile garantire che se genero blocchi di dati, questi sono diversi. es: AES ha blocchi che non sono ripetuti
\end{enumerate}
Riguardo al ciclo:\\
Linear Congruential Generator: R$_{n+1}$ = (a $\cdot$ R$_{n}$+b), nel caso di rand a e b scelti in modo che il ciclo sia di $2^{32}$, nel caso di questi generatori se conosci un valore li conosci tutti.\\ Mersenne Twister: $2^{19937}$-1 è ciclo "infinito", ma i valori si ripetono.\\ Se so che i valori si ripetono, al sicurezza è $2^{\frac{N}{2}}$, altrimenti è $2^N$.\\ Ora che so che RADIUS ura poor PRNG, mi aspetto che auth request ripete: stesso problema di WEP, posso avere più o meno predicibilità e penso ai possibili attacchi. Sono tutti reply attack: monitoro un certo n° attempts in cui ho utenti validi, ogni richiesta avrà una risposta con delle nonce. Creo tabella:
Auth request nonce$|$Access accept packet. Access accept packet mi da il permesso di accedere al sistema. Creo dizionario, dopo un po' entro nella rete, NAS gemera una nuova access request che può contenere un numero che già è apparso, faccio reply di una risposta positiva e rispondo $\Rightarrow$ ho accesso alla rete.\\ Stesso alla user psw: è encryptata con MD5(segreto, auth), se auth si ripete il keystream è lo stesso $\Rightarrow$ two time pad e posso romperlo. Creo dizionario di request auth$|$user psw $\oplus$ MD5(secret, nonce).Raccolgo il cipher text, quando avrò la ripetizione (di uno stesso keystream con una psw diversa), faccio XOR e ottengo lo XOR fra due password e sfruttando la low entropy le ottengo entrambe.\\ Posso anche spoofare le password, incrementando il dizionario: aiuto un attacco passivo, uso psw finte che conosco $\Rightarrow$ ottengo la mia psw in XOR con il keystream, faccio lo XOR con la psw ed ottengo MD5 e quindi il keystream.\\ Ora se un utente arriva ed il keystream si ripete ottengo la password a gratis.
\subsection{Lezione da RADIUS}
Whitebox pentesting: alcuni siti usano nell'URL  uno SHA256(emailuser, rand()), se provo a loggarmi, faccio brute force per capire qual'è il valore rand() usato: devo creare $2^{32}$ hash (se rand ha questa periodicità), con 66M hash/sec, 1 min e ho enumerato tutto i possibili valori, ora so che se entra un altro utente dopo di me, posso usare il mio dizionario per prendere il valore successivo.\\ Cosa ho imparato da RADIUS:
\begin{itemize}
\item Do-it-all-in-one non ripaga: un protocollo applicativo non dovrebbe includere sicurezza.\\ Come rendo scuro un sistema? Sviluppo un protocollo apposito, come ad esempio TLS, e lo uso per rendere sicuro un protocollo non sicuro.
\item AAA protocol non dovrebbe implementare un meccanismo proprio, inoltre non includere algoritmi nel protocollo.
\end{itemize}
Attualmente: DIAMETER per migliorare RADIUS.
\subsection{AAA evolution: beyond RADIUS}
Quando parto da una soluzione, meglio cambiare poco.\\ RADIUS deployato anni fa, oggi RADIUS è standard de facto per AAA, spesso anche usato in Wi-FI, supporto universale per i device vendors.\\ Buon protocollo, ma con limitazioni funzionali:
\begin{itemize}
\item scalability: quando fu deployato c'era pochi utenti, ma ora sono molti. UDP è unreliable, potrei avere problemi di loss
\item diversità nelle tecnologie di accesso: prima dial up, ora Wi-Fi, 3G,4G etc..., devo supportare tutte: type$|$len$|$value era non sufficiente, 1 byte di type troppo poco.\\ Lista di possibili estensioni: più di 256 combinazioni, servono più byte di type.
\item interoperabilità: issue importante, ho un server central, ma non è realmente centralizzato (solo logicamente centralizzato), ho delle repliche ed è distribuito.\\ Tutti i server considerati proprietari, quindi difficile avere interoperabilità.
\end{itemize}
Nota di scalabilità: mando RADIUS request e RADIUS accept una volta per connessione, quindi traffico è irrilevante per la scalabilità. esempio: ho NAS a 48 porte, ogni 20 minuti ho in media una nuova combinazione: $\frac{1}{20}$ $minuti^{-1}$ ma $\cdot$ 48 = 2 call/minuto. Ma se numero di NAS aumenta, tipo a 10000: 400 request/secondo. Dopo access request ho anche accounting request e delivery $\Rightarrow$ traffic può arrivare a diversi Mbps, quindi se scalo può diventare difficile da gestire, potrei avere problemi di packet loss.\\ Quando dimensiono un sistema, di solito uso average load, ma non si considerano casi speciali: esempio, ho un crash e il device si reboota. Tutti i device rebootati mandano peek traffic: posso avere molta perdita, RADIUS non scala bene per colpa di UDP, ora serve reliability e quindi TCP o meglio.
\subsection{IETF evolution}
\begin{itemize}
\item Diameter: iniziato nel 1998, ora completato. Attività mosse in DiME(Diameter, Mainteinence, and Extensions WG).\\ AP a casa: PPPoE/PPPoA: protocolli per patchare la possibilità di far girare PPP su ethernet o ATM, doveva durare poco, ma attualmente alcuni lo usano ancora.
\item RADext: path a RADIUS, usato fino a che Diameter fosse diventato mainstream.
\end{itemize}
RADIUS: molto lavoro già fatto, pesanemente integrato, standard de facto. Diameter: protocollo nuovo, più potente e scelta perfetta per nuovi scenari.\\ Li tengo entrambi: lavoro duplicato ma è conveniente a livello di business, se qualcosa va male in Diameter ho backup che è RADIUS.
\subsubsection{DIAMETER}
Simile ad """un object-oriented "protocol design" """ (non dirlo alle persone), ho classe base da cui derivo classe derivata. Primo protocollo inventato come OO: DIAMETER non è un AA protocol, ma un protocollo di messaging/signaling generico a lvl applicativo.Definisco il DIAMETER base protocol: ho le primitive per supportare messaging/signaling transport.\\ Definisco il protocollo per trasportare i messaggi, lo faccio in un altra classe base che è AAA Transport profile(SCTP, TCP-based), deve essere reliable.\\ Ora derivo le specializzazioni: creo una applicazione DIAMETER differente per ogni uso necessario:
\begin{itemize}
\item DIAMETER mobile IPv4 app: per muovere IP
\item DIAMETER NAS app: questo è per il purpose di RADIUS.
\item DIAMETER credit control app
\item DIAMETER EAP app
\item DIAMETER SIP app
\end{itemize}
Nella base: tecniche per keep alive server, load balance etc.., eredito le proprietà e specializzo per lo specifico porpuse dell'applicazione.
\subsubsection{DIAMETER improvements}
SCTP: perché TCP può non essere buono.\\ esempio: ho un NAS che deve parlare col server AAA. Quando arriva connessione, devo settare comunicazione, NAS manda request al server. La voglio reliable: soluzione base è setuppare TPC connection per ogni connessione: devo fare il 3-way handshake, mando il pacchetto e poi aspetto ack e mando il FIN. Tutto st'accrocco per un singolo pacchetto.\\ Non faccio TCP conn per ogni connessione. Uso una singola connesisone TCP per gestire tutti pacchetti: implemento multithreaded server: ho due thread, uno di questi si blocca. Vorrei poter gestire il secondo pacchetto, ma TCP fa consegna ordinata, quindi non posso creare gap nel protocollo: TCP mi da tutto in ordine, è uno dei goal. Ho un flow multiplo embeddato in una singola connessione TCP e quindi in un singolo flow.\\ Vorrei una connessione reliable che porta stream differenti, e vorrei un protocollo che garantisce che tutti i pacchetti nello stream siano letti nel giusto ordine, ma non di avere ordine fra gli stream: se il primo stream si blocca, vorrei bypassarlo e leggere il secondo. Effetto Head of the line blocking: anche se ho uno switch ad alta capacità, l'effetto impatta sulle performance.\\ Secondo problema di TCP: ho un NAS, per reliability ho una interfaccia ethernet, ma posso averne una di backup, ad esempio di backup (in altra tecnologia) così da garantire continuità. Ma l'IP address delle due connessioni è diverso e TCP socket usa la 4-pla: quindi se link fallisce devo inizializzare una nuova connessione (MPTCP lo risolve), vorrei supportare multi-homing: manage più IP address.\\ SCTP (Stream Control Transport Protocol) da queste due garanzie, protocollo migliore per canale di signaling dove trasmetto più flussi di dati. \\ storiella: perché se funziona così bene uso ancora TCP: chicken/egg problem. Standardizzazione bloccata dopo gli anni 2000, ad esempio anche IPv6. Problema dell'Internet Ossification: 1980/1990, con design spirit di Inernet che era End-to-end principle: nella rete telefonica originale, intelligenza nei device centrali, edge stupidi, in Internet la maggior parte dell'intelligenza ai bordi.\\1995-1998-2000: web, avvento di device più intelligente, NAT primo device necessario per far fronte al lack di device alla rete,  ma poi firewall, media converter ed altra roba messa sopra. Tutta una serie di middle boxes. Anche device come TCP accelerators, performance enanchements, etc... "intelligent devices". Arriva un nuovo protocollo: SCTP, comincio a connettere siti distanti, ma siccome traffico gira su questi device che lo reputano non noto, viene bloccato.\\ Servono middle boxes per supportarlo, ma i produttori lo fanno solo se l'evidenza mostra che è usato, ma come cazzo ti mostro che è usato se mi blocchi il traffico. Ora software networking: middleboxes diventano software e sono controllabili, 5G è rete softwarizzata.\\
\begin{itemize}
\item Reliable transport: uso SCTP, senno TCP se non posso
\item Standardizzazione in caso di errore: se server fallisce, come migrare verso altro server. DIAMETER: standard per scoprire queste situazioni:
\begin{itemize}
\item duplicate detection
\item controllo di ritrasmissioni a livello applicativo
\item fallimento di peer. DIAMETER è protcol p2p, server può startare comunicazione esplicitamente con NAS.
\end{itemize}
\item pacchetti PING-like per testare se il device è attivo o no.
\item Estensione dei limiti funzionali: header di RADIUS era corto, quello di DIAMETER è più complesso:  
\begin{itemize}
\item length di 3 byte
\item 3 byte comando ma anche ID per la specifica applicazione.\\ Id del pacchetto serve per matchare request-response, lo scenario nel mondo reale non è solo point-to-point: ho multi-hop, ogni pacchetto averà un ID specifico. Non so se voglio matchare comunicazione globale o locale, quindi DIAMETER introduce due identificatori:
\begin{itemize}
\item Hop-by-hop ID
\item End-to-end ID
\end{itemize}
\item altri flag: NAS può rispondere o far partire la comunicazione, sono in p2p: devo identificare se pacchetto è request o response, uso flag R, flag P sta per proxable e permette di specificare se il pacchetto può essere modificato da un proxy, flag E: messaggio di errore, flag T: messaggio potenzialmente ritrasmesso. esempio: \\ mando un pacchetto, non ricevo answare e retx. Ricevo risposta: se server era bloccato temporaneamente, potrebbe rispondere di nuovo: uso T flag, così da darti avvertimento.
\item AVPs: i vecchi T$|$L$|$V triplets, ora chiamati attribute,value,pairs. Codice , lunghezza e attributo ma anche altro: metto vendor ID che dice che il linguaggio non è di DIAMETER, ma è customizzato.4 byte di AVP code, perché uno era troppo poco.\\ Flags:
\begin{itemize}
\item V: vendor specific
\item M: sono NAS e supporto DIAMETER v4.2.1, server DIAMETER v3.9.8, NAS vuole usare un attributo della nuova versione, che server non ha. Ricevo packet, con attributo che non comprendo: se l'attributo è importante, non posso skipparlo. Conviene droppare packet e dire al NAS di non aver capito. M serve per dire di rimandare indietro, perché le info non comprese sono mandatory.
Risolvo interoperabilità.
\item P: se c'è encryption o no 
\end{itemize}
\item ho il campo dati
\end{itemize}
\end{itemize}
Mangement di intermediate entities: posso usare RADIUS agent per mandare relayed message,ma non c'era supporto al romaning.\\ Non devo solo fare roaming data, ma anche routing: se mi collego ad un peer distante, quello deve fare route al server del mio paese. 3 cose strandardizzate:
\begin{itemize}
\item Nessuno agent intermedio
\item Relay agent: sono a Roma, ma vengo dall'univeristà di Parigi, RADIUS server di Roma riceve la request, relay agent guarda al realm, ovvero il dominio di registrazione e nella sua routing table sa di dover mandare la richiesta al server RADIUS di Parigi.
\item Proxy agent: simile al relay, ma assumo che può modificare il messaggio: se ho un messaggio con TAG integrità, se uso proxy e modifica $\Rightarrow$ ho rotto tutto, è un MITM attack (proxy non ha la chiave per modificare il messaggio).
\end{itemize}
Eduroam: sistema che permette di roammare lungo le università confederate con eduroam e non richiede di avere credenziali specifiche. Cosa fa un utente che sta a Tor Vergata e vuole accedere, ma è di Malta: server parsa l'address e mi manda a Malta.\\ M se le confederazioni sono molte: che succede se il server di Malta è va giù e viene sostituito con nuovo server che ha cambiato ip? Deve comunicarlo a tutti gli altri, anche se aggiungo un server. C'è management nightmare: se uso relay agents o proxy agents c'è problema: le routing table sono embeddate, soluzione: centralized controller che tiene le routing tables. Tizio di Malta entra da Tor Vergata, va a server di Tor Vergata e server prima di inoltrare richiesta, chiede l'ip di Malta al centralized server: il dato rimane nel RADIUS server di Roma, ma il controllo è demandato al controller,posso usare trick di caching. Separazione fra controllo e dati $\Rightarrow$ redirect agent: gestisce solo le routing table, ora le routing table non sono embeddate nell relay agent, quindi ho le due operazioni separate.\\ Questo rende ad esempio possibile number portability
\section{Transport Layer Security (secure socket layer)}
Analisi approfondita di TLS(/SSL) (il più famoso insieme a , SSH ed IPsec).\\ Disegnato per sicurezza di altri protocolli, ma c'erano problemi anche qui.2 obbiettivi: analisi dedicata di TLS nei dettagli, capire come fare design di un protocollo di sicurezza long-to-live.
\subsection{Introduzione a TLS}
Background storico di SSL/TLS:
\begin{itemize}
\item 1993, esplosione del web: Mozilla rilascia il primo browser e web diventa servizio reale. Subito dopo ciò, si ci rende conto che la sicurezza poteva diventare un problema cruciale.
\item 1995: SSL v2 integrato in netscape 1.1, ma fu rotto quasi subito dopo la release.
\item Capiti gli errori di SSL v2, c'è SSL v3. Approccio molto ben fatto in principio che è attulamente la base della versione attuale di TLS (oggi SSL v3.4 -> TLS v 1.3). HTTPS = SSL = TLS: SSL era il nome commerciale originale, era proprietario, che fu poi standardizzato da IETF e il nome fu cambiato in TLS, in modo particolare TLS v1.0 = TLS v3.1.
\item 3 grandi momenti: 
\begin{itemize}
\item TLS v1.1: 2006, problema serio, perché TLS era stato disegnato pensando al web security, ma così stai facendo un'assunzione implicita sul protocollo di trasporto che usi ovvero TCP. Viene fuori che TLS usa le assunzioni di trasporto reliable, quindi se messo su protocollo unreliable non funziona più.
\item DTLS: standardizzato in parallelo a TLS nel 2006, versione di TLS adattata per girare un un transport protocol non reliable: preso TLS con pro e contro e modificato un minimo per farlo girare su un protocollo unreliable.
\item TLS v1.2: la versione usata oggi,anche se ce n'è una nuova. Corretto errore originale di TLS: protocolli vs algoritmi, il protocollo non dovrebbe contenere nessun crypto algorithm hardcoded. Se l'algoritmo viene rotto $\Rightarrow$ il protocollo è rotto. Devi disaccoppiare per avere un long-time-live protocol.\\ Nella parte pseudorandom del protocollo si dipendeva da MD5/SHA-1 (rotti dal 205 in poi). TLS v1.2 disaccoppia e supporta algoritmi per authenticated encryption.
\end{itemize}
\item TLS v1.3: modifiche importanti, solo ciphers AEAD accettati, primo protocollo della famiglia TLS che garantisce la perfect forward secrecy.
\end{itemize}
\subsection{SSL/TLS: layered overview}
I due protocolli principali in newtowking runnanno a livelli diversi: IPsec garantisce sicurezza a livello 3 e runna sopra IP, quindi prima del layer trasporto e protegge anche il protocollo IP. Con IP sec ho anche integrity protection per pacchetti IP. Non ho nozione dello specifico layer di trasporto, sono il payload del pacchetto IPsec.\\ Mentre TLS ha obbiettivi diversi: nato per proteggere il servizio web, idea era prendi HTTP, fallo girare su un protocollo di sicurezza che era intermedio tra lvl 5 e lvl 4: TLS o DTLS. TLS quindi gira sopra il protocollo di trasporto ma non protegge il protocollo di trasporto. TCP rimane non protetto: mentre IP può garantire integrità del pacchetto IP (se ben configurato), TLS non lo fa: il pacchetto TCP può essere modificato. Un attacker può modificare la parte del TCP header, mentre la parte in cui c'è confidenzialità ed integrità è il messaggio HTTP (che è protetto da TLS).\\ Quello che TLS fa esattamente è proteggere esattamente il payload di TLS, ma non il resto. esempio: \\VPN spesso create con IPsec, ma se uso open VPN: \\
ho il mio indirizzo IP, uso TCP/UDP ed uso TLS, poi di nuovo IP TCP applicativo. Tunneling: l'intera pila poggia su una pipe virtuale.\\ TLS esempio perfetto di protocollo di livello di sessione (pila OSI).\\ TLS non è necessariamente limitato al layer di trasporto, usato anche per altri motivi, anche per esempio EAP-TLS: autenticazione basata su TLS e protezione d'integrità su EAP v1.2. EAP-TTLS: creo tunnel su TLS e dentro scambio le credenziali, mentre in AEP-TLS uso TLS per scambiare certificati.
\subsubsection{Application support}
Socket per HTTP: creo connessione in cui specifico Ip addr, port number. N° porta è l'identificatore del servizio che lancio sulla mia macchina. (HTTPd su porta 80).\\ Server avere un approccio per livelli, dove ogni livello vede solo quello sopra e quello sotto e non più di uno: \\ vorrei usare HTTPS, potrei aprire la porta 80, ma ora potrei usare TLS e specificare che TLS usi la porta 123. Approccio corretto per fare incapsulamento sarebbe: TCP specifica che nel pacchetto ha un pacchetto TCP, dentro TLS specifico nell'header che l'application number è 80. TCP $->$ TLS $->$ HTTP. non fu fatto per ragioni storiche: nel 1993 nessuno pensò di usare TLS al di fuori del web. Approccio: ho HTTTP, se voglio usarlo uso porta 80, se invece voglio usarlo tramite TLS usa la porta 443, c'era un numero di porta dedicato. Perché una brutta idea: se uso per un servizio differente? TLS non ha un numero di porta specifico,se voglio usare ad esempio POP3, che usa porta 110 direttamente su TCP ma su TLS non posso usare la stessa: ora ne devo standardizzare una nuova; porta 995 che è SPOP3. Devo standardizzare una nuova porta ogni volta che voglio supportare un nuovo protocollo di livello 5: Duplication of port, non secure vs secure.
\subsubsection{Confronto con IPsec}
Tutto più clean: ho il classico header IP, ho un field di protocol che dice cosa c'è nel pacchetto: 6 per TCP, 71 per UDP, quindi IPsec standardizza 51 (ci sarà un header aggiuntivo per IPsec).Quindi, se ho IPsec connection e vedo pacchetto IP non so che protocollo sto usando: vedo protocol 51, l'avversario che guarda al mio traffico non sa che protocollo sto usando a lvl 4/5, meglio dal punto di vista della privacy. Ho un servizio detto traffic flow confidentiality: un avversario che guarda il flow di pacchetti non deve poter sapere cosa fai in termini di protocollo o applicazione sto usando (se è criptato); in TLS questa cosa non è vera: ho protezione sul payload ma non sull'header TCP da cui posso carpire le informazioni che mi servono.
\subsection{Obbiettivi di TLS}
Fa due cose allo stesso tempo
\begin{itemize}
\item Nel settare una connessione sicura (secure session in TLS) tra due end devo fare due cose: creare la connessione in se, faccio signal e setup della sessione sicura. Fatta dalla TLS handshake:
\begin{itemize}
\item Quale algoritmo encryption uso, devo essere d'accordo con l'altro end. TLS non specifica o hardcoda un algoritmo specifico. per encryption
\item crypto keys che verranno usate: in TLS v1.2 cambiano in ogni connessione (asymmetric crypto). Qui i segreti sono sharati on the fly con tecniche avanzate.
\item Voglio essere sicuro che sto parlando con il server corretto: authentication.
\end{itemize}
\item La seconda fase prevede il trasferimento dati: ho una chiave, ho definito il crypto algorithm, ho definito algoritmo di integrità, prendo TCP segments e li encrypto per trasferirli all'altro end.
\end{itemize}
Sarebbe meglio avere questi due due fasi implementate in due protocolli diversi: non dovrei fare le operazioni allo stesso tempo. Setto un'associazione sicura e la uso quando ne ho bisogno: questo viene fatto da IPsec, non proteggo traffico on demand ma è persistent. Approccio di TLS meno flessibile: in ogni connessione devo fare entrambe le fasi.
\\ esempio: applicazione di TLS su IoT: voglio un protocollo come TLS ma il sensor device va a batterie. Se devi far girare un'istanza di TLS per ogni connessione, la batteria va giù. Split di TLS in due pezzi, separata la parte dell'handshake dalla parte dell'invio: creo sessione una volta e trasferisco i dati. In TLS v1.3 ha risolto il problema.
\subsection{Protocol stack TLS}
TLS gira su TCP/UDP. TLS wrappa i dati in un TLS Record Protocol, formato per scambiare dati. Sopra TLS posso far girare qualsiasi protocollo applicativo TLS non lo sa e non gli interessa, lo sa TCP per via della porta usata), nel TLS RP avrò: dati applicativi, c'è handshake protocol (che usa la stessa struttura del TLS RP), Alert protocol e change Cipher protocol; anche protocolli per gestione errore.
\subsection{TLS Record Protocol}
TLS v1.3 cambia parecchio. Voglio capire come di fa il design di un protocollo, quindi capire bene gli errori fatti.
\subsubsection{Record Protocol operation}
Preso dallo standard TLS v1.0, e vale fino a TLS v1.2: application data, taglio in frammenti, comprimo per usare meno banda. Aggiungo integrità (MAC), encrypto e poi mando il pacchetto. 2 grossi errori: da un punto di vista del protocollo sembra tutto ok, ma dal punto di vista della sicurezza ci sono. Gli errori sono usciti fuori dopo, specialmente la compressione fu usata per attacco CRIME (prima compressione e poi encryption è deadly) e nel fatto di fare prima integrity e poi encryption: cambia l'ordine? Irrilevante dal punto di vista del protocollo, am l'ordine conta: problema scoperto nel 2002, fixato, rotto di nuovo poi di nuovo rotto e poi di nuovo rotto e ancora e ancora rotto e rotto e ROTTO: padding oracle attack. Alla fine, TLS rende impossibile fare MAC separato da encryption, ma solo AEAD.
\subsection{Compression}
Application data sono verbose: ASCII, XML, HTML etc... Quindi in TLS penso alla compressione dei dati. Devo farlo prima dell'encryption: perché non posso fare encryption prima di compressione? Entropia: qualcosa può essere compressa solo se l'entropia è minore di 1 bit. Un buon encryption scheme è pseudo-random string con la stessa entropia $\Rightarrow$ non ha senso comprimere dopo.\\ Lossless compression: introdotta da SSLv2. Considerato in TLS v1.0 ma non specificato: unico algoritmo standardizzato era null compression method. 2004: supporto per la compressione in TLS, nei successivi anni la compressione comincia ad essere supportata da vari browser fino all'uscita del CRIME attack: combinazione di compressione ed encryption è deadly (aiuta a decriptare).\\
Possibile solo usare HMAC, l'hash function è scelta dall'utente. La chiave è simmetrica ma non pre-shared bensì generata dinamicamente.
\subsection{Encryption}
Prendo il blocco dati, li comprimo, aggiungo MAC ed ora lo encrypto. Posso usare sia stream cipher che block cipher.\\ Algoritmo negoziato durante l'handshake, l'algoritmo non può aumentare la taglia di più di 1024 byte (ma non succede quasi mai, i blocchi sono di qualche decina di byte).\\ Infine ho il plai text compresso, il MAC tag, encrypto tutto ed aggiungo header: 
\begin{itemize}
\item Content type che informazioni ci sono: handshake messgage, application meassge o alert
\item Major version: 3.1 per TLS
\item Minor version: 3.1 for TLS
\item Compression length: lunghezza della compressione.
\end{itemize}
Cosa manca? Reply attack? Integrità non garantisce da reply attack senza una nonce all'interno. HMAC è funzione del segreto e del message content, poi ho enrcyption e se uso encryption deterministica o l'ho disattivata (non è mandatory). Ma posso fare un reply attack, manca la nonce: può essere vulnerabile a reply attack.\\ Perché non c'è un sequence number? Perché è implicito: quando faccio partire una TLS connection e dico che questo è il pacchetto 0, girando su TCP sono sicuro che i pacchetti arriveranno in sequenza $\Rightarrow$ non serve includere il n° sequenza.\\ Quando computo HMAC: prendo i dati, includo header di TLS ed un numero di sequenza, che non trasmetti: il mio receiver saprà qual'è perché il traffico passa per TCP.\\ Ma ora c'è un problema: TLS funziona solo se uso davvero TCP e quindi per UDP si romperebbe la sicurezza. Per questo in DTLS è stato necessario ri-standardizzare l'header, aggiungendo 8 bytes di seq num.
\subsection{More insights on encryption+authentication}
Come combinare encryption e MAC.3 possibilità:
\begin{itemize}
\item TLS: MAC then ENCRYPT: prendo i dati in plaintext (compressi ma dal nostro punto di vista è paintext). Aggiungo il MAC computato sui dati e poi uso encryption su MAC+dati.
\item IPsec: ENCRYPT then MAC: prendo i dati, li ecnrypta e poi computa il MAC sui dati encrypted.
\item SSH: ENCRYPT and MAC: prende i dati e su un lato fa MAC sui dati ed encrypta solo i dati. Argomento fallace è perché criptare il MAC, che è già una trasformazione crypto.
\end{itemize}
Problema: quale delle costruzioni è la migliore, ovvero assumo che uso un encryption scheme semantically secure (IND-CPA) ed un MAC sicuro (unforgiable), quindi MAC perfetto e dovrei dimostrare che con questi combinati ho ancora le proprietà originali.\\ Perché la combinazione non è più semantically secure (no IND-CPA): ragiono su un esempio con SSH: \\in un plaintext-ciphertext so che posso avere A o B. Do un pezzo di cipher e ti dico di scoprire quale lettera è. Ho la possibilità di scoprirlo con un coin-flip. Ora encrypto A = $\alpha$ ed aggiungo il MAC di A che è x12; poi encrypto B = $\beta$ ed il suo mac y34.\\ Non so cosa c'è nel cipher, ma il MAC lo vedo: coin-flip diventa deterministico. Il MAC di A o di B è deterministico, quindi si ripeterà. SSH quindi rompe la semantic security.\\ La strada giusta è quella di IPsec, quella di TLS è sbagliata.\\ Padding oracle attacks: iniziano nel 2002 e finiscono nel 2016
\subsection{Attacchi a TLS}
\subsubsection{Background su block ciphers}
Differisce dallo stream cipher per quello che riguarda l'operazione di encryption.\\ Prendo i dati e li spezzo in blocchi, applico una pseudorandom permutation che trasforma il blocco in ciphertext. Trasformazione deve essere reversibile: se PT$_{1}$ $\neq$ PT$_{2}$ anche i cipher lo sono.\\ Ma lo stesso plaintext viene cifrato con allo stesso ciphertext, quindi devo anche avere un modo di combinare il plaintext.\\ Come non farlo: applico solo sa trasformazione (electronic code book, ECB) $\Rightarrow$ sbagliatissimo: non ho semantic security. Se si ripete il plaintext, si ripete anche il plaintext.\\ CBC encryption (cipher block chaining): se m[0] = m[2] allora se li cifro senza fare nulla ho lo stesso cyper text. Perché non mischiare i blocchi con altri dati: \\ prendo PT$_{1}$ $\oplus$ IV$_{1}$, PT$_{2}$ $\oplus$ IV$_{2}$ etc...  e trasferisco i CT con i rispettivi IV. Genero un random IV ad ogni valore, ma il problema è che renderebbe ciphertext il doppio (IV deve avere la stessa taglia del plaintext). Cipher block chaining: prende IV (32 bit in SHA-256), faccio XOR con il messaggio m[0] ed ora faccio PRP con cui produco il ciphertext.\\ A questo punto uso c[0] come inizializzazione del prossimo blocco.
\subsubsection{CBC padding}
Voglio fare encryption usando block cipher: se i dati sono di 8 byte, il campo dati + MAC non è detto che sia di taglia giusta ( può non essere multiplo di 2). Quindi se uso CBC, TLS deve paddare i dati: aggiungo qualcosa per riempire il blocco, che poi dovrebbe essere rimosso. Idea di TLS: usa l'ultimo byte dice quanto è la taglia del pad, se è 0 vuol dire che non ce n'è pad. Inoltre, se uso ad esempio 1 byte: lunghezza sarà = 1 ed anche il byte in più sarà un 1. (se ne uso 10: metto 10 byte di pad + una lunghezza = 10 e tutti i byte avranno valore 10).\\ Anche se la lunghezza del blocco non ha bisogno di padding, ormai mi aspetto che l'ultimo byte sia la length, quindi devo aggiungere comunque i byte di paddin+len: PKCS specification per il padding, posso estendere il padding fino a 255 byte.\\ Quindi: ho dato è MAC, nell'ultimo byte metto la lunghezza del padding ed aggiungo il pad.\\ Proprietà: come fa un server a decriptare? Mando un dato + MAC + padding + length, encrypto e mando al server. Il server lo riceve e deve decryptare: se ad esempio usa CBC ed usa AES (16 byte) ed ho 48 byte (o comunque multipli okay)di lunghezza sono ok, altrimenti penso ci sia qualcosa di sbagliato, mando messaggio di errore fatal perché la decryption fallisce.\\ Se invece la lunghezza è ok decrypto e riottengo il pattern di partenza: ultimo byte mi dice quanti byte di padding ci sono. Se ricevo un pacchetto in cui i byte paddati non coincidono con il byte di length(es: length = 2 e bytes 7 ed 1), so che qualcosa è andata male, la decryption fallisce.Ora so che i byte rimanenti sono MAC è dati e so quanti byte compongono i singoli campi e posso checkare il pacchetto: se non è ok mando un messaggio che dice che il MAC check è fallito (bad\_record\_mac).\\ Perfetto se stessi considerando solo di networking: se c'è un errore devo farlo sapere all'altro end. In security questo non si fa: questo meccanismo è la base di un attacco, posso decryptare il pacchetto usando solo questi error codes $\Rightarrow$ padding oracle attack.
\subsubsection{Padding oracle attack}
Sfrutta le scelte sbagliate nel protocollo. Scoperto nel 2002, fa capire perché l'approccio del MAC then ECNRYPT è sbagliato. Abbiamo il padding, necessario quando si usa un block cipher, il blocco deve essere un multiplo del block size e spesso questo non avviene quindi paddo con dei bytes extra. Riservo sempre l'ultimo byte per la lunghezza del padding (quanti bytes extra ci sono).\\ In crypto quando qualcosa fallisce NON si spiega la ragione: l'attacker può usarlo per decriptare l'intero messaggio.\\ L'attacco: funziona se viene usato CBC di qualunque tipo (anche il migliore come AES-CBC che oggi è considerato sicuro).\\ Ricevo un messaggio in ciphertext: IV + c[0]....[n]. Il mio obiettivo è decriptare un singolo blocco, per esempio c[1]. Assumo che l'attacker può fare Chosen Cyphertext Attack: l'attacker può forgiare ogni ciphertext arbitrario e mandarlo al server e vedere la risposta. È un attacco molto realistico: se vedo una connessione e dei messaggi, dopo il msg n° 2 posso creare il msg n° 3 ed inviarlo. Quasi sempre il msg n° 3 che è inventato non sarà corretto, quindi la connessione si interromperà ma posso vedere la risposta.\\ Voglio decriptare c[1], mando un ciphertext arbitrario e mi aspetto decyption failed o Bad MAC. Ma cosa vuol dire bad MAC: attacker ha selezionato un cyphertext random, es 93142197 e questo è stato decriptato lato server e la decrypt è meaningless es 19123111 ma se ricevo bad\_record\_mac ho informazioni sul plaintext che ho inviato: la parte finale è 0, o 11 o 222 quindi so che gli ultimi bytes sono un padding corretto (se ricevessi un bad encryption non lo saprei).\\ Uso un oracolo, che mi dice se il padding è ok o sbagliato:
\begin{itemize}
\item Padding Ok è bad MAC
\item padding wrong è decryption failed
\end{itemize}
In CBC prendo il plaintext, lo choppo in blocchi, uso IV col primo blocco m[0] e lo metto in XOR. Poi lo passo al PRP ed uso il risultato per m[1].\\ In decryption faccio l'opposto: ricevo IV, prendo c[0] applico PRP$^{-1}$ e faccio XOR con IV ed ho m[0]. Poi prendo c[1], faccio decrypt e metto in XOR con c[0] ed ottengo m[1] e così via.\\ I bit di c[1] sono mischiati, quindi se cambio dei bit ho risultati impredicibili. Ma il fatto che uso c[0] nello XOR con c[1] senza passare per non linear transformation è la chiave per l'attacco.\\ Voglio scoprire un messaggio in plaintext: vedo un messaggio legittimo IV c[0] c[1]....\\ Levo la parte del messaggio che non mi interessa, faccio si che c[1] contenga l'ultimo byte del pacchetto e lo mando al server. Posso indovinare se l'ultima lettera è A?\\ Prendo c[0] $\oplus$ A $\oplus$ 0.Il messaggio va al server e farà i check:
\begin{itemize}
\item È di taglia giusta, perché ho tolto blocchi interi
\item Posso decryptare? Sì
\item Suppongo che il contenuto del messaggio fosse: Ciao I am Flavia, sto modificando l'ultimo byte facendo XOR con A e 0. Quindi l'ultimo byte del plaintext è 0 (quello del blocco affianco).
\item La  del server è bad record MAC: il messaggio è valido perché termina con 0, quindi lo rimuove e checka se la parte del messaggio è hashing corretto. Non lo è, però ho un'informazione e riguardo l'ultima lettera.
\end{itemize}
Per due lettere, il meccanismo è analogo:\\
\includegraphics[scale=0.5]{immagini/CNS2910_1.png}\\
Prendo messaggio originale, lo spezzo così che l'ultimo blocco è quello che attacco, faccio XOR del blocco prima tra l'ultimo byte, il byte che volgio testare ed il byte 0x00..0 così che se il guess è corretto il byte diventa 0 e ricevo bad MAC, altrimenti ricevo decrypt failed. È un test su una singola lettera nel plaintext.\\ Se lo ripeto più volte: ho indovinato l'ultima lettera, posso testare la penultima: guesso se termina con FA. Prendo c[1] $\oplus$ FA $\oplus$ 01 01.\\ Sta volta uso 01 01 quindi di nuovo avrò che il padding è ok se il guessing è ok.\\ Quanto è lungo l'attacco ad un intero blocco: meglio di un brute force attack, in brute force avrei 256$^8$ se avessi 8 byte di blocco, qui passo a 256x8 = 2$^{11}$.\\ Ma dopo il primo tentativo, la connessione si interrompe.\\ esercizio: ho un cipher f1 aa 11 04 || 34 35 f1 20 || 11 01 9c 01 || ac c3 83 02 || 65 61 fb 08 || 91 11 5f 10. Voglio scoprire il 16° byte  con padding oracle attack è 0x0f = 00001111.\\Per prima cosa devo rimuovere la parte che non mi interessa: il byte che attaco deve essere l'ultimo. Per modificare il byte: ho 11 01 9c 00000001 || ac c3 83 02 e gli ultimi sono cambiati da PRP$^{-1}$, quindi devo cambiare l'ultimo byte del blocco precedente così da avere effetto sul byte del blocco affianco.\\ Ora mi verrà che l'ultimo byte è 00000..0 e quindi potrò vedere il mio guess.\\ Quindi: se voglio scoprire un byte:
\begin{itemize}
\item Choppo il messaggio lasciando il blocco di cui voglio indovinare il byte alla fine
\item Faccio XOR del ultimo byte del blocco precedente con la lettera che voglio indovinare e con 0x00.
\item Quando verrà decriptato: se il byte è corretto, avrò che l'ultimo byte che sto cerando di indovinare è un 0 e quindi verrà visto come un padding. Verrà scartato ed il server cercherà di checkare l'integrità. Questo fallirà ed avrò bad MAC e quindi saprò ch ci ho preso. Se invece ricevo dab decryption, ho sbagliato il guess.
\end{itemize}
Se voglio indovinare più byte devo andare linearmente (0, 11, 222, 3333 etc...).\ Come faccio a prevenire l'attacco: ho standardizzato due messaggi per due situazioni diverse. Correzione: restituisco una sola risposta, triviale (la maggior parte delle implementazioni corresse mandando sempre bad mac).\\ Se la decriptionva a buon fine, se check fallisce restituisce bad MAC che però è il vero bad MAC. Sono sicuro che attacker non può capirlo? Uso il tempo come indicazione per capire se il msg di errore è avvenuto a livello di encryption o di vero bad MAC. Side channels: problemi, come in questo caso per colpa del tempo.\\ LAN, quindi rete eth: la risposta può essere diversa per veri bad MAC o decryption failed. Mandando pochi messaggi è possibile discriminare quali dei due errori ho avuto.\\ Il programma sarebbe sequenziale, mentre un crypto programmer cercherebbe di fare entrambi i check in parallelo. (se faccio dei check e setto un flag di conseguenza e poi passo ad un compiler con flag -03 che però non capisce che che sto facendo programma per sicurezza).\\ Tutte le implementazioni corrette:
\begin{itemize}
\item TLS 1.2: valida il MAC anche in caso il padding fallisca
\item Ma in che caso i dati vengono validati?Se messaggio è formattato bene, so quanto devo paddare, ma dopo decryption non so quanti dati devono essere validati. Se padding è di 2 bytes il msg finisce ad un certo punto, ma al variare del padding ho taglie di messaggi diverse. Decisioni complesse, quindi quanti dati valido? TLS 1.2 valida tutto, considera come max size msg, ovvero anche se padding fallisce considera msg | MAC e ultimo byte
\item Kenny Patterson: HMAC, il tempo di computazione non è lineare, ma a scalini. Quando computo HMAC ogni blocco è computato con una f. di compressione. Posso misurare $\Delta$t tra i le varie lunghezze di blocchi? Attacco lucky thirteen: c'era ancora time channel
\item POODLE: altro attacco, 2014
\item 2015: altro subtle time channel
\item 2016 attacco alla patch per fixare l'attacco lucky 13.
\end{itemize}
\subsubsection{Lezioni}
Problema è sempre quello: prima decryption e poi computazione MAC. Prima MAC e poi ENCRYPT protegge da questi attacchi: non posso più fare CCA, prima checko l'integrità e solo poi vado avanti. Ma se l'integrità non è verificata non ho modo di fare CCA; l'ordine delle operazioni conta.\\ Quando si ha a che fare con la crittografia bisogna stare attenti: non va in conflitto con security by obscurity, quello dice di usare open algorithm. Ma quando sono a runtime devo stare attento se quando differenzio una risposta, può esserci un oracolo che le differenzia e da delle informazioni all'attacker.\\ Non implementare crypto da se, attenzione ai side channel attacks.\\ Usa librerie cryto fatte da gente esperta: openSSL, LIBSODIUM, etc... (stesso vale per l'hardware).\\ L'attacco è pratico? Posso forgiare un cipher al server e dopo la riposta ho un errore fatale, che fa si che la connessione si rompa. Devo mandare 2048 messaggi, la prossima connessione partirà con una chiave nuova. L'attacco sembra non pratico, ma in qualche caso specifico può diventarlo. Telefono checka le e-mail, probabilmente ogni 5 min (perché si usa IMAP): setta connessione TLS, ed ogni volta la passoword e l'id sono nella stessa posizione. Sono attacker, conosco il formato di IMAP e voglio conoscere la password: so che sta nel blocco 3. Vedo il primo messaggio, provo con il primo messaggio, anche se fallisco dopo 5 min posso riprovare. L'informazione è strutturata sempre nella stessa maniera: posso fare più trial.\\ Dimostrato nel 2003 che era possibile intercettare e raggruppare le psw degli utenti tramite IMAP (usarono euristiche come dictionary atk etc..)\\ Quindi la vulnerabilità è estremamente pratica.\\ Cryptography Doom Principle (Moxie Marklinspike, crittographer che ha standardizzato la sicurezza di WA): se devi fare ogni operazione crittografica prima di verificare il MAC sul messaggio ricevuto, in qualche modo sei condannato.\\ TLS 1.3 standardizza AEAD, visto che non era possibile cambiare l'ordine delle operazioni (modifiche sw troppo importanti), non più MAC then ENCRYPT.
\subsection{Block ciphers}
L'obbiettivo è quello di generalizzare un substitution cipher: esempio sostituisco una lettera con un altra (vedi esempio de Giulio Cesare). Definisco un blocco: input è plaintext di taglia nota (in AES per esempio è 128 bit) e l'ouput è una stringa di altri 128 bit, diversa. Rimpiazzo dei 128 bit con altri 128 bit: ho $2^{128}$ possibile input, il rimpiazzo è guidato da un chiave segreta che è il tipo di sostituzione che vado a fare: k seleziona una delle possibile permutazioni.L'algoritmo di blocco (difficile da disegnare) dovrebbe implementare una pseudo-random permutation o PRP, chiave dovrebbe selezionare una tra tutte le possibili permutazioni (in pratica, sceglie tra un insieme di queste)
\subsubsection{PRP}
S è l'insieme di tutti i possibili plaintext, se n = 3, la cardinalità di S è 8. Una permutazione è una biezione $\Pi$ che associa ad ogni elemento di S un altro elemento di S. Voglio che la trasformazione sia reversibile.\\ Pseudo-random: il block cipher dovrebbe selezionare uniformemente una delle possibili permutazioni.Se ho 8 simboli, ho un totale di permutazioni=$2^{n}!$. A seconda del valore della chiave avrò una certa permutazione associata. La permutazione $\Pi_k$ è associata alla chiave k.Quante permutazioni associate a quante chiavi: se ho 3 bit, $2^{3}!$ = 40320 (posso avere anche la permutazione identica, ma nella pratica non si usa). Con 8 bit, ho 256 simboli differenti che si permutano in $256!$: 8.58x$10^506$. In AES n = 128 bit, numero di simboli è $2^{128}$, n° permutazioni è $2^{128}!$ che è una roba non pensabile.Dovrei avere una chiave di $10^{40}$ bit mentre in AES sono di 128, 192 o 256. Se considero chiave di 256 chiave ho un numero totale di chiavi di $2^{256}$ che è il max numero di permutazioni che è molto molto minore del totale delle permutazioni (ricorda che ad ogni chiave è associata una specifica permutazione); probabilità di selezionare la permutazione random è $\epsilon$.\\ Considero quindi un subset dell'insieme delle PRP, ma siamo comunque ok (vedi ad esempio AES).
\subsubsection{Problema 1-Plaintext più lungo della taglia del blocco}
Ho capito che cos'è il PRP, mi fido che è ben fatto da crypto guys (se vedo alcuni mapping non posso sapere a cosa mappa un certo valore basandomi sugli altri, questo come nozione di sicurezza del PRP).\\ Plaintext più lungo della taglia del blocco, devo fare qualcosa: posso pensare di dividere il messaggio in chunks di taglia uguale alla taglia del PRP block (128 bit in AES ad esempio) e passare ogni blocco al PRP. Sbagliato: se m[1] ed m[3] sono lo stesso messaggio, il PRP fa si che il ciphertext si ripete, perché la chiave è la stessa\\
\includegraphics[scale=0.3]{immagini/CNS211_1.png}\\ Perdo la semantic security, non IND-CPA secure.L'approccio è chiamato Electronic Code Book (ECB), e non va bene (big red flag).
\subsubsection{Problema 2-Stesso plaintext}
Ho visto il problema di ECB, ma se il messaggio è più corto? Ho ad esempio 8 lettere di messaggio ed il PRP usa blocchi di 8 byte, quindi uso un singolo blocco e qui ECB potrebbe funzionare. Ma se lo encrypto di nuovo, riottengo lo stesso ciphertext.\\ Riuso l'idea dell'inizialitation vector: lo uso per ogni nuova encryption: prendo plaintext lungo quanto un singolo blocco, genero l'IV che deve essere di taglia uguale al plaintext. Li combino con XOR, ottengo sempre 8 byte di risaluto ed ora uso PRP indicizzata dalla chiave k ed ho il mio ciphertext.Posso trasmettere l'IV in chiaro con il plaintext, quindi c'è dell'overhead. A receive side: prendo cipher, faccio $PRP^{-1}$, faccio XOR con l'IV ed ottengo il messaggio; va tutto bene se la taglia del messaggio è $\leq$ della taglia del blocco.\\ L'IV non si ripete mai? Ma non basta, deve anche essere impredicibile (alla base del BEAST attack). Ovviamente, se riuso lo stesso IV ed il plaintext è lo stesso l'encryption è lo stesso, inoltre non deve essere predicibile: una nonce non deve essere impredicibile, basta che sia fresca. In questa specifica applicazione non è così.
\subsubsection{Modes of operation}
Non usare mai ECB, usabile solo se :
\begin{itemize}
\item Messaggio è minore o uguale ad un blocco
\item Messaggio non si ripeterà
\end{itemize}
Solo in queste condizioni strette si potrebbe usare (ma meglio non usarlo)\\ Per messaggi ripetuti: si genera un IV, random e della stessa taglia del blocco.\\ Per messaggi più lunghi: modes of operation: ho due ingredienti:
\begin{itemize}
\item Il blocco stesso, cambia il modo di implementazione a seconda dell'algoritmo
\item Modo di operare, ho visto solo CBC ma ce ne sono altre. Questo permette di criptare testo di lunghezza arbitraria e mi permette di criptare senza ripetizioni del cipher ne momento in cui ho ripetizioni nel plaintext.
\end{itemize}
\includegraphics[scale=0.3]{immagini/CNS211_2.png}\\
I più usati in pratica: CBC o CTR(Counter Mode, preferita dal prof a CBC)\\ NIST raccomanda, oltre queste 2, CFB e OFB.\\ Alcune più sofisticate, che combinano authentication ed encryption\\ Semantic security: prendo il plaintext, lo taglio in blocchi e genero un IV indipendente e truly random che associo ad ogni blocco, a questo punto combino in XOR con questi IV e poi produco con il PRP il mio ciphertext. Siccome IV è random, non ho ripetizioni, ma ogni volta che mando il blocco devo spedire pure l'IV, quindi non è soluzione pratica; è il meglio che posso fare.
\subsubsection{CBC}
Modo più comune è usare CBC:
\begin{itemize}
\item Prende blocco di messaggio, aggiungo IV al primo messaggio e faccio XOR con m[0]
\item Trusto che PRP è pseduorandom permutation e quindi che c[0] sia come un "random" IV per il blocco 2
\item Prendo c[0] come IV per il blocco m[1], quindi lo uso per fare lo XOR e poi passo a PRP.
\end{itemize}
c[i] = ENC=PRP(K, c[i-1] $\oplus$ m[i]).Overhead è solo un blocco extra, se messaggio ha taglia di m blocchi, il cipher è di m+1 blocchi.\\ La fase di decryption procede al contrario: prendo IV, metto in XOR con $PRP^{-1}$ di c[0] ed ottengo m[0] che userò in XOR col risultato del $PRP^{-1}$ del c[i] successivo. Encryption consuma tempo perché non è parallelizzabile, mentre la decryption lo è: se voglio decryptare solo un blocco faccio accesso alla RAM e decrypto solo quello (se ho salvato il ciphertext lì).\\ È il modo più comune di fare block cipher, è sicuro se IV non è predicibile e non si ripete.\\ È lento per l'encryption ma almneo è veloce in deryption (appropiato per accesso ad encrypted file system o DB). Servono 2 circuiti diversi per encryption/decryption: le due direzioni sono diverse\\ Inoltre, il plaintext deve essere multiplo del block size: se non lo è, paddo (esiste standard per questo), è necessario anche in altri casi.
\subsubsection{Altri modi: CFB e OFB}
CFB: prendo un blocco, prendo IV e ne faccio XOR. Applico PRP ed ottengo il ciphertext. Sto prendendo plaintext e applico due trasformazioni che sembrano encryption scheme: nella parte dell'XOR sembra uno stream cipher con keystream noto. Quello che CFB fa è: criptare l'IV, fare XOR col blocco di plaintext ed ottenere il ciphertext. Trasformo il block cipher in qualcosa tipo uno stream cipher. Ho un pad keystream ed ho l'XOR classico dello strema cipher. Ora faccio CBC chaining, uso blocco c[0] a cui applico PRP e faccio XOR col blocco plaintext successivo.\\ OFB: principio è lo stesso, parto dall'IVe ci applico il PRP ma ora uso direttamente questo risultato. Lo prendo, faccio PRP e lo metto in XOR col plaintext, per il blocco 1 faccio come in CFB. Posso partire da IV ed applicare i vari PRP, sembra molto di più uno stream cipher: parto da un seme e genero i vari keystream per criptare i plaintexts. CBC/CFB non sono parallelizzabili, mentre OFB sì: posso precomputare tutti i keystream.\\ Decryption in CFB: non devo più invertire il PRP, prendo IV, lo encrypto con PRP e faccio XOR col blocco in ciphertext per ottenere il plaintext.\\ Lo stesso vale per OFB; in CFB posso fare decryption in parallelo. Perché non usarli: problema dello short cycle:\\ OFB: IV $\rightarrow$ PRP $\rightarrow$ PRP $\rightarrow$ PRP... esempio: blocco di 3 bit, permutazione è selezionata random, parto dal primo e comincio ad iterare: mi fermo prima di 8, periodicità di 5.\\ Non tutte le permutazioni sono le stesse: alcune chiavi k potrebbe portare a permutazioni che risulta in short cycle. Se ho un IV sfortunato la periodicità cala. Posso avere anche cicli più piccoli, come 3.\\ Per usare questi sistemi serve anche conoscere le reali proprietà delle permutazioni, devo progettare AES per non avere short cycle, entrare nei dettagli etc... quindi piango.\\ Problema di CBC, CFB ed OFB, che in un modo o nell'altro fanno chaining dei blocchi: non ho garanzie che non finirò in uno short cycle a meno di aprire la scatola nera dell'algoritmo di PRP.\\ Come caso degenere, esempio: se ho 3 blocchi 011 011 011 ed uso CBC con IV 010 ottengo 3 ciphertext identici, quindi male.\\ Anche nel caso dell'hash chian(OTP) ho questo problema, per questo motivo in crypto si odia il chaining.
\subsubsection{CTR}
Counter mode encryption, molto semplice:
\begin{itemize}
\item Prendo un contatore, nonce completamente predicibile. Lo incremento per ogni nuovo blocco
\item Faccio il PRP del counter (il counter deve avere la stessa taglia del blocco usato nel PRP). Ottengo i keystream differenti per costruzione del PRP, che è una biezione (non è più un hash) e se è fatto bene il keystream è scelto random fra le $2^{2^{135}}$.
\item Periodicità: periodicità del counter, se vai da 0 a $2^{128}$ ho un PRNG perfetto che non si ripeterà, quindi non ho short cycles
\item Faccio XOR del blocco m[0] con counter ctr ed ottengo c[0], stessa cosa per m[1] $\oplus$ ctr+1 = c[1]etc... Non c'è chaining, ogni blocco è indipendente ed ho chiavi indipendenti, che sono generate a partire da un contatore.
\item Se conosco il counter e l'algoritmo, posso parallelizzare: se voglio blocco 200, prendo ctr = 200, uso PRP e computo c[200]. Stesso vale per decryption: voglio m[200], faccio PRP[200] $\oplus$ c[200] = m[200]
\end{itemize}
Non ho short cycles, incredibilmente efficiente in hardware perché parallelizzabile.\\ Iniziamento non molto famoso, ovviamente se counter si ripete $\rightarrow$ gameover, anche se lo suo per lo stesso messaggio: deve essere simile ad un IV.\\ AES-Crt: blocco è di 128 bit (AES), è stato standardizzato che se accetto che la taglia massima di encryption è $2^32$ blocchi, ovvero $2^32$x16 bytes = 500 GB, prendo gli ultimi 32 bit e li uso come counter ed uso gli altri 96 bit come IV: parto da 0 come crt ma l'IV sarà truly random.\ Vantaggi:
\begin{itemize}
\item Rende block cipher stream
\item Combina vantaggi di CFB e OFB (molto efficiente per encryption di file system)
\item implementazione efficiente hw e sw
\item richiede implementazione solo dell'encryption block (PRP)
\item Posso fare random access: se indicizzo blocco col contatore, posso convenientemente accedere (per esempio memoria)
\item Se ben usato (IV truly random e non si ripete, usato per al più $2^{32}$ blocchi) è l'approccio più sicuro: è garantito che per ogni permutazione di AES-CTR non ho problemi di weak permutation, per costruzione non posso avere short cycles.
\end{itemize}
\subsection{Vulnerabilità di IV predicibili}
Per criptare due messaggi: i due IV devono essere diversi, ma questo non basta: l'IV deve anche essere non predicibile e questo sembra contro-intuitivo.\\ Suppongo di voler criptare m1$|$m2: prendo IV, choppo m1 ed m2 in blocchi, uso cipher block chaining così che l'ultimo blocco c di m1 sia usato come IV del primo blocco di m2. Uso IV per criptare m1, assumo che m2 usi l'ultimo ciphertext del messaggio m1. Sembra avere senso: perché la costruzione sopra in cui ho i due messaggi vicini è diversa da questa in cui il messaggio m2 è su una nuova riga? L'intuizione mi dice che se la prima costruzione è sicura (e lo è), allora anche la seconda lo è. Ma io ho scoperto che l'IV deve essere impredicibile: se uso la parte finale c[n] del messaggio 3, posso predire il prossimo IV: nella seconda costruzione sto violando la proprietà 2 di impredicibilità, lo vedo come ultimo ciphertext del messaggio trasmesso precedentemente.\\
Ho un messaggio legittimo, trasmesso da qualcuno. Posso solo vedere il ciphertext, ma ricevo un vantaggio: nel blocco m[i] del messaggio (assumo blocchi di 8 lettere) ho la stringa pass = XXX (non so cosa ci sia in XXX). Vantaggio 2: ??? non è sequenza arbitraria ma ha solo due possibilità: JOE o UGO. Basta per rompere il sistema? Lo sarebbe se fosse possibile fare dictonary-like attack: posso fare Chosen Plaintext Attack, 3° vantaggio è quindi di criptare un messaggio a scelta. Posso criptare PASS=JOE, guardo il ciphertext e vedere se combacia. Ma l'attacco non è possibile: se encryption scheme è semantically secure, ho un ciphertext diverso per lo stesso plaintext.\\ Se l'IV può essere predetto, posso rompere questa proprietà. Posso imparare, sapendo che l'IV è predicibile, selezionando un testo di mia scelta così da rivelare quale password è stata trasmessa nel messaggio precedente? Entro nel browser e faccio criptare a TLS un plaintext di mia scelta: IV è predicibile, posso vedere ct[last] e so che viene usato come IV del prossimo messaggio, che è quello che l'attacker fa criptare: ho X predicibile, che è l'IV. Scelgo il messaggio: XOR è prima del PRP, se ho X $\oplus$ qualcosa come messaggio, prima del PRP posso vedere il qualcosa. Qualcosa = JOE$\oplus$UGO, quindi riapplicando l'XOR ho solo il qualcosa. Nel messaggio prima avevo PASS=XXX $\oplus$il cipher c[i-1]. Se metto nel testo PASS=JOE$\oplus$[c-1], mi rimane solamente il PASS=JOE se il guess era corretto.Quindi:
\begin{itemize}
\item Predico X
\item Scrivo un messaggio che contiene X $\oplus$ c[i-1] del messaggio precedente $\oplus$ PSWGUESS
\item Convinco l'implementazione a criptare il mio messaggio (verrà fatto X $\oplus$ X $\oplus$ c[i-1] $\oplus$ PSWGUESS = c[i-1] $\oplus$ PSWGUESS), se il c[0] risultante è uguale a c[i] (ovvero X), il guess è corretto. Altrimenti, la password è l'altra.
\end{itemize}
Effetto di overall è che se IV è predicibile posso fare trial and error attack.\\ Da un punto di vista del sistema: ho il browser web acceso e questo può tx un messaggio con TLS. Il messaggio ha la psw, per attaccare usando il browser uso software di attacco per far criptare un nuovo messaggio scelto da me.\\
\includegraphics[scale=0.3]{immagini/CNS411_1.png}
\subsubsection{Exploit in TLS-BEAST attack}
Sembra un attacco teorico: se ad esempio la psw è 8 byte: $2^{64}$ try, quindi il CPA si può fare ma è idealizzato. Inoltre devo fare il CPA.\\ Il fatto che l'IV non dovesse essere predicibile era ben noto dal 1999 (Rogaway, IPsec), inoltre problema di standardizzazione in TLS 1.0: TCP trasmette in ordine, per evitare un IV fresco per ogni messaggio, l'idea è di usare il ciphertext dell'ultimo messaggio come IV del successivo, basandosi anche sul fatto che TCP lo permetteva.\\ In TLS 1.1+ fu corretto, aggiungendo un IV esplicito per ogni messaggio (mandatory in DTLS). Nel 2011 pochissimi usavano TLS 1.1\\
Beast attack: si credeva impredicibile, software issue: devo installare Trojan nel PC della vittima e fare injection di messaggi nel browser.C'erano nuove tecnologie come Websockets, HTML5 ed era possibile avere connessione aperta e catturare extra sources per fare injection. Altro motivo per l'impraticabilità era la complessità: per decriptare un blocco di AES, quindi 128 bit di blocco, devo fare brute force di tutti i $2^{128}$ possibilità: non devo basarmi solo su psw, perché si usano spesso cookies ($>$ 64 byte e molto entropici).2011: Attacco lineare nella dimensione del cookie (demo su youtube)\\ Come trasformare brute force attack in tempo lineare: per crackare un blocco dovrei enumerare tutti i possibili valori dei blocchi che sono $2{128}$ tentativi. Ma è possibile fare injection di testo: quando mi connetto ad un server, mando un messaggio. Non conosco l'auth cookie, perché è nello storage sicuro del browser. Ma quello che posso fare è non solo fare injection di un nuovo messaggio, ma anche iniettare del testo di preambolo prima della connessione al server: creo un commento che aggiungo al messaggio (così che il server non lo parsi) e posso misurare la taglia del commento, posso muovere il limite da cui partire: se posso aggiungere testo per cui un carattere noto cada alla fine del blocco, devo fare 256 tentativi per crackarlo. Chosen Boundary attack: selezioni e cambi i limiti del blocco criptato.Ora è possibile attaccare: attacco la prima lettera, poi la seconda (lo forzi o aspetti il prossimo accesso). Complessità dell'attacco, se cookie è di N byte l'attacco è di N$\cdot$256 tentativi, contro il $256^N$.
\\ Quindi, l'IV predicibile in CBC è exploitabile
\subsection{CRIME attack}
Compression leaks. Stessi autori del BEAST (orami avevano i trojan per muovere i limiti del blocco). TLS usa compression e poi encryption, anche se non posso decrpitare, la taglia della compressione rivela delle informazioni ma in teoria non sono usabili.\\ Compressione disabilitata dopo questo attacco.Problema non solo di TLS, ogni volta che c'è compressione e poi encryption può esserci un problema. C'era un paper del 2002 che spiegava come la compressione potesse rivelare informazioni sul plaintext.\\ Idea: posso comprimere e cripater due stringhe: AAAABC $\rightarrow$ 4ABC (se non trasmetto numeri), se non ci sono ripetizioni non posso fare nulla (es ABCDEF). Ora faccio encrypto: ad esempio con stream cipher vedo una stringa di 4 caratteri che non ha leak. Idea: plaintext injection nel BEAST attack, aggiungendo un preambolo per shiftare preambolo. C'è una password in plaintext, non la conosco. So che sarà compressa + criptata ad una taglia es di 6 byte. Ricordo che potevo aggiungere un preambolo: posso ad esempio mettere AAA o BBB o SSS come preambolo, ovvero una sequenza compressa. Guardo al risultato: se comprimo AAASHARON il risultato è 3ASHARON, BBBSHARON è 3BSHARON, ma SSSSHARON sarà 4SHARON. Primi due casi ho 8 byte, ma nel caso 3 ho 7 bytes $\rightarrow$ capisco la prima lettera. Se posso aggiungere un preambolo e forzare l'implementazione ad aggiungere il preambolo, allora posso rivelare una lettere e decriptarla.\\ Lo stesso problema può accadere se ho un DB, all'interno del cui ho dei dati privati, e che viene poi compresso e criptato. Se posso fare injection di dati nel DB, ho la stessa vulnerabilità; quindi non è solo un problema di TLS. Attacco CRIME funziona anche per block ciphers, cambiano i dettagli a seconda dell'algoritmo di compressione. Possibilità di fare injection di testo, prima dei dati utenti: in BEAST era padding con commenti inutili, mentre ora è injection di testo scelto. Dettagli di uno specifico meccanismo, in questo caso DEFLATE, ma può essere applicato ad altri compression schemes.\\DEFLATE: due tecniche, una bit oriented, un'altra è il Lempel Ziv algorithm: lavora a livello di byte, prende 3+ caratteri, fa un replacement: giuseppe bianchi and marco bianchini, primo passo del parser vede che "bianchi" si ripete (anche spazi si ripetono, quindi vanno compressi).Lascia la prima stringa inalterata, per l'altra aggiunge una coppia(-18,7): -17 dice di andare indietro di 18 caratteri (pointer) e il secondo numero dice quanti byte prendere (in realtà è (-17,8) contando gli spazi). Risultato è: giuseppe bianchi and marco (-18,7)ni.\\ Ho un testo utente legittimo, voglio indovinare la password:\\ GET /comment:twid=a HTTP1.1 Cookie: twid=flavia... Aggiungo un preambolo alla richiesta, che sia una parte di commento, in questo caso /comment:twid=a. Conosco format della richiesta di Twitter, quindi so che parte con twid= e cerco di indovinare la 6° lettera, ad esempio la a. Tutti  i compression scheme usano una finestra per spostarsi, in quanto diventa difficile andare indietro, ad esempio di 2GB.\\ Una volta compresso: GT /commenti:twid=a HTTP/1.1 Cookie: (-24,5)flavia. Prende solo il preambolo twid=, ma posso ripetere finché non becco la lettera f, perché mi rendo che la taglia di byte da leggere diventa 6 invece di 5. Ora provo la lettera 2 e così via...\\ Attacco lineare all taglia del segreto, quindi molto pratico da effettuare. Anche se auth cookie è di 64B ho un O(64x256) nel caso peggiore, quindi attacco molto pericoloso.
\subsection{TLS Handshake Protocol}
A partire da TLS v1.2, in TLS v1.3 differisce.
Quando si usa l'handshake:\\
connessione ad un server. https://www.fineco.it, ogni votla che mi connetto parte una handshake iniziale che ha come obiettivi:
\begin{itemize}
\item Mutually authenticated, ma di solito è unilaterale: quando apro TLS connection la banca mi prova la usa autenticità e una volta che il tunnel TSL viene stabilito (encryption ed integrity) mando username+password. La banca non sa se sono autentico a meno, uso PAP all'interno di TLS. Se posso assegnare all'utente un certificato di sicurezza, posso usare la mutual authentication.
\item Non c'è algoritmo di encryption specifico: protocollo è disaccoppiato dal algoritmo crittografico di sicurezza, quindi si negozia l'algoritmo che verrà usato. Ma può essere attaccata la negoziazione: se convinco client e server che si può usare solo RC4? Bisogna proteggere questa fase.
\item Servono nonces, scambio delle random quantity
\item Serve scambio dei segreti per computare i segreti. Serve quindi asymetric cryptography,  non posso mandare in plaintext
\end{itemize}
Se mi collego, e poi mi ricollego di nuovo il processo ristarta. Ma nella connessione scambio molti messaggi col server, idea di TLS è che ogni connessione deve usare una chiave di encryption e di HMAC diverse anche se le connessioni sono su due server diversi. Questo permette di ridurre il riuso di chiave ed evitare crypto analisys. Ma per usare nuova chiave, l'asymmetric crypto e computazionalmente costosa: symmetric vs asymmetric è $10^4$ più veloce. L'idea è che posso riusare del lavoro fatto nel primo handshake per farne uno abbreviato, quello che si fa nel mondo reale. Su altre connessioni TCP uso handshake più leggere, abbreviate: sesison resumption, si riusa parte del lavoro della prima connessione per computare chiavi differenti per la connessione.\\Obiettivi dell'handshake:
\begin{itemize}
\item Negoziazione sicura dello shared secret, con asymmetric crypto
\item Autenticazione opzionale, facendo autenticare sia client che server, in modo da essere robusti a MITM. Non mi proteggo da ARP poisoning, DNS spoofing etc..., ma TLS garantisce che l'atk non può vedere o modificare il contenuto dei messaggi: ho integrity protection e encryption.\\ Ma se la banca non è autenticata, atk si finge la banca, prende il traffico lo modifica e lo manda a Fineco ma questo è protetto dall'autenticazione.\\ Tecniche robuste contro attacchi classici e strong, ma se attacco è del governo manco per nulla.
\item Negoziazione affidabile: attacker non deve poter fare danni nella fase di negoziazione. Se mi scambio gli algoritmi di encryption ed un atatcker convince entrambi gli host che si può usare un protocollo weak è finita. Va protetto
\end{itemize}
\includegraphics[scale=0.3]{immagini/CNS511_1.png}\\
Tutti i messaggi precedenti alle frecce grige sono in chiaro, aggiungo header di 5 byte ai singoli messaggi e li passo a livello 4 a TCP.\\ Header TLS:
\begin{itemize}
\item Content type
\item Major version
\item Minor version
\item Length
\end{itemize}
Poi c'è payload del messaggio.\\ esempio: mi collego ad una banca, prima di inserire in miei dati solo la banca si autentica a me, io mi autenticherò passando username e password. Scambio TLS v1:
\begin{itemize}
\item Primo messaggio è mandato dall'utente al sito della banca, che è il Client Hello. Incapsulato su TCP/IP, la parte di TLS: 
\begin{itemize}
\item v1.0 (SLL 0301)
\item length
\item handshake type
\item 32 byte di nonce, contiene sia timestamp che random bytes. Ora deprecato il timestamp, non è autenticato quindi non è detto che sia un timestamp vero, quindi ora sono 32 byte du random e non più 28+4.
\item (Session id lenght, session id): può servire se voglio riusare una sessione precedente.
\item Parte della negoziazione: cipher suites e compression ): in TLS, la prima cosa da fare è condividere il segreto, si fa con asymmetric crypto, anche detta public key cryptography. Devo scegliere l'algoritmo da usare: in TLS tutto va negoziato (no hardcode nel protocollo), possibilità di negoziare il public key algorithm: esempio TLS\_RSA: uso di RSA come algoritmo asimmetrico. C'è anche una seconda parte: TLS\_RSA\_WITH\_ algoritmo da usare per symmetric encryption ed integrity, quindi cosa userò quando comincia l'encryption dei dati. esempio: TLS\_RSA\_WITH\_RC4\_128\_MD5.\\ Oggi la lista di ciphers è estesa (è possibile specificare NULL per la parte dell'algoritmo di encryption e richiedere solo integrity, conferma del fatto che i due servizi sono diversi).\\ La parte importante è che il protocollo non sia legato intrinsecamente al crypto algorithm.
\item Versione di Aprile 2012: negoziazione anche del compression algortihm (prima di CRIME), ora hardcoded NULL.
\end{itemize}
Nel Client Hello do una lista di possibili algoritmi i supportati), ed il server ne sceglie una. Se il server è vecchio, può scegliere l'algoritmo peggiore (le opzioni sono messe in ordine crescente per livello di sicurezza). Negoziazione: client offre, server sceglie.
\item Server Hello: 
\begin{itemize}
\item Handshake type
\item versione del protocollo: anche la versione di TLS viene negoziata, anche questo è fondamentale. Per questo è presente la versione nel client: appariva in due parti, nel record protocol e nell'handshake. Nell'handshake c'è la proposta di quale versione usare, quindi il server può anche sceglierne una diversa se non supportata.
\item Nonce del server, 32 byte(4 TS + 28 random).Server risponde 3 ore dopo? No, il timestamp probabilmente è considerato diversamente sui due sistemi, quindi 4 byte di TS erano uno spreco perché non c'era una time reference precisa.
\item Ho un session ID ora
\item Cipher suite: server risponde con un protocollo diverso da quelli che aveva il client come priorità
\end{itemize}
\end{itemize}
Ho appena visto la negoziazione della versione, downgrade attack (arma grande, difficile da fixare): client usa TLS v1.0 (SSL v3.1), server anche supporta TLS v1.0 ed assumo che la versione sia sicura.\\ Assumo che l'attacker non possa rompere TSL v1.0 ma possa rompere SSL v2.0\\ Attacker prende Client Hello, è in plaintext e senza integrity perché non ho un segreto shared, quindi non posso criptarlo, messaggio molto vulnerabile.\\ Vado nella parte del messaggio dove c'è a versione e cambio: cambio due byte, da 0301, metto 0200.\\ Server vede che versione è più bassa, ed accetta la connessione (oggi non è possibile scendere sotto TLS v1.0, poodle attack: downgrade di TLS + padding oracle). Ora attacker rigira la risposta al client, così che questo creda che il server abbia supporto alla versione di TLS/SSL più vecchia.\\ Agreement sulla versione di SSL v2.0, convinti che questa è l'unica versione usabile; è un problema fondamentale di tutte le negoziazioni.
\subsubsection{Public key cryptography}
Esempio di problemi risolti da asymmetric encryption: comunicazione tra l'app ed il server è essere criptata, ma il problema è dov'è la chiave. Se ho una SIM card, ho lì l'encryption key. Ma come risolvo problema del downaload di un applicazione: scarico e mando encrypted traffic, ma dove metto la chiave? Se metto nel codice, male: hacker wannabe 101 cerca pattern nel codice, faccio regex (della chiave) e guardo nel binary code dove il charset è ristretto a base64, quindi ho trovato la chiave. O predo la chiave da una comunicazione separata, ma come faccio: se devo chiederlo al server devo poterla trasferire.\\ Asymmetric cryptography: differenza con le simmetric key è che la chiave usata per criptare è diversa da quella usata per decriptare. Solitamente: prendo plaintext, applico una trasformazione che deve essere reversibile, ed ho il mio ciphertext. Ho visto stream e block cipher, ma si basano entrambi su una chiave preshared.\\ 1970: schema dove
\begin{itemize}
\item La chave usata per encryption è diversa da quella per decrypt.
\item Impossibile derivare una chiave dall'altra: se ad esempio conosci quella per encrypt non puoi trovare quella usata per decriptare
\end{itemize}
Se ho questo schema: user cripta i dati con una chiave, trasferisce ma la destinazione usa una chiave differente.\\ Problema risolto nel 1977 Rivest, Shamir, Adleman, RSA.\\ Le chiavi devono poter essere linkate, siccome sono asimmetriche e conosco K$_{ENC}$, K$_{DEC}$ è nascosta, quindi come corollario posso avere K$_{ENC}$ pubblica. Come faccio quindi a risolvere il problema di sopra: scarico l'applicazione, che contiene anche K$_{ENC}$. Tutti vedranno K$_{ENC}$, è in plaintext. L'utente vede K$_{ENC}$, tutti lo vedono quindi chiunque può criptare qualcosa: ma nessuno vede la chiave K$_{DEC}$. Quindi, se uso la K$_{ENC}$ per criptare i dati da mandare la server, nessuno potrà decriptarli.\\ Public key = K$_{ENC}$, private key = K$_{DEC}$ (conosciuta solo da 1).\\ Perché continuo ad usare symmetric key:
\begin{itemize}
\item Risolvono due problemi diversi, non posso dire che una è meglio dell'altra
\item entrambe possono essere sicuri o non sicuri
\item Asymmetric è più flessibile, ma servono protocolli più complessi.
\item (Algoritmi noti sono rotti dal quantum computing, symmetric no).
\item Asymmetric è da 3-5 OOM (order of magnitude) più computazionale complessa.
\end{itemize}
Nessuno usa asymmetric encryption per trasferire dati, si fa questo: user ha la K$_{ENC}$, deve trasferire i dati, prima di fare questo genera una chiave k simmetrica, trasferisce ENC$_{K_{ENC}}$(k).  Solo il server può leggere, quindi ricava la chiave k simmetrica ed usa un algoritmo apposito (AES) per trasferire i dati.\\ Due possibili approccio fino a TLS1.2 per fare key management:
\begin{itemize}
\item Key transport (es RSA, ora non più usata): 
\begin{itemize}
\item server manda una key pubblica, che sarà nel certificate message (non può essere trasmessa in plaintext), client può anche salvarla se sa che la riuserà
\item Client genera un segreto random, che critpa usando la chiave pubblica del server e trasportato al server
\item Ora è possibile fare symmetric key encryption.
\end{itemize}
\item Key agreement (DH algorithm), successo nel trasferire lo stesso segreto alle due side:
\begin{itemize}
\item Tutti e due generano una chiave privata e pubblica, es Y, $g^{Y}modp$ ed  X, $g^{X}modp$,se conosci $g^{Y}modp$ non puoi ricavare Y (vedi su la lezione sulle cripto hash)
\item Client e server si mandando le chiavi pubbliche, l'attacker non riesce a ricavare facilmente X ed Y. Ma non è difficile per gli end: se ho $g^{X}$ ed Y posso fare $(g^{X})^{Y}$, quindi ho $g^{XY}$ e posso ricavare le chiavi.
\item Se l'attacker li moltiplica, ottiene $g^{YX}?$?, No è la somma $g^{Y+X}$.
\end{itemize} 
\end{itemize}
Quindi TLS può usare uno o l'altro modo, anche questo viene negoziato.\\ TLS handshake:
Client manda Client Hello, server risponde con Server Hello e qui è stata negoziata la versione del protocollo TLS.\\ Ora con RSA per esempio: dopo il Client Hello, server manda una chiave pubblica certificata (attacker non può cambiarlo). Ora client applica key transport mechanism: genera segreto k shared e lo trasferisce al server criptato con K$_ENC$,  non può essere modificato (o meglio lo assumo, rotto nel 2018). Server riceve il messaggio ed estrae il segreto shared. C'è altro trick: metto nel messaggio del client anche il codice della SSL version per prevenire il downgrade attack: se l'attacker l'ha modificato, io non mi fido e invece di essere d'accordo rimando la versione che avevo scelto inizialmente invece di quella accordata: ricordo di nuovo al server che potevo usare una versione più recente di SSL, ma lui ha detto di usare una versione più bassa. 2 byte di overhead. Ora l'attacker non può modificare i messaggi: né quello con la chiave pubblica né quello con la chiave shared dal client per il server. Quindi, il server si accorge che qualcosa è andato storto: mi rendo conto che il client mi ha rimandato la vecchia versione proposta e non quella accordata, quindi stoppo la connessione. Metodo fondamentale per risolvere downgarde atttack, nel momento in cui da un certo punto in poi la cripto è up: da quando è up, ripeto quello che ho negoziato. In un protocollo non posso essere sicuro di quello che accordo in plaintext.\\ Ma riguardo cipher attack? Attacker potrebbe convincermi ad usare altro protocollo di encryption, etc... Sto proteggendo solo la versione, inoltre cosa cambia tra i dure approcci: in DH non posso trasferire informazioni nei messaggi, ho delle quantità fisse, non è un public key crypto system, usa asymmetric crypto ma non è un crypto system così com'è: non posso trasferire qualcosa, come ad esempio la versione di TLS.
\subsection{Asymmetric cryptography}
Nell'encryption asimmetrica: prendo plaintext, uso chiave chiamata encryption key (algoritmo di derivazione della chiave noto), ottengo ciphertext e mando alla destinazione. Qui si una una chiave per decriptare che è diversa da quella usata per l'encryption, sono ovviamente collegate l'un l'altra ma non si può derivare una conoscendo l'altra (a meno di conoscere altri dettagli). Spesso K$_{ENC}$ = public key e K$_{DEC}$ = private key: chiunque può criptare ma una sola persona può decriptare.\\ HTTPS/TLS:
\begin{itemize}
\item nella fase di handshake, viene mandata una segreto shared. 
\item Nella seconda fase si usa key derivation functions per derivare dal segreto shared la chiave di encyption e di integrity.
\item Refresh della symmetric encryption e message authentication usando sempre lo stesso segreto della prima fase (nel caso di nuova sessione?).
\end{itemize}
Hybrid encyption, spesso applicata. non voglio mettere su una connessione TLS, ma voglio fare encryption di un dato, esempio un e-mail ed uso una chiave pubblica, ad esempio quella dell'utente a cui voglio inviarla. Ma il messaggio è lungo:
\begin{itemize}
\item genero una chiave K simmetrica, random number
\item Uso un cipher ordinario, come AES, ed encrypto messaggio usando la chiave K ed AES $\rightarrow$ symmetric encryption.
\item Prendo k e la cripto con asymmetric cipher, e lo mando col messaggio. Mando un messaggio in cui nell'header ho asymmetric encryption della chiave, che è la chiave simmetrica usata per fare enctyption, più il  campo dati. Sfrutto la velocità e scalabilità dell'asymmetric encryption.
\end{itemize}
\subsubsection{PubKey crypto}
Encryption e decryption sono una l'inverso dell'altra: DEC(ENC(M)) = M, quindi posso fare in questo ordine: public key encryption, un crypto system in cui sto criptando i dati in modo che tutti possano criptarli ma sono uno possa decriptarli. Assumo che l'operazione sia commutativa: ENC(DEC(M)) = M, ha senso dal punto di vista algoritmico (ma non semantico). Ha senso per fare la digital signature, è la duale del public key enrcyption: nel caso della digital signature
\begin{itemize}
\item Prendo la mia chiave privata, prendo un messaggio M ed applico D$_{A}$(M), posso applicarla solo io perché solo io ho la chiave. Chi può invertirla? Tutti, quindi se trasmetto M, D$_{A}$(M): chiunque può vedere il messaggio, l'extra information TAG (AUTH) e quindi se applico al TAG la trasformazione inversa, ovvero lo encrypto posso verificare se ottengo il messaggio originale o altro $\rightarrow$ creo integrity ma con asymmetric crypto.
\end{itemize}
Posso quindi creare due applicazioni diverse: usa è per l'encryption di dati, una per la digital signature. Devo trasmettere un messaggio: applico asymmetric crypto del primo caso. Applico una hash function crypto, comprimo ed ottengo digest e nessuno può ottenere lo stesso hash con un messaggio diverso. Quindi quello che faccio è la cosa seguente: ho messaggio compresso, trasformo usando l'operazione di decryption (che usa la chiave segreta), quindi trasmetto il messaggio più uno short tag. Per verificare se messaggio è vero o no: uso la public key per invertire l'encryption dell'hash computato prima. Quindi faccio hash del messaggio verifico se i due hash combaciano. Un attacker può solo scoprire cosa c'è nel tag invertendo con la chiave pubblica, ma non può in nessun modo ricavare un messaggio m' che mi generi lo stesso hash. L'hash del messaggio ha due obiettivi:
\begin{itemize}
\item Restringere il messaggio così da applicare trasformazioni su un singolo "blocco".
\item Sicurezza: se non uso un hash, il sistema è vulnerabile, non solo per ottimizzare le perfomance, avrei un problema di malleability
\end{itemize}
Message integrity con authentication alla sorgente: in MAC, ho una sorgente, messaggio M ed il tag(K,M), ma se K è shared non posso autenticare la sorgente, non sono sicuro che l'abbia generato la sorgente, può averlo fatto anche la destinazione. Non c'è la non repudiation property (source authentication). Nella digital signature il concept è lo stesso, ma la chiave K usata nel TAG è privata ed è della sorgente.\\ Attacker può comunque prendere messaggio, sostituire la sua HASH + enc del messaggio e spacciarsi per la sorgente e rendere il messaggio valido.
\subsubsection{Basic Algorithms}
Pionieri: Diffie-Hellman key agreement, problema dell'asymmetric cryptography, vera soluzione 1977 da Riverst, Shamir, Adelman, RSA cryptosystem. Problema di usare l'approccio di DH in crypto systems fu risolto nel 1985 da El Gamal.\\ Problema generale: trovare un problema asimmetrico difficile in una direzione e facile nell'altra. esempio: hash function è un problema asimmetrico (da hash a digest facile, da digest ad hash è undefined).\\ Diffie ed Hellman trovarono problema: computazione del logaritmo discreto, ma non poterono costruire un crypto system su questo. RSA: non trovarono un modo di adattare il problema, ma trovarono altro problema:\\ ho p e q primi molto grandi, è facile fare il prodotto di p e q. Ma se ho n  = p$\cdot$q, è difficile trovare p e q, problema di fattorizzazione, approccio per trovarli è brute force (tento tutte le possibilità).\\ Ho un primo p ed un "numero" g (generator, poi scoprirò). Prendi x: $g^{x}modp$ p è di 1024 bit, x anche può essere grande. Operazione è veloce o no? Se devo elevare $g^{132412394533242543}$, ma c'è trick per renderla più veloce. Devo computare $g^{1437}modp$. Prendo l'esponente e lo traduco in forma binaria: 1437 = 10110011101$_{2}$.Lo metto in colonna in ordine inverso (dal bit meno significativo al più significativo) e creo square e multiply.
\begin{table}
\begin{tabular}{||c c c||}
\hline\hline
bit & Square & Multiply\\
\hline
1 & g & g \\
\hline
0 & $g^2$ & g \\
\hline
1 & $g^4$ & $g^5$ \\
\hline
1 & $g^8$ & $g^13$ \\
\hline
1 & $g^16$ & $g^29$ \\
\hline
0 & $g^32$ & $g^29$ \\
\hline
0 & $g^64$ & $g^29$ \\
\hline
1 & $g^128$ & $g^157$ \\
\hline
1 & $g^256$ & $g^413$ \\
\hline
0 & $g^512$ & $g^413$ \\
\hline
1 & $g^1024$ & $g^1437$ \\
\end{tabular}
\end{table}
La prima linea è di inizializzazione, nel caso ci fosse 0, metterei 1 nel multiply (altrimenti metto g). Nelle seguenti linee: se ho un 1, nella multiply moltiplico il valore di Multiply della riga sopra per il valore della Square della riga corrente, mentre nella Square faccio una operazione, ovvero il quadrato della componente precedente.\\ Quante operazioni compio: 10 operazioni di Square + 6 Multiply, passo dalle 1436 operazioni iniziali a 16. Risultato generale: $log_2(numero)$ + le Multiply: $log_2(numero)$ nel caso peggiore, 0 nel caso migliore, avg: $\frac{log_2(numero)}{2}$ quindi in totale = 1.5 $\cdot$ $log_2(numero)$. Calo da complessità esponenziale a complessità logaritmica: lineare con la taglia in bit del numero. Dlog è complesso perché, dato un y = $3^xmod104729$, e chiedo di trovare un x tale per cui y = 33490. Se la funzione fosse monotona potrei applicare algoritmi molto efficienti per trovare il valore di x. Ma in questo caso non è così, la complessità rimane esponenziale. Quindi ho il problema che volevo trovare.\\
Diffie-Hellman, protocollo di key agreement, permette di settare tra client e server un segreto shared, non trasferendolo da una parte ad un'altra, bensì usando computazioni: asimmetrico nel modo in cui ricavo il segreto shared, scambio dei valori pubblici.
\begin{itemize}
\item Alice genera una quantità random, x. Bob genera y, random.
\item Alice manda a Bob $g^xmodp$, si computa facilmente. Bob manda $g^ymodp$
\item Alice prende $(g^y)^xmodp$ ed ottiene k. Bob fa lo stesso, ed ottiene la stessa chiave k. Ho applicato di nuovo modular exponentiation, veloce.
\item L'attacker ha visto lo scambio, ma non può computare la chiave perché deve invertire il dlog. Da $g^x$ e $g^y$ è difficile computare $g^{xy}$. Bob ed Alice hanno i "segreti" x ed y, quindi per loro è facile fare la computazione.
\end{itemize}
esempio: p = 29'996'224'275'833, g = 3 (sono parametri). x = 123456789, y = 234567890. Alice computa $g^xmodp$, Bob $g^ymodp$. Poi, alice farà $(g^{y})^{x}modp$, mentre Bob farà $(g^{x})^{y}modp$, ed avranno la stessa chiave K. In Diffie-Helmann: numeri di almeno 1024 bit.\\ La sicurezza dell'algoritmo si basa sul fatto che l'attacker deve per forza invertire uno dei due dlog: se conosco di più, ovvero x o y, allora posso riuscire a ricavare k: questa è la proprietà di asimmetria.\\ Limiti di DH:
\begin{itemize}
\item Non implementa un crypto system: risolve un problema, ora che ho la chiave K shared, posso usare AES-128$_K$(data), quindi ho risolto il problema pratico di trasferire i dati su un canale non sicuro. Ma non risolve la challenge originale: prendi un messaggio M, criptalo con una chiave K$_{ENC}$ e decriptalo con K$_{DEC}$
\item Non posso derivare facilmente una digital signature, non posso usare la chiave SK per la DS (non avrei la non repudiation property)
\end{itemize}
1977: RSA per risolvere il problema (ora non si usa più perché non ha implementazione efficiente su curve ellittiche).
\subsubsection{RSA Algorithm}
Algoritmo i Rivest, Shamir e Adelman; patented fino al 2000.\\ Problema asimmetrico difficile: dato N = p$\cdot$q, è difficile fattorizzare N; p e q devono essere numeri primi grandi.\\ Operazioni di encyption e decyption sono semplici, e sono modular exponetiation. Questo può supportare sia l'encryption che la digital signature: posso modellare un crypto system ed avrò i due servizi a seconda dell'ordine delle operazioni.\\ Principio dietro RSA:\\ $m^xmodN$, c è un numero, quindi ogni messaggio viene tradotto in un numero, deve avere taglia $\leq$ N.\\ La funzione è periodica: se considero $3^xmod10$ = \{3,9,7,1,3,9,7,1,....\}.Il periodo è lungo 4, se provo a cambiare m, la lunghezza periodo è uguale o minore: ese. $9^xmod10$ = \{9,1,9,1,..\}, worst case period = 4.\\ Teorema di Eulero: computazione del max di $m^xmodN$, è la funzione $\Phi$(N), più formalmente se m è co-primo con N (GCD(m,N) = 1), allora vale che $m^{\Phi(N)}modN$ = 1.\\ Computazione $\Phi$(p):
\begin{itemize}
\item se p è primo, $\Phi$(p) = p-1.
\item se N = p$\cdot$q, con p e q primi, avrò che $\Phi$(p$\cdot$q) = $\Phi$(p)$\cdot$ $\Phi$(q) = (p-1)$\cdot$(q-1).
\item Numero primo $p^k$, $\Phi$($p^k$) = (p-1)$\cdot$ $p^{k-1}$.
\end{itemize}
Conseguenza della periodicità: $m^xmpodp$ è periodica con periodo $\Phi$(N)s. Prendo ad esempio $mod11$, quindi $\Phi$(11) = 10.\\ Mi rendo conto che 9$\cdot$7$mod11$ = $2^6 \cdot 2^7 mod11$ = $2^{13}mod11$. Quindi posso lavorare sull'esponente facendo mod $\Phi$(N): $2^{13mod10}$ = 8 = $2^{3mod10}$ = $2^3$.\\ Conseguenza: $m^x = m modN$ se x = $1 mod\Phi(N)$, in questo esempio x = 1 + k$\Phi$(N).\\ Quando $3^xmod10$ = 3? $3^5mod10$ = 3, ma l'equazione si risolve ogni volta che x = $1mod\Phi(N)$.\\ Costruzione di RSA:
\begin{itemize}
\item Genero p e q primi grandi e li tengo segreti.
\item Computo N = p$\cdot$q e lo rendo pubblico, trovare la fattorizzazione è difficile (escludendo quantum computing la complessità cresce esponenzialmente con il numero di bit di N)
\item Computo $\Phi$(N) = (p-1)$\cdot$(q-1), non posso computarlo conoscendo solo N, devo conoscerne la fattorizzazione. Ma ora ho il problema del mio recevier: la sicurezza di RSA risiede nel fatto che nessuno può computare $\Phi$(N).
\item Genero una public key e: 1 $<$ e $<$ $\Phi$(N), e deve essere co-primo con $\Phi$(N). (e è sempre dispari perché N è pari)
\item Genero chiave privata d tale che e $\cdot$ d = $1mod\Phi(N)$. Quindi $(m^{e})^dmodN$ = m.
\item Per risolvere: $m^x$ = $m mod N$ devo risolvere x = $1mod\Phi(N)$, ora ho $(m^e)^d$ = $m mod N$, devo risolvere $e \cdot d$ = $1 mod\Phi(N)$. Se conosco $\Phi$(N) il problema è facile,altrimenti no.
\end{itemize}
Assunzione di sicurezza: dato N deve essere difficile trovare i fattori p e q, inoltre nno deve essere possibile computare $\Phi$(N), e senza $\Phi$(N) è difficile computare d ed e.\\ d è la chiave di decrypt (privata), mentre la chiave di encrypt (pubblica) è la coppia (N,e).\\ Perché funziona: se ti do N,e ed un messaggio cryptato $m^emodN$, è difficile trovare x tale che $(m^e)^xmodN$ = m. Unico modo è fare brute force a meno che non conosci $\Phi$(N). Problema dell'aritmetica $modN$: le frazioni non si considerano, le soluzioni sono tutte intere.\\ esempio: p = 11, q = 17 (segreti), computo N = p$\cdot$q, 11$1cdot$17 = 187. $\Phi$(N) = 10$\cdot$16 = 160 (segreto). Prendo e = 7, che è pubblico e primo con 160.\\ Ora, cerco d tale che e$\cdot$d = 1mod160, in questo caso d = 23.\\ CHi computa ha la trapdoor (conosce $\Phi$(N) che è quel qualcosa in più), rende pubblico N ed e. Per criptare M: C = $M^7mod187$, per fare decyption faccio $(M^7)^23mod187$.\\ Come funziona: sono la banca, genero p e q localmente e li moltiplico per avere N. Ora genero e e siccome ho computato io N, conosco $\Phi$(N) e posso generare d. Dico all'utente di usare \{N,e\} come chiave pubblica, una volta fatta l'operazione mi tengo solo \{N,e\} d in privato. L'attacker può vedere il ciphertext, ma non può decriptare $m^emodN$.\\ NB: messaggio numerico deve essere più corto di N (in termini di digits).\\ L'attaccante deve cercare tutti i numeri possibili: se riprova con e, ottiene qualcosa di diverso (non è symmetric encr, a meno che non conosca la fattorizzazione p e q), mentre per la banca è semplice perché ha la chiave privata d per decypt.\\ N è grande, p e q anche. e può essere piccolo, mentre d è sempre grande: è utile spesso selezionarli entrambe grandi ma è possibile selezionare e piccolo (es = 3, così l'encryption è veloce). La probabilità che d sia piccolo è infinitesima, di solito si fissa una chiave pubblica e, da cui poi tanto si computerà d, non posso selezionare entrambe.\\ (Non raccomandato scegliere la chiave e piccola perché posso esserci attacchi, per ottimizzazione si può scegliere piccola, ma si apre a delle vulnerabilità).\\ Perché RSA funziona, trapdoor funciton: diventa facile computare $(m^e)^dmodN$ = m, con algoritmo di Euclide esteso:\\ assumo di avere due numeri coprimi, esempio 51 ed 11; GCD[51,11] = 1. Devo trovare a e b $\in$ Z tali che 51$\cdot$a + 11$\cdot$b = 1.\\
1x51 + 0x11 = 51\\
0x51 + 1x11 = 11\\
Divido 51 per 11, segnando il resto: 4, r = 7. Prendo l'ultima riga, la moltiplico per 4 e la sottraggo a quella sopra:\\
1x51 - 4x11 = 7.\\ Ora divido 11 per 7, ottengo 1 con r = 4.\\ -1x51 + 5x11 = 4.\\
2x51 - 9x11 = 3.\\ -3x51 +14x11 = 1.\\ Ho trovato (a,b) = (-3, 14); complessità logaritmica, efficiente dal punto di vista computazionale.\\ Applico per computare l'inverso modulare: \\ ho e = 13 e $\Phi$ = 60, devo computare l'inverso di e: so che è co-primo con $\Phi$(N), devo cercare $\Phi$ $\cdot$a + e$\cdot$b = 1. Applico l'algoritmo di Euclide esteso e trovo (a,b)=(5, -23). La riscrivo isolando 13$\cdot$(-23)  = 1 - 60$\cdot$5, ma l'uguaglianza vale anche per il modulo: $13\cdot(-23)mod\Phi(N)$ = $1mod(\Phi)(N)$, quindi ho trovato che d = -23 = 60 -23 = 37, quindi ora per decryptare mi basta elevare il ciphertext alla d.
\\RSA signature: prendo messaggio, faccio hash ed applcia DEC(H(M)). Quindi neòl caso di RSA faccio $H(M)^dmodN$, e mando il messaggio, la chiave pubblica e il TAG. L'altro end prende TAG, lo eleva alla e quindi ottiene $(H(M)^d)^emodN$ = H(M) e può controllarlo facendo hash del messaggio.
\subsection{Digital certificates and public key infrastrucutures}
Digital signature: so come generare chiave pubblica e privata. Creo H(M) sul messaggio che voglio inviare, lo faccio perché la condizione per fare aritmetica $modN$ il messaggio deve avere taglia più piccola di N, così lo ottengo. Trasformo il dato (encyption è termine improprio qui) facendo H(M)$^dmodN$: solo io posso fare questa computazione, ma chiunque può fare la trasformazione inversa (H(M)$^d$)$^e$ che mi fornirà H(M), ora computo di nuovo H(M) e controllo se torna con quello ricavato.
\subsubsection{Problemi}
Un attacker cerca di rompere la costruzione: prende q' e p', scelti da lui, computo N', genero chiave pubblica e' e chiave privata d', posso farlo perché conosco $\Phi(N')$. Modifico il messaggio M, faccio hash del così ottenuto M' e lo firmo con la mia chiave segreta d'. È vero che l'utente iniziale ha la chiave privata, ma l'utente finale deve avere la chiave pubblica e:
L'utente finale ha già la chiave e salvata, ma come posso avere salvato ad esempio la chiave pubblica per un utente che non conosco? Devo per forza ottenerla dalla rete. Attacker può intercettare la comunicazione, dire di essere l'utente originale e rimpiazzare la chiave pubblica con la sua, e'. Quindi verificando il messaggio, questo è autentico ed è un problema.\\
Altro problema, RSA key transport: voglio connettermi alla banca con TLS, mando Client Hello e la banca deve mandarmi la chiave pubblica, scelgo una quantità random k e la trasferisco usando la chiave pubblica della banca.\\ Ma quando la banca mi manda la chiave pubblica, l'utente può mettersi nel mezzo e sostituirsi alla banca. Può farlo perché la comunicazione è attualmente in chiaro , quindi quando l'utente sceglie la random k, la manda criptata con la chiave dell'attacker, quindi attacker lo decrypta, lo legge e lo riencrypta con la chiave pubblica della banca rimandandolo a quest'ultima.\\ Ora l'attacker sa che il data exchange sarà criptato con la chiave k.\\ Problema anche in Diffie-Hellman: i due utenti Alice e Bob scelgono due chiavi e si mandano $g^xmodp$ e $g^ymodp$. Attacker si mette nel mezzo: sceglie una chiave z random, ferma la trasmissione di $g^xmodp$ e genera $g^zmodp$. Dall'altra parte, manda anche all'altro end $g^xmodp$, quindi lo manda sia a Bob che a Alice.\\ Alice computerà una chiave K$_{1}$ = $g^{xz}modp$, Bob computerà K$_2$ = $g^{yz}modp$. Ma l'attacker li ha entrambi: ha intercettato i due messaggi $g^xmodp$ e $g^ymodp$ quindi può elevarli alla z, che ha generato lui; attacker agisce da proxy.\\ Quindi ogni applicazione dell'asymmetric cryptography ha il problema del certificato della chiave pubblica: o ho TUTTE le chiavi salvate, oppure se devo recuperare la chiave dalla rete sono vulnerabile ad un MITM.\\ 3 scenari differenti, stesso problema: ho un nome a cui è associata una chiave pubblica, tutti gli attacchi rompono l'associazione e cambiano una delle due parti. Bisogna legare, in termini di cryptographic bind tra il nome e la chiave pubblica.
\subsubsection{Digital certificate}
Digital certificate: qualcosa che mi permette di legare (in senso molto stretto) una chiave pubblica ad un soggetto, soggetto può essere persona, compagnia, entità legale. Devo fari si che l'associazione sia crypto binded, non rompibile.\\ Non si può risolvere il problema a meno di aggiungere un trick extra: non ho la crittografia attivata: i messaggi sono in chiaro, devo ancora attivare la crittografia.\\ Mi riferisco ad una terza parte fidata (certification authority), a cui chiedo : il nome è associato alla chiave pubblica?\\ Ma se la risposta avviene on-line, c'è il problema del MITM di prima.\\ Prendo il nome, la chiave pubblica e chiedo di firmare digitalmente il messaggio che contiene sia il nome che la chiave pubblica: es Flavia $|$ 123456 diviene un unico messaggio digitally signed dalla certification authority, l'assunzione immancabile è la fiducia nella certification authority.\\ Certificato:
\begin{itemize}
\item Fase 1: sono una banca e voglio generare chiave pubblica e privata, devo farlo offline, non voglio trasmettere online nulla.Genero in locale e salvo in locale la chiave pubblica.\\ A questo punto, offline, chiedo alla certification authority di firmare il nome della banca e la chiave pubblica.\\ La CA è sicura che sia la banca a richiedere la digital signature: devi presentarti lì di persona, mostrando documenti etc..., fase complessa con aspetti legali.\\ Infine, mi arriva il certificato CERT = (Bank\_Name, BankPK)$_{CA_sign}$, l'integrità del messaggio è garantito dalla digital signature della CA.
\item Fase 2: un customer si collega alla banca, e questa mi manda il certificato, che contiene il nome e la pubblic key. In TLS avrò client hello, server hello e poi il certificato che contiene nome e public key che sono cryptographically bounded. Ora il customer deve assicurarsi che il CA sia trusted, nel PC c'è lista trustata: controlla nella lista e se è trustata deve controllare la correttezza della signature. Come faccio per fidarmi della digital signature: Browser - settings - security - manage certificates, si apre un box che contiene la lista di certificati fidati. Quindi cosa vuol dire fidarsi di una CA: la CA firma il messaggio che contiene (nome,PK) della banca, quindi c'è un messaggio e questo messaggio ha un HASH, e se uso RSA questo hash sarà elevata alla private key della CA, la chiave non è dell'entità d i cui si fa il crypto binding ma quella della CA. Quindi per invertire il TAG mi serve la chiave pubblica della CA, ma deve essere pre-installata nel PC, altrimenti sono soggetto ad un MITM. Quindi fidarsi della CA è molto forte: anche la public key della CA è nella forma di un certificato, quindi avrò crypto binding tra il nome della CA e la chiave pubblica. Ma chi firma questo crypto binding? Può essere self-signed, in quanto ci sono root authorities che non possono fare altro se non firmarsi il certificato da sole. In principio non bisognerebbe fidarsi della self-signed certificate a meno che non è firmato da una root authority.\\ Trick per sito che vuole un certificato può essere self-signed, ma questo non vuol dire che sia vero.\\ 
spazio utile: ssllabs per testare certificati dei siti.\\  
\end{itemize}
Ma sono sicuro che sto parlando con la banca? Fin'ora ho solo verificato che la chiave pubblica inclusa nel certificato è associata al nome incluso nel certificato, ma chi mi di dice che il certificato me lo ha mandato la banca? Non è questo il ruolo del certificato: non mi garantisce che sto parlando con quel nome. Un attacco di questo tipo ha senso? Può avere ripercussioni il problema che chi sta dietro il certificato non è l a banca reale?\\ Quando l'utente manda la chiave pubblica, prima si è riferito al CA che ha fatto il crypto biding della coppia (nome $|$ public key). Ora l'utente può prendere il messaggio e fare replay, ma non può sostituire il nome o la chiave e firmarlo, dovrebbe avere la chiave dell'utente. Quindi il certificato è sicuro: il messaggio è stato prodotto usando la chiave segreta associata al segreto, ovviamente l'attacker non può mandarti un certificato valido perché deve fare verifica off-line burocratica.\\ Attacker non può più sostituire nulla nel crypto binding: l'associazione è forte, quindi risolve problemi della digital signature.\\ Risolve l'altro problema? Devo provare che l'entità con cui sto parlando possiede la chiave privata associata alla chiave pubblica. Come fare a dimostrare che possiedo la chiave privata associata al certificato? Chiedo all banca di firmare qualcosa di fresco oppure chiedo alla banca di decryptare qualcosa di fresco. Quindi:
\begin{itemize}
\item Banca mi manda certificato con chiave e chiave pubblica, di cui verifico la certezza (è firmato dal CA). Posso ottenere ora la chiave pubblica della banca, e sono sicuro  che sia autentica.
\item Mando una nonce alla banca e chiedo  di firmarlo
\item Banca ritorna la nonce firmata con la sua signature.
\item Ora applico la chiave pubblica ricavata prima per invertire la nonce che è stata modificata dalla banca per vedere se mi torna.
\end{itemize}
Ma ora faccio la duale:
\begin{itemize}
\item Banca mi manda il certificato
\item Io trasformo la nonce usando la chiave pubblica della banca, chiedendo il risultato
\item Ora, la banca mi manda il risultato pulito usando la sua chiave privata
\end{itemize}
Quindi, i certificati non garantiscono che la persona sia reale, ma un protocollo deve includere l'uso di certificati per poter provare l'autenticità dell'entità, in TLS uso i due meccanismi appena visti uno server side ed uno client side.\\ Approccio di TLS con RSA key transport:
\begin{itemize}
\item Ricevo il certificato dalla banca
\item Non genero la nonce, bensì la chiave simmetrica che userò dopo in encryption. Mando la chiave criptata con la chiave pubblica.
\item Ora posso scambiare dati criptati usando AES-128$_{K}$: quindi se eri la banca bene, puoi decyptare, altrimenti non puoi.
\end{itemize}
Difficile però includere diverse root authorities in un singolo PC, idea è di avere fiducia in una catena di certificati: mi voglio connettere alla banca, mi manda certificato firmato da un CA non in lista, quindi posso avere una gerarchia di CA, così vedo il certificato del CA che ha firmato quello della banca e vedo se è trusted.
\subsection{Public key certificate}
Public key certificate è una struttura dati che fa binding tra una chiave pubblica ed il suo legittimo proprietario. Approccio base: mi affido alle CA, che rilasciano CERT$_ID$ = binding del nome e della public key. esempio: sito web protetto con TLS: voglio proteggere l'URL del sito. Mi fido del certificato che mi viene fornito perché è rilasciato da un CA di cui mi fido. esempio: certificato di Bob: contiene chiave pubblica, CA identity CA$_id$, CA signature del certificato di Bob.\\ Devo anche verificare che il server che mi manda questo certificato abbia la private key: mando challenge (nonce) e questa nonce viene firmata con la chiave privata del server.\\ Ora il protocollo di sicurezza può partire, questo ad esempio avviene in TLS.
\subsubsection{Public key infrastructure}
Un PKI consiste è il set di tools che serve per creare, distribuire, revocare un certificato per chiavi pubbliche.\\ Formato tipico per un certificato è X.509, ma per definire un PKI servono molti altri meccanismi: Public key Cryptographic standards, ogni PCKS\#n è riferito ad un determinato servizio.\\ Formato del certificato X.509, definisce tutti i campi ed encoding per un certificato per chiave pubblica:
\begin{itemize}
\item version, altri dati: specifica la versione del protocollo (3 è l'ultima) e definisce altre cose:
\begin{itemize}
\item validity period: il certificato è valido per un certo periodo di tempo
\item Serial number del certificato: ogni certificato deve essere unico, identificato con un serial number unico per la CA (è locale quindi per la singola CA)
\item Altre estensioni: posso inserire nel certificato altri parametri, esempio l'uso della chiave nel certificato.
\end{itemize}
\item CA identity: chi è il CA che ha rilasciato il certificato. Rappresentato in modo gerarchico
\begin{itemize}
\item Issuer: è in forma gerarchica, CN è l'ultimo livello, in questo caso è il full name della CA.
\item Subject: CN è URL del sito web che sto certificando (se ad esempio certifico un web server)
\end{itemize}
\item User identity
\item User public key: public key, formato dipende dall'algoritmo per cui sto certificando la chiave pubblica. Ad esempio se è RSA avrò i parametri pubblici di RSA, quindi l'esponente ed il modulo N. In DH: avrò quei parametri pubblici.
\item CA digital signature: CA prende l'intero certificato e fa hash del certificato e lo firma con la sua private key: è quella legata alla public key della CA. La cosa importante è verificare l'autenticità del certificato: devo esponenziare questo campo con la chiave pubblica della CA e verificare.
\end{itemize}
Mi collego alla banca: mando HTTP GET in chiaro, ma invece di passare il messaggio direttamente a TCP lo passo a TLS (che è user library in user space) per criptarlo. Quindi parte TLS handshake.\\ Esponente RSA fissato: è provato che se seguo pattern binario la sicurezza è la stessa ma le prestazioni sono migliori (5 esponenti fissati).
\subsubsection{Certificate Signing Request}
Una certificate signign request è un messaggio mandato da un applicante ad una CA per avere certificato sulla digital identity.\\ Posso avere come approccio la generazione di tutte le chiavi fatta dalla CA, quindi anche la mia coppia private,public (cosa che succede nelle VPN), ma non funziona.\\ Il formato più comune per CSR è PKCS\#10. Come funziona:
\begin{itemize}
\item L'applicant genera chiave pubblica e privata
\item Genero CRS che contiene informazioni che identificano l'applicant, l'X.509 subject feild, le  estensioni e la prova che posseggo la chiave privata. Quindi firmo con la mia chiave privata
\item CA deve verificare che posso chiedere un certificato per il dominio che richiedo, non è una cosa standard:
\begin{itemize}
\item Manda e-mail all'e-mail trovata come maintainer del domain, c'è un link di verifica
\item CA mi richiede di creare qualcosa sul dominio che mantengo, esempio creare un record .txt per quel dominio
\end{itemize}
CA verifica la signature che ho fatto usando la mia chiave privata. Se tutto va bene crea certificato X.509, potrebbe inserire estensioni (a partire dalla versione 3): authority key identifier, subject key identifier, key usage (posso usare la chiave pubblica solo per una specifica azione), alternative names, basic costraint extension (molto importante).
\end{itemize}
\subsubsection{Root certificates}
Come certifico le CA? Anche la CA ha un suo certificato, e chi fornisce il certificato è un'altra CA. Struttura gerarchica, la root CA è il livello più alto della certification chain: è una CA che si auto-certifica l'identità. Un root certificate è un certificato in cui subject ed issuer sono lo stesso. Come posso fidarmi di un self signed  certificate? Chiunque potrebbe farlo, quindi come faccio a capirlo: root certificates sono built in nel sistema operativo e non possono essere rimossi da utenti senza privilegi.
\subsubsection{Certificate chains}
In molti scenari reali, il certificato non è rilasciato dalle root CA, che sono poche quindi approccio non scalabile.\\ Uno o più CA intermedie:  ho una catena di certificati, nella catena non ho per forza le stesse entità intermedie. Certificate chain è una lista di certificati che è formata da uno o più CA con una serie di proprietà (solitamente  inizia con una end-entity certificate):
\begin{itemize}
\item L'issuer di ogni certificato matcha il subject del certificato successivo della lista
\item Ogni certificato 
\end{itemize}
A volte web server non mandano root certificate (perché può non essere nel SO).\\ Ma il chaining è pericolo? Ho una catena di certificati validi, posso generarne uno fake: creo un certificato fake ma legittimo e lo firmo con il CA finale (che sarebbe ad esempio il mio server).\\ Non è possibile: c'è un check, il basic constraint extension. Se nel campo certificate authority c'è NO (false) (e questo c'è nel certificato dell'end user) vuol dire che con la coppia chiave pubblica$|$privata non posso firmare altri certificati.
\\Wildcard certificate:
Certificato valido per tutti i sotto dominii di tutto un dominio, esempio *.google.com perché in pratica una compagnia ha diversi servizi associati ai vari sub domains.\\ Periodicamente, le CA devono rilasciare una lista di revoca dei certificati, che gli utenti possono controllare. Come faccio ad ottenerla: inserisco la lista in una estensione specifica, distribution point da cui è possibile scaricarla.
\subsubsection{Let's build our own authority}
OpenSSL x.509: OpenSSL è toolkit crittografico composto da 3 componenti (librerie scritte in C). openssl ha variante di RSA basato sul teorema cinese del resto.\\ Alcune applicazioni vogliono un unico Ca certificate bundle con tutti i certificati: quindi metto tutti i file in uno solo, concatenando il root e l'intermediate.\\Ora voglio provare ad usare Apache2 per configurare l'uso del certificato: proteggo il server http con i certificati che ho creato.\\ Apache2 supporta il meccanismo del virtual host: se voglio installare più web server su un unica macchina, avrei bisogno di più ip. Oggi posso avere più siti web su un unico web server con i virtual host.\\ ServerName è un field importante, serve per abilitare il virtual host.\\ Mi collego via browser e ricevo un warning: il certificato è valido, il browser/SO non ha la chiave pubblica della CA. Devo specificare esplicitamente https, nei siti noti c'è redirect automatico, voglio averlo anche io: lo metto nel file configurazione di apache2.\\
\subsubsection{HTTPS Downgrade Attack}
Siti ibridi, homepage in HTTP e login in HTTPS e questo era una vulnerabilità, homage non protetta è vulnerabile ad un impersonification attack: mi metto nel mezzo tra server e vittima, performo HTTPS con il server ma HTTP con la vittima.\\ Posso pensare ad un downgrade attack: identifico la vittima:
\begin{itemize}
\item Mi metto in mezzo alla vittima ed il default gateway.
\item Faccio redirect dei pacchetti che hanno l'IP del sito localmente.
\item La vittima si collega a me e replico con una versione del website.
\item Faccio fare l'autenticazione
\end{itemize}
HSTS: Http Strict Transport Security: se provo a collegarmi ad un server con http, il server mi risponde di usare HTTPS, risponde con un cookie che specifica che devo usare https ed ha un tempo di fine (1h). Prima query è in chiaro, quindi da un punto di vista teorico la vulnerabilità rimane.\\ Alcuni browser web hanno incluso una lista di siti che contiene i siti più noti che supportano HSTS.\\ Non tutti i client supportano HSTS e c'è comunque vulnerabilità a DNS Spoofing Attack.\\
\subsection{Diffie Helmann protection}
Anche qui avevo il problema del MITM: Alice e Bob hanno g, p, e fanno exponentiation della loro chiave privata random x ed y.\\ Attacco è possibile: attacker selezione z e manda ad entrambi $g^zmodp$ e quindi entrambi mandano la loro chiave all'attacker, quindi il data exchange passa per l'attacker nel mezzo (può ricavarsi tutte e due le chiavi per cifrare i messaggi e mandarli ai due end).\\ Senza dare nulla, l'approccio DH può essere attaccato via MITM ed è detto Anonymous DH.\\ IETF, protocollo BTNS (versione light di IPsec, Bettern Than Nothing Security), che è il DH Anonymous: so che c'è una vulnerabilità ma è meglio che non sapere nulla. Approccio base di DH è questo, quindi vorrei fixare il protocollo: 
\begin{itemize}
\item Alice e Bob scelgono x ed y e chiedono alla CA di certificare i valori. Ovvero ottengo crypto binding tra il nome ed il valore di chiave pubblica. Ma x non è una chiave pubblica, bensì una quantità pubblica che corrisponde ad una quantità privata che solo loro hanno.
\item Chiedo alla CA di ricevere qualcosa che è [Alice, $g^xmodp$]$_CA$ e Bob, $g^ymodp$]$_CA$.
\item Ora attacker può generare z, ma non può sostituirsi ad Alice o Bob, perché non può ottenere il certificato. 
\end{itemize}
Questo è il fixed DH exchange. Ho un altro problema: i valori $g^xmodp$ e $g^ymodp$ ora sono fissati, suppongo che mi collego alla banca nel 2018. Quando mi collego, avviene scambio dei certificati, il segreto che computo è $g^{xy}modp$. Mi collego nel 2020, ma non posso cambiare quantità x, perché il certificato è valido per una durata di anni (è processo legale che richiede vari step, non puoi farlo in 5 min): ogni volta che mi collego, computo sempre lo stesso segreto, quindi la chiave è sempre la stessa. Questo può o non può essere un problema, ma non è una best practice: ci sono nazioni che fanno attacchi severi, si salvano il tuo traffico per anni. Log del traffico dei cittadini: è criptato, ma aspetto per un tempo ragionevole e portò rompere la chiave privata dal tuo pc e scoprire il tuo traffico criptato.\\ Protocollo che non permette questo soddisfa la proprietà perfect forward secrecy.\\ vorrei non usare quantità fisse: se non faccio nulla, sono vulnerabile (DH anonymous), se certifico, ho fermato il MITM ma uso sempre lo stesso segreto nell'agreement ($g^{xy}$). Cosa posso fare: Alice genera chiave pubblica standard usata in una digital signature e ha il certificato della CA. Ma ora posso dire che il DH public coefficient ed la mia identità e firmarla da solo. Bob può prendere il certificato, prendere chiave pubblica di Alice, che si aspetta sia certificata dalla CA. Non è un certificato self signed: Bob riceve due quantità da Alice, uno è il certificato vero e l'altro è la DS del coefficiente pubblico di DH.\\ Così:
\begin{itemize}
\item Evito MITM: l'attacker non può computare nessuno dei due certificati.
\item Posso generare un segreto fresh ogni volta.
\end{itemize}
Ephemeral DH: ho due certificati (A, $g^xmodp$)$_Ska$, che è dinamico, viene generato localmente e il Ska cambia sempre; (A, Pka)$_{CA}$. Garantisce perfect forward secrecy? Trump mi fa rompere la CA, può vedere le mie connessioni passate? No, perché anche se qualcuno frega la chiave privata della CA, il segreto $g^xmodp$ va perduto perché le x ogni volta cambiano.\\ Quindi anche se la chiave privata viene scoperta, la sicurezza degli scambi precedenti è sicura.\\ Non c'è un singolo DH, ma 3 versioni:
\begin{itemize}
\item DH, MITM probelm
\item Fixed, long term secret
\item Ephemeral, la migliore
\end{itemize}
In TLS:\\
DH senza nulla è la versione fixed, ma devo specificare l'algoritmo usato per la digital signature dalla CA: può essere RSA, DSS etc...
\subsubsection{Symmetric vs Asymmetric}
Nell'handshake, quando viene mandato il certificato:
\begin{itemize}
\item In DH non c'è certificato, solo 1 messaggio
\item In fixed DH mando dopo il server hello il certificato e poi verifica di client
\item In RSA key transport: mando il certificato e niente altro.
\item Solo in ephemeral DH uso tutti e due i messaggi: mando certificato della Pk ed nel server key exchange mando $g^x$ firmato da me.
\end{itemize}
\subsubsection{Interlude: entity authentication con asymmetric crypto}
Problema è che il certificato non basta, non è authentication perché chiunque può mostrarlo. La cosa giusta da fare nell'authentication è la prova che conosco la chiave privata legata alla chiave pubblica. Certificato è solo binding tra nome e chiave pubblica, ma l'authentication è veicolata dal fatto che conosco la chiave privata associata.\\ So come provare il certificato: 
\begin{itemize}
\item Faccio digital signature di una nonce + ogni altro plaintext, e restituisco la signature fatto con chiave privata ed il certificato. Posso firmare solo se possiedo la chiave privata collegata alla chiave pubblica, ovviamente la chiave pubblica deve essere legata all'identità
\item Uso encyption: mando public key all'authenticator,  authenticator mi manda nonce + text ed encrypta usando la pub key e provo che sono autentico mandando indietro la nonce + ogni altro testo, dual approach. Ma se faccio MITM: vedo lo scambio dei 3 messaggi, ma dopo messaggio finale MITM può rompere la connessione e cominciare a parlare con  l'authentication. Se dopo authentication proseguo con clear text, attacker può rompere la sessione e mettersi al posto del client. Ma se compito è aprire una sessione dopo l'authentication, devo fare altro oltre l'encryption per esempio AKA, ovver authentication and key agreement.
\item Provo che so decyptare, ma oltre questo l'authenticator mi manda anche una symmetric encryption della challenge, includendo k nella asymmetric encryption. Ora, se posso scoprire k (e posso perché ho la chiave privata associata alla pubblica con cui è stata criptata la chiave) allora posso dimostrare di essere autentico risolvendo la challenge. Questo è ciò che viene fatto da TLS.
\end{itemize}
\subsection{Ancora sul TLS handshake}
Fase 3 dell'handshake (però non dirlo a nessuno perché è notazione del prof), il client risponde e posso autenticarlo, usando questo approccio:
\begin{itemize}
\item Server mi manda nonce, client risponde mostrando certificato e signature della nonce. Dov'è la nonce? Prima di questa fase ho avuto client hello e server hello, nel server hello è stata trasferita la nonce e questi messaggi sono utili per garantire che per sessione la nonce sia fresca
\item Client key exchange: trasmette chiave simmetrica o le informazioni per generare la chiave server side (DH o RSA)
\item Certificate verify, che deve contenere la firma della nonce + testo. Il fatto di cui mi interessa è che ci sia la nonce, se c'è altro non è un problema (applico hash per firma e poi manipolo quello). Devo includere signature della nonce, ma se aggiungo del testo, se questo testo è noto da server e client non da problemi. Idea interessante di TLS: includo anche la nonce del client, questo previene attacchi di tipo reflection migliorando la sicurezza. In muthual authentication devo fare crypto binding delle due direzioni della comunicazione. Posso includere anche il certificato, il server key exchange, il certificate request. Siccome uso TCP e so che i messaggi sono consegnati in maniera affidabile, posso creare un singolo grande messaggio che collega tutti i messaggi mandati e ricevuti nell'ordine. Metto client hello, poi server hello, poi certificate etc..., giustapposizione dei messaggi. Questo conterrà nonce del client e nonce del server. Siccome trasmissione è reliable ed in ordine il server avrà lo stesso ordine di messaggi, quindi avrà lo stesso grande messaggio. Firmo il messaggio e nel certificate verify mando il tag del messaggio grande.\\ Se qualcuno prova a fare MITM e modifica client hello per fare ad esempio un downgrade attack o per rimuovere un cipher: ora il log a sender side diviene diverso da quello ricevuto dal server. Quindi anche se va tutto bene, quando firmo il messaggio, il server proverà a verificarlo e vedrà un hash diverso. Quindi includendo tutto il log (messaggi scambiati fin'ora) posso proteggermi da MITM o downgrade attack. Avevo risolto il downgrade attack sulla versione di TLS (usando 2 byte per ripete la versione di agreement), ma ora posso rendermi conto del cambiamento di ogni singolo byte per via della proprietà di anti-collisione dell'hash. Posso proteggermi da tutti i downgrade attack (anche detti biddown se attacco qualcosa di specifico, come rimuovere un chiper). C'è un però: è opzionale, certificate e certificate verify sono opzionali.
\end{itemize}
È quindi possibile risolvere questo attacchi biddown e downgrade una volta e per tutte: mando tutto in clear text, nel momento in cui passo a protected mode (encryption on) ripeto tutto lo scambio.
Fase 4: sicurezza di TLS risiede nel messaggio finale: metto l'hash di tutto quello che ho visto fin'ora (fino al change chiper spec), quindi il server mi rimanderà finished; entrmabi proveranno di aver ricevuto tutto. Finish message è ciò che mantiene TLS sicuro. Serve per passare all'attivazione dell'encryption, ma la parte più importante è il messaggio di finish, in modo da evitare che qualcuno abbia fatto MITM.Authentication del server avviene in questa fase: prima d'ora non sapevo che il server fosse autentico.\\ Client authentication avviene nella fase del Crtificate verify, il server mi ha mandato la sua pubkey nel 3° messaggio, ma capisco che è autentico nella fase finale, quando mi manda il finish criptato con la chiave simmetrica k negoziata.\\ L'autenticazione del server è in qualche modo implicita, capisco che sto parlando con il server reale solo alla fine.\\ Perché non è possibile fare encryption senza authentication (mentre il contrario sì)? Si romperebbe la sicurezza: c'è client e server. Sono attacker, mi metto come MITM. Prendo client hello, server hello, e gli altri messaggi... log interno. Prendo il log, cambio facendo downgrade o tolgo un chiper (ai messaggi del client), e mostro al server il log' risultante.\\ Ora client manda finish message, che è l'hash del log che è tsato criptato e integrity protected con i cipher scelti. Suppongo di avere TLS\_X\_WITH\_RC4\_NULL. Quindi alla fine della negoziazione, client non sa che c'è stato errore, quindi manda un enrcyption con RC4, quindi è paddato con keystream. Non c'è integrity: siccome MITM sa cosa ha trasferito al server, sa cosa c'è in log', Mette finish del client in xor con H(log) e xor con H(log') = [H(log)']$_RC4$. Fa lo stesso anche con l'altro lato, quindi verso il client ed è game over.\\ Se disabilito integrità: grande problema, è necessaria per autenticare l'intera sessione. Pensarci bene perché è così che si protegge negotiation process / defeate negotiation attack. Secure negotiation: trasmetto dati in plaitext e li ripeto (tipicamente usando hash) quando l'integrità viene attivata.\\ Finish fa ancora parte dell'handshake, anche se è già criptato. Change chiper sec rimosso da TLS v1.3: protocollo più semplice mai fatto: 1 messaggio di un byte con un valore fisso costante (01), che serve per attivare cihper spec.
\subsection{TLS key computation}
Suppongo di aver adottato RSA key transport, la chiave simmetrica generata ai due estremi è stata generata dal client e trasferita via RSA. Chiave è generata dal client, ma chi sa se client ha i meccanismi corretti per generarla pseudorandom. Quindi problema della qualità della random key. In DH: chiave è $g^{xy}modp$, ma sono sicuro che $g^{xy}$ è uniformemente distribuita in 1,...,p-1. Non tutti i possibili esponenti possono essere implementati, non posso avere ad esempio $g^{13}$ (devo avere prodotto di due primi). Problema di uniform distribution. Anche altro problema: servono più chiavi, una per encryption ed una per integrity, ma se mi serve IV serve altro random value: servono più chiavi e fin'ora ho solo scambiato un segreto.
\subsubsection{Secret hierarcy}
Gerarchia di protocolli seri:
\begin{itemize}
\item Pre master secret è quella generata da RSA o DH durante l'exchange se può essere sempre la stessa coppia client-server. Generato durante il public key exchange, è il segreto raw iniziale.
\item Mischio la chiave alla nonce di client e server, ottengo il master secret. Così, anche se ho usato DH fixed so che il master secret cambia. Il premaster secret può non essere perfettamente uniforme, ma master secret deve sembrare come un random value (voglio alta qualità). Risolvo il primo problema. È un segreto uniforme, pseudorandom.
\item Ora mi servono più chiavi da una singola: in TLS ne uso fino a 6:
\begin{itemize}
\item encryption
\item authentication
\item IV se necessario: ho due direzioni, da client a server e da server a client. Le chiavi di scrittura (se client invia a server) sono diverse da qielle di lettura (se client legge da server)
\end{itemize}
Devo espandere il segreto, derivandone più chiavi. Posso fare abbreviated handshake in TLS, posso fare re-keying exchange: genero pre-master secret,che sarà valido per l'intera sessione, quando voglio aprire nuova TCP connection faccio nuovi client e server hello, nuove nonces client e server side. Prendo premaster, li estraggo ed espando generando le nuove chiavi che mi servono. Quindi cambio l'encyption key in ogni connessione TCP, tramite abbreviated handshake (c'è di nuovo finish message che è la parte più importante, evita ade sempio di rinegoziare due nonces uguali).
\end{itemize}
Extract then expand:
\begin{itemize}
\item Extract vuol dire aggiungere randomness nella generazione della chiave, aggiungo a premaster le nonces e genero la master key. L'idea è che la chiave deve essere uniformemente distribuita: es. AES uniformemente distribuita nel range 0-$2^{128}-1$. Uso tecnica hash-like per abere output pseudo-random: equi probabile per tutte le possibili chiavi a 128 bit.
\item Expand: una specie di PRNG che riceve un seed e lo espande come una sequenza illimitata di materiale pseudorandom. Se l'output è teoricamente illimitato posso tagliare a blocchi di ad esempio 128 bit ed ottenere le mie chiavi.
\end{itemize}
Quali funzioni usare: il blocco di expand deve essere una funzione buona, quindi sicura e veloce. Si pensa di poter combinare hash così da ottenere putput illimitato: mai combinare building blocks e fare qualcosa che non è pensato per quella specifica applicazione. TLS 2 errori:
\begin{itemize}
\item PRGN embedded nel codice e fissata, usava MD5 e SHA-1, male male. PRF è un algoritmo, va disaccoppiato dal protocollo, deve poter essere negoziato; questo può avvenire da TLS v1.2
\item Hash functions poi rotte
\end{itemize}
TLS 1.2 usa PRF fa una 1-way hash, composizione buona ma non dimostrabile sicura: funzione di espansione  (approccio naive). Derivo da un seme A$_{0}$, derivo gli altri A$_{i}$ da un HMAC chaining (A$_{i}$ = A$_{i-1}$). Si fa questo perché sembra essere più sicuro, quando Hugo Krawczyk (uno degli inventori di HMAC) vide questa applicazione, problemi: chaining accorcia la grandezza dei loop. Chaining non è una costruzione sicura.\\ Nuovo tool: Hmac Key Derivation Function (HKDF), pensato specificatamente per derivare le chiavi. Prendo HMAC ed uso un counter, non faccio più chaining;dimostrato che è sicuro, incluso in TLS v1.3. È possibile negoziare l'hash function usata nel PRF: TLS\_DHE\_DSA\_WITH\_AES-GCM\_SHA256:
\begin{itemize}
\item Diffie Hellman in ephemeral mode
\item Firmo Diffie Hellman con Digital Signature
\item Il with entra nel dominio simmetrico
\item AES-GCM cihper: authenticated encryption, quindi a che serve SHA256 finale se ho già integrity?
\item SHA256: hash da usare nel PRF. PRF è algoritmo quindi non deve essere hardcoded, inoltre deve essere possibile negoziarlo.
\end{itemize}
HMAC usato in TLS v1.2 per uno scopo diverso da quello originale (non per forza detto che è usato male), quindi questa non è una buona pratica. Fino al 2010 non c'era una funzione di derivazione di chiavi (KDF) sicura, nuova costruzione HKDF, dimostrato che è sicura: funzione derivata specificatamente per key derivation. Sembra counter mode: 
\begin{itemize}
\item Prendo master key, che è unica e ne devo derivare varie chiavi.
\item Prendo hash function e costruisco la funzione di HMAC
\item Messaggio: 
\begin{itemize}
\item prima parte è context string, arbitraria (es il mio nome). Se faccio solo HMAC$_K$(mio nome) è pseudorandom, non predicibile ma l'output è limitato a 256 bit se ad esempio uso SHA256
\item Aggiungo counter (esattamente come in counter mode), che permette di avere valori differenti
\end{itemize}
\end{itemize}
Metto context string perché, se ho la master key, ma se il SO ha negoziato la master key ed ora ho processi A e B che vogliono riusare la master key per negoziare le loro crypto keys. Ad esempio, se voglio rendere sicuro il mio sistema: uso TLS, metto nel SO il premaster/master secret negoziato e lo uso come sorgente per le chiavi dei miei processi. Ma se faccio HKDF = HMAC$_{MS}$(0), HMAC$_{MS}$(1),.. ma così se due processi generano le chiavi, avranno lo stesso set. Mettendo invece anche una label identificativa il problema non si pone. HMAC$_{MS}$(\#process$|$0)...\\ In TLS v1.2: quando computo master secret input è premaster secret, label, nonces (client e server random). Quindi è HMAC$_{premaster}$(label $|$ Nc $|$ Ns $|$ counter).\\ Quindi HKDF-Ciro = HMAC$_k$(Ciro $|$ 0), quindi metto un identificatore del processo.\\ Funzioni di extrac ed expand? Si riusa HKDF per entrambe le funzioni in pratica, in teoria c'è analisi più specifica sulla fase di extract: questo perché la teoria della HKDF si rompe se si usa premaster nell'HMAC. Ci sono quindi meccanismi migliori per fare  fase di extract, uso della stessa funzione è buona.
\subsection{TLS connection management e supporto alle applicazioni}
\subsubsection{Alert protocol}
TLS definisce messaggi speciali per fare alert (signaling) tra i vari field del pacchetto, possono essere in plain o criptati. BAD\_RECORD\_MAC è ad esempio errore fatal, mandato dal server quando integrity check fallisce. Ci sono vari alert, possono essere waring o fatal. Se è fatal la connessione termina, altrimenti se è warning il client può decidere di andare avanti o può proseguire.\\ Un alert è particolarmente importante: ricordo che TLS non protegge TCP, se invio pacchetto e c'è TCP header e TLS, la parte protetta con integrity e encryption è la parte TLS, la parte TCP è plaintext ed anche non protetta (non c'è integrità). Quindi attacker può cambiarla arbitrariamente. È possibile forgiare TCP reset packer e killare la connessione, spoofing dei segmenti TCP. Apre la porta al DoS attack. È più preoccupante un altro attacco, truncation attack: TLS fornisce encryption ed integrity, ma c'è il problema della session integrity.\\ Non ho ancora discusso della sicurezza dell'intera sessione: client manda dati al server, attacker vuole far ricevere solo la prima parte di messaggio, quindi manda TCP fin, spoofing di un FIN, quindi chiude una connessione. Attacker fa trasmettere la prima parte e poi si finge il client e chiude con FIN spoofing. Server è sotto un attacco più sottile di DoS, session integrity, truncation attack. QUando analizzo protocollo, devo garantire non solo che encryption e integrity sono on, ma anche che l'intera sessione è protetta: all or nothing. In questo caso l'attacker fa si che il server riceva solo una parte del messaggio ed ha tolto l'ultima parte.\\ Come risolvere: quando comincio a trasmettere mando taglia, ma funziona solo se è preshared (in streaming non è possibile sapere la taglia a priori)\\ Aggiungo alert, signaling message, che fa il lavoro del FIN ma dentro sessione TLS, quindi con sicurezza attiva. Idea: mando dati applicazioni, quando ho deciso di terminare la trasmissione mando un messaggio di close notify, solo dopo questo mando TCP FIN. Siccome è alert message metto warining level, è una semantica half close, server manderà la sua close notify quando avrà finito.\\ Ora è possibile scoprire l'attacco: server sta ricevendo dati e poi arriva FIN, ma non ho visto close notify sono sotto attacco. Mentre se vedo tutti i pacchetti per bene e poi vedo close notify è ok.\\ Intergrity protection ha peso maggiore dell'encryption nelle applicazioni: non è tutto solo encryption.\\ Risolvo il problema della session integrity, ma non del DoS, ma almeno capisco che sono sotto attacco.
\subsubsection{Renegotiation}
Abbandonato in TLS v1.3, in TLS originale idea era: mando TCP connection usando una chiave, poi un'altra con altra chiave... definivo una singola TLS session con singolo handshake + handshake abbreviati dopo il primo.\\ Non è l'unica cosa da fare, perché funziona bene solo per connessioni TCP piccole.\\ Oggi le TCP connection solo lunghe e vorrei cambiare la chiave ad un certo punto. Posso farlo solo se inizio una nuova connessione TCP, come faccio a farla entro una già esistente. Vorrei anche cambiare livello di sicurezza della sessione: parto da AES-128 e poi voglio passare a AES-256.\\ Renegotiation imlementata in TLS:
\begin{itemize}
\item Parto con handshake iniziale
\item Sono in encryption exchange, ma mentre scambio i dati parto con un renegotiation handshake all'interno della stessa sessione.\\ Il messaggio di renegotiation ora è protetto, sono in TLS session
\item Dopo renegotiation avrò una nuova sessione TLS con nuovi chipers e nuove chiavi
\end{itemize}
In teoria, il renegotiation handshake non è distinguibile da quello iniziale. C'è problema tecnico: se sto scrivendo del testo e mi arriva un renegotiation: come viene gestito il testo dai buffer TLS e TCP. Posso exploitarlo? Analyst Mash Ray: white box analysis, code review di TLS. Trova bug che può permettere attacco, ma il problema non è della versione di TLS bensì de protocollo. Bug era molto "creativo", quindi nessuno si prese la briga di fixarlo. Dopo averlo letto, studente scoprì come sfruttarlo.\\ Regola d'oro: rendere i protocolli semplici.\\ Bug: attacker apre TLS session col server e manda dei dati. I dati sono generati dall'attacker in plaintext e sono criptati in TLS session. Quando client comincia a parlare col server, in realtà attacker ha fatto MITM: attacker manda la negotiation all'interno della sua TLS session. Server pensa che sia una renegotiation: switcha ai nuovi parametri usati.\\ Quindi il client manda i dati, attacker ha messo preambolo a questi dati ed il server crede che il messaggio sia legittimo.\\ Dopo la creazione della sessione, attacker non può fare nulla, ma ha creato situazione di plaintext injection attack. Può fare injection di dati prima di quelli generati dal client.\\ ATtacker fa MITM col client, quando client apre TLS connection l'attacker salva messaggio in buffer, apre TLS connection col server, fa plaintext injection al server e poi forwarda il messaggio al client.\\ In teoria non si può fare molto con plaintext injection, se applicazione è semplice: vittima manda messaggio in cui vuole comprare la pizza, autentica con un cookie il messaggio. Attacker non può modificare il messaggio, ma può aggiungere preambolo: attacca HTTP, aggiungendo X-ignore-this dopo la newline, così da ignorare l'ordine nella get e lasciare solo il cookie di autenticazione.\\ Attacco all'API di twitter: twittava in chiaro password degli utenti.\\ Per prevenire renegotiation: immediatamente disabilitato, poi patchato, nel 2010 standardizzato. Renegotiation extension: cercare di far si che server possa riconoscere una renegotiation è crypto binded alla precedente, include finish della sessione prima; era di nuovo un esempio di session integrity.
\subsection{Altri dettagli sulla sicurezza dell'RSA key transport}
RSA key transport è un po' un bordello, quindi si è abbandonato il metodo.\\ RSA key transport:
\begin{itemize}
\item Server mi manda RSA pub key
\item Client cripta il premaster secret con la pub key (premaster calcolato prima). È un encryption scheme, che non garantisce integrità e la tecnica non è robusta a chosen chipertext attack
\end{itemize}
Chose chipertext attack: tutte le operazioni sono fatte $modn$. Attacker vede C = $M^e$ e si chiede se può criptare C, servirebbe M = $C^d$, ma non ha d. Assumo che attacker può accedere ad un decryption oracle, che può decriptare tutto tranne che C. Come posso decriptare se ho un oracolo così fatto, scegliendo un chipertext? Posso dare qualunque C' diverso da C: scelgo valore random r, faccio $r^emodn$, faccio poi $r^e \cdot Cmodn$ = X. Mando X all'oracolo, X è diverso da C e mi ridà $X^d$, mq ora ho $(r^e \cdot C)^d$ = r$\cdot C^d$. Ho scelto io r, mi basta dividere per r, ma sono in aritmetica modulare, siccome è difficile che r sia coprimo con n quindi posso fare inverso modulare: $r{-1}modn$: faccio $X^d \cdot r^{-1}$ = M.\\ Proprietà di non malleabilità: un chiper è non  malleabile se preso un encrytion C si qualche messaggio M, l'attacker non può creare un chipertext differente C' che si decripta in un messaggio M' che è legato in maniera sensata ad M.\\ Fix di RSA: padding, standard
PKCS \#1, primo standard per la famiglia PKCS, ovvero public key criptography Standard, specifica RSA encryption e decryption.\\ Primo probelma: suppongo m = 1, C= $m^e$ = $1^e$ = 1. Chipertext è uguale a plaintext.\\ Secondo problema: ho un oracolo che può fornirmi il cihpertext. Cosa accade se dopo 10 giorni vedo un chipertext che è uguale ad uno precedente. RSA soddisfa IND-CPA? No, vengono introdotti dei tricks, costruzione del vero RSA (non della versione Vanilla):
\begin{itemize}
\item Aggiunta di due byte all'inizio: 00 02. Metto poi 8 byte random e diversi da 0 (suppongo sia IV), in modo che due messaggi identici hanno ciphertext differente, protezione da CPA. Delimitatore di quantità random, che è 00. Dopo lo 0, partono i dati. Risolvo il problema di criptare ad esempio il messaggio 1 e risolvo il problema del CPA, ora la capacità di RSA è ridotta rispetto al $modn$ (ho meno valori possibili, visto il padding)
\item Quindi ora, quando client risponde al messaggio del server, usando la pub key per criptare, cripterà il premaster secret con il padding.\\ In SLL v3 specification: viene detto che quando server prova a decriptare il messaggio e se la decodifica non torna (formato non è corretto), invia abort message. Se decription funziona, vai avanti.
\end{itemize}
Sembra un Padding Oracle: mando un messaggio che specifica problema nella sicurezza, perché c'è un abort o un go on. Non è un'informazione grandissima, ma tanto basta.
\subsubsection{Bleichenbacher's Oracle}
Adaptive Chosen plaintext Attack, 1998. Ha scoperto che se è possibile ripetere più
volte un chosen ciphertext attack basato sui messaggi visti prima, è possibile decriptare qualsiasi messaggio se ne ho $2^{20}$. Target up in SSL v3.0, corretto in TLS v1.0+, ma non sei protetto da side channel attack, difficile correggere del tutto il problemi. In giro almeno fino al 2019, rimosso in TLS v1.3\\ È un CPA, obiettivo è decriptare un chipertext C: C = $M^emodn$, in RSA è pericoloso decriptare C perché M è il premaster secret.\\ 
\begin{itemize}
\item Scegli r e costruisci un nuovo chipertext
\item C' = C$r^emodn$ = $(Mr)^emodn$, RSA vanilla. Modifica meaningfull del messaggio M, operando sul ciphertext applico l'operazione al plaintext.
\item Sto chiedendo all'oracolo, che è il server, se il messaggio inizia con 00 02: se decryption fallisce, la modifica non è corretta, altrimenti vuol dire che è corretta ed ottengo informazioni.
\item Server è l'oracolo: manderà abort oppure procede with the session.
\end{itemize}
È possibile usare l'informazione del messaggio con questo leak.\\ Toy example: padding di RSA è il seguente: l'inizio del messaggio è 0 o 1, ho solo 8 bit come stringa iniziale. Oracolo mi diare se $(M \cdot r)$ comincia con 0 o 1. Rivela il primo bit, bastano $log_{2}n$ query per decriptare l'intero messaggio. esempio: p = 13, q = 19, n = 247 (8 bit). Prendo e = 29, d = $e^{-1}mod216$ = 149. Assumo di vedere C = 90, cos'è M: $90^{149}mod247$, ma attacker non ha d. Può fare la seguente cosa: testo se un messaggio di mia scelta inizia con 0 o 1. Mando al server 90 $\cdot 2^{29}$ = C', questo è $(C \cdot 2)^{29}$, 2 è il mio r. So che questo C' è $(M\ cdot 2)^e$, faccio test per vedere se 2M inizia con 0 o 1. Attacco in questo caso non è nemmeno adaptive, posso fare 2M, 4M, 8M... sto testando sempre se il plaintext $2^i \cdot M$ inizia per 0 o 1. Server mi dirà: \{0,1,1,1,0,1,1,1\}, quindi $2Mmodn$ inizia per 0 e così via... Ora, assumendo che il modulo fosse 256, avendo messaggio M = 11010001, quindi se fosse lineare, starei scoprendo bit per bit, ma non è questo il caso.\\ Quando mando al server $(2^i \cdot M)modn$, lo decripta e mi dice qual'è il primo bit, come attacker non so nulla, il messaggio che voglio scoprire è fra 0 e 246. Provo tutti i valori tra 0 e 246: moltiplico per 2 e faccio modn, mi rendo conto che da 64 in poi inizia per 1, poi a 123 ricomincia da 0. So che 2M inizia per 0 solo se M è in un certo range. Provo con (2M,4M), ottengo i range per le coppie (0,1), (1,1), (1,0), (0,0). Mi interessano le coppie (0,1), ad ogni step scarto dei valori. Alla fine riesco a scoprire tutti i bit e scopro il valore.\\ RSA è sicuro quanto un singolo bit: se posso rivelare un singolo bit di un messaggio, RSA è rotto.\\ Corollario: è possibile attaccare la parità, ovvero se numero M è pari o dispari.\\ In Bleichenbacher's oracle servono molti più messaggi, in quanto la matematica dietro è più complessa, devo testare se messaggio inizia per una stringa molto più lunga.\\ Attacco sempre pratico: client trasmette i dati, attacker si salva la sessione, e sa che valore ha mandato (il premaster secret criptato). Dopo 10 giorni attacker comincia: manda il primo messaggio C' per sapere se inizia con 00 02. Posso riprovare dopo un tot di giorni. Messaggio non cambia, è sempre quello: attacco non è nella sessione di TSL, ma alla creazione della sessione, il client ha finito di trasmettere. Attacco funziona perché coppia chiavi (pubblica,privata) non cambia nel tempo.\\ Attacco fu scoperto nel 1998 e corretto in TLS v1.0+, ma dopo questo vennero scoperti una miriade di side channels: alla fine lo stesso Shamir consigliò di deprecare RSA in TLS.\\ 2016: DROWN attack. Alcuni siti permettevano ancora downgrade a SSL v2, salta fuori che è peggio: due casi
\begin{itemize}
\item Server supporta SSL v2 downgrade, combinarono oracle a studio dei server, 17\% dei server permettevano SSL v2
\item Public key e private key erano le stesse del server: è come migrare il server, in quanto miglioro la sicurezza usando versioni migliori di TLS e lasciare in un altro server la stessa chiave, e questo secondo server può essere dowgradato. Riuso del certificato, ovvero attacco il server debole per rompere quello più sicuro. 16\% dei server erano vulnerabili 
\end{itemize}
ROBOT attack, 2018. Hanno Bock (consultant), comincia a testare Facebook servers. Conosce bene crittografia. Facebook supporta TLS v1.2, 1M loc, se ricevo RSA PKCS, TLS risponde in maniera propria. In 1M di loc ci sono eccezioni, molta attenzione al main software path, ma se mando un messaggio fatto male: ora il sw deve rispondere, il codice entra in side exception handling, che solitamente è meno testata. Bock scopre che mandando un messaggio valido o non valido e creando errori nel messaggio poteva ri-creare situazione in cui padding oracle Bleic. ritornasse. Protocol fuzzying: provare con parti random di un web server se è possibile  triggerare vulnerabilità.\\ Facebook fixa vulnerabilità, ma Hanno ne trova un'altra, quindi si affida ad amici che fanno systematic analysis e scoprono molti altri server vulnerabili. Implementazione di TLS è una jungla (cazzo!)\\ Contromisure:
\begin{itemize}
\item Cambiare completamente il padding, quindi ripensare completamente il deploy di RSA
\item Careful implementation
\end{itemize}
Soluzione: dimenticare TLS.\\ Take home: implementation è tricky, gli errori possono durare a lungo. Errore era in SSL originale, c'era Oracle ma senza saperlo.\\ Secondo messaggio: storia molto simile al MAC-then-encrypt, anche qui ho CCA, difficile fixare questo tipo di errori, si finisce per rimuovere il protocollo. Practical security (cross site scripting, SQL injection, buffer overflow), ma non si capisce sempre crypto attacks.\\ Dopo ROBOT's penetration, chi scoprì l'attacco andò dai siti web a dire di stare attenti ma quelli risposero che usavano military grade encryption. MALE MALE MALE.
\end{document}