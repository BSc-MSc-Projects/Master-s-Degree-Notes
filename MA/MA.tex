\documentclass{article}

\usepackage{geometry}

\geometry{
	a4paper,
	top = 2cm,
	left = 2cm,
	right = 2cm,
}

\title{Malware Analysis}


\begin{document}
\maketitle
\tableofcontents
\section{Introduzione alla RCE}
Reverse Code Engineering del malware, quando si analizza il malware bisogna ricostruire quello che fa a partire dal codice macchina: programma eseguibile o anche un frammento di codice macchina. La pratica di risalire da un frammento di codice al codice sorgente è il RCE.
\subsection{Definizione di Reverse Engineering}
Per definizione, per RE si definisce il processo di estrazione della conoscenza e degli schemidi progetto di qualunque oggetto costruito dall'uomo. È una attività già formalizzata nell'800, quinid ben definita già da molto tempo.\\ Un modo per capire come funziona l'attività del RE è di accostarla alla ricerca scientifica:
\begin{itemize}
\item il ricercatore scientifico accumula dei dati quantitativi, che non sono sufficienti. Bisogna estrarvi una conoscenza più astratta della realtà che c'è dietro e derivare un modello formale che rappresenti in modo formale la realtà che c'è dietro. Dall'insieme e dalle osservabili non è immediato derivare il modello formale, il processo è difficile e richiede intuizione, creatività: non basta leggere i dati, il modello va inventato. La stessa cosa vale per chi fa RE: il lavoro non è di tipo meccanico.
\item si usano opportune metodologie, altrimenti non ci sarebbe certezza che la teoria che si sta costruendo è sensata e può ragionevolmente spiegare i dati che si stanno osservando
\end{itemize}
I due grandi pilastri sono 	quindi intuizione e metodo.\\ La più grande differenza è che la ricerca investiga fenomeni naturali, mentre chi fa RE si concentra su gli artefatti umani: il ricercatore scientifico può fare delle ipotesi che non vengano semplicemente spiegate dal cervello, che ha dei limiti dal punto di vista dei modelli che può elaborare. Il ricercatore scientifico non ha certezza di trovare una teoria ed una forma a cosa sta osservando, mentre per chi fa RE c'è questo vantaggio. Quindi, concretamente c'è la garanzia che questa conoscenza era posseduta da qualcuno e che può essere riscoperta. 
\subsection{Reverse Code Engineering}
È il processo di ricostruzione delle finalità, degli algoritmi, delle strutture dati implementate da un programma per un calcolatore elettronico.\\ Ricordiamo che, con calcolatore elettronico si intende un qualunque dispositivo programmabile, non per forza un PC o laptop. Ci sono vari modi di riferirsi al RCE, come de-compilazione, ingegneria inversa dei programmi, in generale si parla di reversing.\\ Vedendo il reversing come una black box, l'input del RCE è una rappresentazione "a basso livello" del programma per il calcolatore, come ad esempio il file eseguibile contenente il codice macchina. L'output può essere variegato: è una rappresentazione a più alto livello, ma non possiamo dire con esattezza cosa debba esser in quanto quando si fa reversing non c'è un singolo use case:
\begin{itemize}
\item conoscere gli effetti del programma, allora l'output è la descrizione degli effetti del programma
\item voglio esaminare il protocollo di comunicazione, otteniamo la descrizione del protocollo
\item etc...
\end{itemize}
Si passa sempre da un livello di bassa comprensione ad uno più altro.\\ Ma alto e basso sono relativi, quindi occorre capire e definire cosa si intende per alto e basso livello.
\subsubsection{Motivazioni del reversing}
Cerchiamo di capire perché si fa reversing, non tutte le motivazioni hanno lo stesso livello etico o valore legale. Alcuni casi di utilizzo:
\begin{itemize}
\item abbiamo scritto un programma perdendo i sorgenti. Facciamo il reversing del programma per capire cosa faceva, questo è lecito. Se il programma è dell'azienda ed ho il consenso è lecito. È una attività che spesso si usa per programmi legacy in azienda.
\item programma di terzi, commerciale, che deve interagire con un altro programma scritto da noi. Vogliamo fare in modo che i due programmi si interfaccino bene, può essere necessario fare reversing sul programma per vedere come integrarlo meglio. Se è lecito o no dipende da licenze e legislazioni del paese
\item programma scritti da terzi, facciamo reversing per capire come rendere un programma simile più efficiente e compatto. È ancora meno lecito del caso precedente, tipicamente l'altra parte può dire che il lavoro viene sfruttato per avere vantaggi
\item siamo in una azienda che tiene alla sicurezza dei dati, per l'azienda è un problema affidarsi a programmi di 3° parti. Come fa l'azienda a fidarsi che non vengano esfiltrati i dati dai dischi da parte del programma utilizzato? Uno dei modi per verificare che sia tutto apposto è fare il reversing dell'applicazione. Sembra lecito, ma tutte le licenze dei programmi commerciali la vietano, quindi a rigore non si può fare
\item superare i meccanismi di protezione digitale, se c'è un sistema di protezione e si fa il reversing si sta commettendo un atto illegale.
\end{itemize}
Si fa reversing per analizzare il comportamento del malware per analizzarne gli effetti, o per produrre delle forme di protezione verso di essi come gli antivirus: gli antivirus sfruttano due grandi tecniche per il funzionamento
\begin{enumerate}
\item riconoscere la firma del virus
\item riconoscere una sequenza di operazioni "tipiche" fatte dai virus 
\end{enumerate}
il problema è fare il reversing dei virus in circolazione per poter costruire l'antivirus.\\ Si può fare reversing del malware anche per scrivere nuovi malware, in modo da cambiare le firme ed esulare gli anti-virus appena diffuso\\ Un'altra attività di reversing viene fatta sui SO, al fine di attaccarli ad esempio tramite 0-day (ovvero attaccare con dei malware mai visti prima), o anche per rendere il SO più robusto.\\ C'è poi un'area del reversing che si occupa delle debolezze degli algoritmi crittografici, nella loro implementazione.\\ Alcune di queste attività sono illegali, alcune addirittura in quasi tutte le nazioni del mondo. Ci sono attività  legali o tollerate, perché il punto chiave è il perché si fa questa attività ad esempio a scopri di didattica, anche se fatto su programmi commerciali. Anche se nelle licenze d'uso c'è scritto che non è possibile fare RCE, questo può essere vessatorio, quindi poi la decisione finale spetta al sistema giuridico. \textbf{esempio:} sembra essere consentito in Italia fare reversing sul software di una stampante al fine di vendere cartucce di inchiostro riciclate. Le stampanti bruciano dei bit del componente elettronico della cartuccia quando questa è terminata, quindi occorre capire come questo viene fatto dal software.\\ La cosa diviene illegale se io faccio la stessa cosa per vendere cartucce nuove compatibili con la stampante.\\ L'attività ed i tool sono sempre gli stessi, ma quello che cambia è \textbf{il perché} viene fatta: spesso il confine fra legalità ed illegalità è sottile. L'attività di RCE non è lecita o illecita di per se, è lo scopo che se ne fa che la rende tale (Ricorda la differenza fatta a SERT fra hacker e cracker).
\subsubsection{Principi generali del RCE}
IL RCE può essere considerato l'operazione inversa della programmazione: nella programmazione l'idea di algoritmo viene tradotto in linguaggio di alto livello e poi compilato in codice macchina; in RCE si fa l'opposto.\\ Un programmatore segue dei principi nello scrivere il codice, quanto più si adottano correttamente i principi, quanto più si fa bene. Anche nel RCE vanno seguiti una serie di principi e regole generali per fare un buon lavoro. I principi generali:
\begin{itemize}
\item[\#] \textbf{Maggiore è la comprensione dell'interno sistema, tanto più rapida ed efficiente è l'attività di reversing}. Non si può fare reversing di qualcosa di qui non si sa nulla, l'hacker che fa il reversing deve essere competente in diversi settori dell'informatica: SO, architetture dei calcolatori, etc... Meno cose si sanno, più difficoltà si avrà nel lavoro di reversing. Altre competenze da avere è capire il processo di compilazione del programma, come i costrutti di alto livello vengono tradotti in assembly, siccome i dettagli dipendono dal compilatore cambiano a seconda del compilatore. Il formato del file eseguibile è un altro elemento importante: una cosa sono le istruzioni macchina, un altro è il contesto di esecuzione del file eseguibile, ovvero di come poi il programma verrà effettivamente eseguito. Inoltre, il lavoro di reversing è una "battaglia di teste" e siccome chi ha scritto il malware non ha interesse che sia analizzato, mette in piedi una serie di misure di offuscamento, protezione, rilevazione del debugger, in macchine virtuali e sandbox e altro. Chi fa il malware vuole proteggersi da chi vuole analizzarlo, inventando sempre nuove soluzioni ed il lavoro è cercare di capire e superare queste soluzioni.\\ Per questo motivo, fare reversing è sempre un atto creativo, dove si usa sempre la testa ma va comunque supportato dalla conoscenza e quindi richiede un continuo sforzo ed aggiornamento.
\item[\#] \textbf{Per capire il codice scritto da alti, è necessario capire come funziona il proprio}. Se chi ha scritto il malware ha necessario, ad un certo punto, una struttura di dati dinamica, come un albero bilanciato, è necessario sapere come gestire e implementare una struttura dati dinamica. L'obiettivo del processo di reversing non è capire cosa fa una singola istruzione macchina, bensì tutto ciò che fa il programma per capire cosa pensava chi l'ha concepito. È essenziale saper programmare bene
\item[\#] \textbf{L'attualità e la conoscenza dei tool di reversing determina la qualità del processo di reversing}. Le applicazioni moderne sono costituite da una grande quantità di codice macchina, che non è gestibile a mano. Ormai sono sempre più grandi di ciò che si può gestire senza strumenti sofisticati, quindi bisogna saper usare gli strumenti giusti. Oggi, ghidra è uno dei principali strumenti per fare RCE (open source). Gli strumenti sono sofisticati, la curva di apprendimento non è del tutto lineare, per cui è necessaria molta pratica per saperli usare.
\item[\#] \textbf{La chiave del reversing è la capacità di identificare e comprendere gli schemi ricorrenti nel codice, di conseguenza non esiste sostituto dell'esperienza}. La bravura di chi fa RCE si misura nel numero di ore dedicate al fare reversing. È sempre un principio generale, tanto più si ha esperienza, quanto più è rapido fare RCE.\\ I compilatori producono i costrutti di alto livello in certi pattern di codice macchina, con i decompiler si ottengono i pattern ad alto livello ma non è detto che il codice ad alto livello sia più facile da capire cosa il programma fa dal codice ad alto livello. L'abilità principale dell'hacker esperto è quella di saper riconoscere le strutture nascoste nel linguaggio macchina
\end{itemize} 
I principi generali di per se non sono sufficienti, serve anche metodologia. Alle volte si dice che la programmazione è arte, compresa solo da altri programmatori. In effetti, anche l'attività di RCE può essere considerata una forma d'arte perché richiede intuizione, capacità di problem solving. Punto importante: metodo, costanza, tempo e impegno mettono in condizione chiunque di poter fare questo mestiere.
\subsubsection{Metodologia}
Non si può prescindere dalla metodologia, bisogna avere ben chiari gli obiettivi. Bisogna avere chiaro la domanda a cui dare risposta e deve essere chiara, perché se ci cerca di scoprire tutto, può richiedere moltissimo tempo. Occorre stabilire in che modo ottenere l'obiettivo: se c'è un virus pericolo che può infettare la macchina, non c'è cosa più pericolosa di essere infettati sulla macchina con cui si fa reversing: non deve mai avvenire, quindi non è detto che si può operare sul malware eseguendolo. Inoltre, essendo un processo creativo, bisogna di continuo verificare la correttezza di cosa si sta facendo. Tutto ciò rende necessario formalizzare una metodologia, meglio se scritta. Gli approcci fondamentali del reversing:
\begin{itemize}
\item analisi "black box" o live code analysis: eseguo il programma e cerco di capire cosa fa, in ambienti più o meno controllati. Non è sempre possibile farlo e non può fornire informazioni su porzioni di codice non eseguite
\item analisi white box: analizzo il codice e cerco di capire cose fa "guardando nella scatola". Il problema dell'analisi è il costo in termini di effort e di tempo
\item analisi mista o gray box: approccio che in linea di principio combina metodi white box e black box, mischiando i due livelli. Il principale tool che si usa in questa fase è il debugger, in generale la metodologia è molto efficace e quindi nei malware ci sono una serie di elementi per cercare di renderla difficile.
\end{itemize}
Esempio di metodologia generale in 9 passi:
\begin{enumerate}
\item descrizione preliminare: descrivere cosa si sa, da dove viene il malware, cosa ha prodotto etc... Tutto quello che si sa va scritto
\item formalizzazione dell'obiettivo: non possiamo pensare di sapere tutto. Decidere cosa scoprire, ad esempio l'IP a cui si collegava, che file ha esfiltrato. Passo cruciale, perché l'attività di reversing va centrata su questo passo, inoltre permette di definire quanto tempo ci metterò a fare reversing
\item ottenimento del codice macchina: può essere immediato, alle volte il codice è offuscato o protetto con cifratura. È un passo non banale, alle volte è necessario fare il de-offuscamento a mano
\item osservazione del funzionamento: se posso, faccio analisi black box, è analisi dinamica. In alcuni casi non posso farla, se il codice ad esempio sfugge al controllo o se non posso eseguirlo
\item disassemblaggio white box, se non ho risolto il problema al passo 4, tipicamente si usano disassemblatori interattivi, ovvero che consentono di interagirvi per indirizzarlo nel suo lavoro. Può non funzionare correttamente, perché sono state usate delle contro misure
\item localizzazione del frammento assembly: trovo il frammento che può rispondere alla domanda. Servono delle tecniche per trovare il punto di interesse. Sono varie le tecniche per fare il passo
\item analisi del frammento assembly: una volta trovato il punto, si cerca di comprenderlo. È la fase più critica, è necessario ed essenziale annotare tutto ciò che si trova e se durante la fase ci sono punti che tornano ad essere interessanti si reiterano i precedenti punti
\item verifica dei risultati: occorre verificare che quanto scoperto è corretto
\item riepilogo in un report: viene riepilogato tutto ciò che è stato fatto, cosa si è appreso, cosa si è ottenuto etc...
\end{enumerate}

\end{document}