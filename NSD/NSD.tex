\documentclass[14pt]{article} % the document type needs to be change
\usepackage{geometry}

\geometry{
	a4paper,
	left = 1.5 cm,
	right = 1.5 cm,
	top = 2cm,
}

\title{Network and System Defense}

\begin{document}
\maketitle
\tableofcontents
\newpage
\section{Introduzione}
In cybersecurity, ci sono almeno 5 aree esterne differenti che si mischiano fra loro, e poi 3 aree interne relative all'hardware ed al software.\\ "L'attore principale" sono applicazioni Internet-based:
\begin{itemize}
\item vulnerabilità del sistema, della rete etc
\item come difendersi o evitare tali vulnerabilità, o almeno mitigarle
\end{itemize}
Molta configurazione pratica: laboratori da 0, ma è utile rifare tutto a casa (\textbf{CAZZI}).\\ Per via del fatto che è davvero difficile oggi separare cosa è la rete da cosa è il software, è utile parlare di tutti i diversi campi che riguardano sia la rete che il software.\\ Punti chiave:
\begin{itemize}
\item Access networks and perimetral security: Ethernet, VLAN, IPv6, 802.11x, firewall, packet classification algorithms
\item Core networks: BGP, MPLS, DDos e Botnets, VPNs con BGP
\item End to End security: PKIs, DNS security, HTTPS, Overlay VPNs
\item Sw and Operating System
\item Virtualization \& Cloud 
\end{itemize}
Piattaforme da utilizzare per la parte da 6 CFU (Hardware):
\begin{itemize}
\item GNS3
\item Tinycore Linux
\item VMware o Virtual Box: Virtual Box sembra lavorare meglio con GNS3
\item Cumulus Linux: serve per emulare le funzionalità di livello 2 di uno switch. Net OS, \textbf{scaricare la versione 4.1} e non le ultime. C'è un insieme di VM pronte per essere installate su un virtualizzatore
\item Immagini CISCO (da cercare online, if you know what I mean)
\item Lubuntu
\item Ubuntu server
\end{itemize}
\section{Lezione 1: Introduzione alla cybersecurity}
Ci sono 3 fattori principali quando si parla di cyberscurity:
\begin{itemize}
\item Cosa bisogna proteggere?
\item Da quali minacce? Ci sono diverse fonti di attacco da cui proteggersi
\item Come fare per contrastare queste minacce? Integrità, encryption, etc...
\end{itemize}
\paragraph{Definizione di computer security:}le misure ed i controlli che garantiscono confidenzialità, integrità e disponibilità degli assets dei sistemi informativi, incluso hardware, software etc...
\begin{itemize}
\item confidenzialità: i dati sono privati, le informazioni confidenziali non sono note ad entità non autorizzate. Esempi: usare SSH o HTTPS per scambiare credenziali con un server.
\item Privacy: (slides)
\item Integrità: se dei dati, indica che essi sono cambiati solo in maniera autorizzata. Se voglio traferire un messaggio con cui trasferisco del denaro, non voglio che il contenuto sia compromesso. Se si parla di integrità dei sistemi, si intende che il sistema performi come deve (vedi slides)
\item disponibilità: si assicura che il sistema lavori in maniera corretta e che i servizi non siano negati ad utenti autorizzati
\end{itemize}
Altri aspetti importanti:
\begin{itemize}
\item Authenticity: dell'utente, indica che sia possibile verificare che l'utente è chi dice di essere. Della sorgente: capacità di verificare che un messaggio arriva da una sorgente effettivamente affidabile.
\item altri ... (slides)
\end{itemize} 
\textbf{ALTRE IMPORTANTI DEFINIZIONI CHE SI TROVANO SULLE SLIDES}. \\ Quali sono le diverse superfici di attacco:
\begin{enumerate}
\item la rete stessa: i sistemi sono Internet-based, quindi è sempre possibile attaccare la rete. Attaccare la rete implica sfruttare vulnerabilità della rete Internet: ARP, WEP, link fisici, intrusioni nella rete etc...
\item attacchi al software: vulnerabilità nell OS, server web, code injection etc...
\item attacchi all'umano: molte vulnerabilità create da noi stessi, social engineering, errori umani, mancanza di competenze etc...
\end{enumerate}
È importante non pensare che la sicurezza sia data solo dalla tecnologia, ma anche dalle persone.\\ \paragraph{CIS critical security controls:} 20 applicazioni pratiche che la SANS suggerisce di applicare per verificare la sicurezza di una corporation (anche piccola Enterprise). Ogni controllo ha dei sotto-controlli più "concreti"\\ Quindi, non avendo il tempo di fare tutto, ci concentrare su alcuni aspetti fondamentali di tutto l'insieme visto.
\section{Lezione 2: vulnerabilità intrinseche IP/TCP}
Prima di capire le vulnerabilità, è necessaria una revisione dei protocolli di rete.
\subsection{Architettura di IP}
Intenet non è altro che una inter-connessione di reti che possono avere differenti tecnologie.\\ La comunicazione fra i device è possibile con IP, che è implementato sui livelli 1 e 2 (che sono indipendenti). Ogni dispositivo è identificato dall'indirizzo univoco a 32/64 bit (che seguono la nomenclatura CIDR), le diverse sotto-reti comunicano tramite i router ovvero un device IP che ha almeno 3 livelli dello stack Internet. Ha diverse interfacce che possono essere in diverse tecnologie: ADSL, Ethernet, Fibra etc...\\ Le operazioni compiute dai router:
\begin{itemize}
\item IP forwarding: longest prefix mathcing, manda un pacchetto verso la prossima interfaccia del percorso di rete. È diretta se la destinazione è nella subnet, mentre indiretta se serve trovare il next hop, quindi andare fuori dalla propria sotto-rete. Il next hop è un dispositivo nella propria sotto-rete che sa come arrivare alla destinazione.
\end{itemize}
Il goal di IP è quello di mandare il pacchetto a destinazione, l'Internet è diviso in Autonomous Systems all'interno dei quali si può configurare il routing come si vuole. Ma poi, serve scambiare informazioni fra As, si parla quindi di protocolli intra-AS come OSPF, RIP e protocolli inter-AS, come BGP.\\ C'è l'header per ogni pacchetto che ha i diversi campi (TTL, src, dest, protocol, checksum etc...), diverso fra IPv4 ed IPv6, differenze fra i due:
\begin{itemize}
\item header più piccolo in IPv6
\item no fragmentation in IPv6
\end{itemize}
\paragraph{Routing table:}struttura dati che associa una destinazione ad un next hop. Per il routing si usa il longest prefix matching, l'IP più "specifico" è quello scelto. La differenza principale fra host e router è che se un host riceve un pacchetto da una src non fra quelle locali, lo scarta; un router lo forwarda. Il forwarding standard basico è fatto in base all'indirizzo di destinazione. Dopo l'estrazione del pacchetto IP, se l'IP dest è locale, lo manda al livello superiore, altrimenti lo guarda nella tabella di routing: se non si trova un matching, il pacchetto è scartato. Altrimenti: se c'è un next hop, bisogna scoprire chi è, altrimenti basta scoprire il MAC associato all'IP.\\ La maggior parte dei pacchetti che vengono mandati vanno fuori dall'AS con cui si firma il contratto: possono esserci diversi AS di transito prima di arrivare al data center del sito a cui si cerca di connettersi.
\subsection{Vulnerabilità di TCP/IP}
Per design, IP e TCP non si preoccupano della sicurezza in quanto sono stati progettati quando la sicurezza non era un problema. C'è un certo numero di vulnerabilità intrinseche nei protocolli, ed è ancora così in quanto sono stati progettati senza pensare all'aspetto di sicurezza:
\begin{itemize}
\item Identification: i device della rete possono essere falsificati, è possibile spoofare l'IP address per dire di essere , ad esempio, un certo server web. Sia per i pacchetti generati dall'utente che per quelli forwardati. Se mandiamo un pacchetto di un legittimo server DNS, stiamo impersonificando il server stesso. Non ci sono meccanismi in IP per verificare l'autenticità di chi manda il pacchetto
\item Reputation: come è possibile verificare che l'origine del pacchetto è effettivamente la sorgente?
\item Confidentiality: non vogliamo che le informazioni scambiate sulla rete siano viste da 3e parti. Vorrei che solo il server veda i dati in uno scambio client-server. Ma intercettare il pacchetto in maniera "malevola" è fattibile ed anche decodificarlo perché l'header e l'IP sono in chiaro.\\ Il problema non è nella rete interna, ma passa per diversi AS (noi ci fidiamo solo del nostro): anche se il path fra sorgente e destinazione è fidato, è possibile fare hijaking su BGP, fra gli attacchi più pericolosi e in voga.
\item Integrity: voglio essere sicuro che il pacchetto non sia modificato durante il percorso fra src e dest. C'è il checksum, ma è calcolato su dati in chiaro, quindi non va bene come "codice" per poter assicurare l'integrità
\item Packet replication: non c'è sequence number nell'IP header nella sessione, non ci sono meccanismi per proteggersi da questo threat
\end{itemize}
\subsubsection{Dynamic mapping}
Le cose si complicano perché i protocolli Internet usano diversi meccanismi per implementare il mapping: ad esempio il DSN, ARP, 802.3 bridging (lo switch impara in automatico quale MAC è dietro quale porta) e questo è dinamico in quanto può cambiare nel tempo. Questo problema è presente in ogni layer ed anche questi meccanismi non sono stati penasti per essere sicuri.
\paragraph{DNS spoofing:}possiamo spoofare una richiesta DNS, quindi fare un hijack della sessione. Attacco triviale (e vecchio):
\begin{itemize}
\item sfruttiamo il mapping MAC-IP per diventare MiTM
\item sfruttiamo il mapping fra il nome di dominio e l'IP per impersonare il server web
\item mappiamo l'IP del sito al nostro
\end{itemize}
Lo scenario è che l'attaccante sia nella stessa sotto-rete.\\ L'idea è quella di mandare dei messaggi ARP spoofati per far credere alla vittima di essere il default gateway e di fra credere al default gateway di essere la vittima.\\ Per impersonare il sito target è possibile usare il comando \textsf{wget}, ed usare Apache2 per impersonare il server web.\\ L'attacco sfrutta 3 vulnerabilità:
\begin{itemize}
\item ARP spoofing
\item DNS redirection
\item Impersonification
\end{itemize}
Per emulare un server DNS è possibile configurare bind ma c'è un piccolo servizio DNS da configurare e che permette di forwardare all'attacker tutto ciò che sono è noto.
\subsection{Security requirements per le vulnerabilità}
Vediamo cosa è possibile risolvere questi problemi: per la confidenzialità si usano algoritmi a chiave simmetrica per garantirla.\\ Un algoritmo a chiave simmetrica, entrambe le end della comunicazione condividono la stessa chiave, c'è anche la differenza fra:
\begin{itemize}
\item stream ciphers 
\item block ciphers, che usano concatenazione per evitare ECB, in cui se la stessa chiave viene riusata per molto è possibile scoprire il contenuto di blocchi che hanno lo stesso contenuto
\end{itemize}
Per l'integrità, vogliamo una prova che nessuno modifichi il contenuto del messaggio, quindi si produce un MAC basato su hash functions: l'hash è calcolato non solo sui dati ma anche sulla chiave.\\ L'Authenticated Encryption permette, con "la stessa chiave" di fornire sia confidenzialità che integrità. Si può ottenere un algoritmo del genere anche con le tecniche base di symmetric encyption.\\ Per l'autenticità è possibile usare pub key cryptography per realizzare dei meccanismi di firma, come RSA e Diffie-Hellman. Serve comunque avere una certificazione che un utente è il vero proprietario di una chiave, quindi che sia crittograficamente legato alla chiave: Public Key Infrastructure.
\section{Lezione 3: Ethernet LAN security}
Ethernet è una tecnologia di livello 2, il primo standard open e multi-vendor nato nel 1976.\\ L'802 è una famiglia di standard dell'IEEE, fra di essi ci sono diversi gruppi di lavoro e quello che si occupa di Ethernet è 802.3 e quindi si parla di questi standard, in quanto Ethernet era il nome commerciale.\\ La trama Ethernet originale ha un header molto snello, c'è il campo type / length che può essere due campi differenti a seconda della situazione (vedi slides)\\ Anche in questo caso è tutto in chiaro, quindi ci sono gli stessi problemi. Gli indirizzi MAC sono a 48 bit
\paragraph{Hub, Switch e Bridge:} un hub è uno strumento che forwarda pacchetti verso tutti. Uno switch ha un forwading database, ovvero una struttura dati che associa MAC alle porte. Lo switch impara in automatico quale è il MAC dietro una certa porta. Il problema sta proprio nel fatto che lo switch impara in automatico il mapping.\\ Si usa l'algoritmo spanning three per evitare i loop, che si possono creare perché, per motivi di ridondanza, è possibile avere dei loop nella rete.\\\\ Per operare al di sopra di Ethernet, IPv4 usa ARP e DHCP mentre IPv6 usa NDP, protocollo simile. NDP non è sicuro, ma c'è un protocollo per renderlo sicuro e lo stesso protocollo è stato progettato già pensando alla sicurezza.\\ Anche nel caso di DHCP, usato per ottenere un IP, abbiamo gli stessi problemi detti sopra.
\subsection{Ethernet LAN vulnerabilities}
Quali sono le possibili contro-misure per difendersi dai threats. I problemi di Ethernet sono dovuti al fatto che per natura si auto.configura. Siccome è possibile avere accesso a qualunque trama Ethernet, l'attaccante può:
\begin{itemize}
\item imparare qualcosa sulla topologia della rete
\item avere accesso agli switch
\item eavesdropping
\item etc...
\end{itemize}
Vediamo alcune categorie di problemi
\subsubsection{Network and system access}
Come è possibile ottenere accesso ad un Ethernet segment: unioni non autorizzate:
\begin{itemize}
\item tramite accesso fisico allo switch, se la porta è attiva
\item accedere ad una socket a muro
\item rimuovere il cavo ad un PC e metterlo in un altro
\item inserire uno switch fra il PC esistente e la socket
\end{itemize}
Espansioni della rete non autorizzate.\\ Un altro modo per ottenere l'accesso è quello di accedere in maniera remota, o anche fare il probe della rete per scoprire informazioni, come ad esempio usare nmap. Altri modi:
\begin{itemize}
\item break-ins
\item switch control: uno switch che ha password di default o non la ha e la password può essere resettata fisicamente. Se si ottiene il controllo di uno switch si possono fare varie cose, ad esempio se lo switch gira su cumulus Linux
\end{itemize}
Una volta che si è ottenuto l'accesso alla rete, si può fare lo sniffing del traffico di rete. Oggi non è più vero che la trama arriva a tutti e che sono tutti connessi, se la possibilità di intercettare fisicamente il pacchetto, è possibile fare un MiTM etc...\\ Ma il problema è anche nella procedura di learning dello switch: il MAC non è autenticato, quindi mandando una trama con un MAC reale allo swtich, viene ingannato nel pensare che il MAC è dietro una porta sbagliata.\\ Quindi l'attaccante può redirigere il traffico trasmesso dalla vittima verso se stesso, anche se è più che altro teorico perché anche la vittima manda messaggi.\\ Si può generare un pacchetto con un indirizzo falso, o anche intercettare un pacchetto e cambiarlo in quanto anche a livello 2 non c'è integrity check o HMAC tag.\\ Per cambiare il MAC, ci sono vari modi:
\begin{itemize}
\item cambiare il MAC della NIC (comando Linux)
\item usare raw socket programming
\item in-kernel programming
\end{itemize}
Per intercettare il pacchetto, basta mettere l'interfaccia di rete in modalità promiscua per ricevere tutto il traffico anche non diretto a me, così da poter cambiare il MAC.
\paragraph{MAC flooding:}l'idea è di inondare lo switch con un grande numero di trame con diversi indirizzi MAC per saturare la memoria dello switch in modo da far si che tutti i pacchetti siano inviati verso tutte le porte in broadcast.\\
\paragraph{ARP e DHCP poisoning:}si può fare ARP poisoning (vedi sopra) ma anche facendo poisoning di DHCP facendo creder di essere il server, spoofando tutto lo scambio DHCP.\\ Nell'ARP poisoning si avvelena tutta la ARP cache, che è usata per memorizzare risultati di precedenti ARP request per evitare di rifarle ogni volta (ogni riga ha comunque un TTL). È sempre possibile configurare la cache ARP in maniera statica. Le ARP gratuite sono utili per vari motivi, non solo pericolosi. È possibile fare un MiTM: ho un PC ed un default gateway, mando una ARP response (op code 2) col MAC della vittima al default gateway ed una ARP response col MAC del default gateway alla vittima.\\ (N.B:ARP non è stato progettato specificamente per mappare IP-MAC, ma per associare un indirizzo di layer 3 ad uno di layer 2).
\paragraph{Session hijack: }a livello 2, Ethernet non ha conoscenza della sessione, è come IP: non c'è relazione fra pacchetti inviati in sequenza, e questo è un'altra vulnerabilità. È possibile dirottare una sessione e iniziare a trasmetterli in una sessione già stabilita, perché Ethernet non ha protezioni per questo.
\paragraph{Denial of service:}l'obiettivo è quello di negare il servizio, si può fare nel layer Ethernet. È possibile esaurire le risorse di una macchina, cercare di far crashare lo switch ma anche farlo a livello di protocollo. STP (Spannign Three)permette di gestire grandi topologie di rete, con molti switch e link ridondanti, quindi loop fisici. Con STP si disabilita la porta che causa il loop, ma anche STP non  autenticato e si può far credere di essere uno switch.
\subsection{Contromisure}
Per risolvere i problemi, è stato deciso che tutte le trame Ethernet vengano marcate come non sicure. Per farlo è necessario metterlo in un dominio protetto, ovvero protezioni perimetrali come il firewall etc... Altrimenti, è possibile usare soluzioni crittografiche, in ogni caso ci sono 4 categorie di contromisure:
\begin{itemize}
\item router based security: se si può rimpiazzare uno switch con un router, si risolvono molti problemi. Un IP router messo fra tutti i nodi, non c'è un singolo dominio di broadcast, che è interrotto perché il router non forwarda i pacchetti in broadcast. In questo modo si risolvono ARP, STP, MAC table based attacks. (Si possono avere dei singoli dominii di broadcast usando le VLAN.)
\item access control: se non si può sostituire con un router, si può comunque controllare chi accede alla rete in differenti modi. Per farlo, uno dei modi è usare il protocollo 802.1X port authentication: l'idea è che la porta di uno switch è aperta solo se il client s è autenticato correttamente verso un'altra entità, che è l'authentication server. Un altro modo è usare le ACL (Access Control List)
\end{itemize}
\end{document}