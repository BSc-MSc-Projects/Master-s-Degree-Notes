\documentclass[12pt, oneside]{extbook} % the document type needs to be change
\usepackage{geometry}
\usepackage{listings}
\usepackage{graphicx}
\usepackage[utf8]{inputenc}
\usepackage[T1]{fontenc}
\usepackage[italian]{babel}

\geometry{
    top = 1.5cm,
    bottom = 1.5cm,
    left = 2cm,
    right=2cm,
}

\begin{document}
\chapter*{Introduzione}
In cybersecurity, ci sono almeno 5 aree esterne differenti che si mischiano fra loro, e poi 3 aree interne relative all'hardware ed al software.\\ "L'attore principale" sono applicazioni Internet-based:
\begin{itemize}
\item vulnerabilità del sistema, della rete etc
\item come difendersi o evitare tali vulnerabilità, o almeno mitigarle
\end{itemize}
Molta configurazione pratica: laboratori da 0, ma è utile rifare tutto a casa (\textbf{CAZZI}).\\ Per via del fatto che è davvero difficile oggi separare cosa è la rete da cosa è il software, è utile parlare di tutti i diversi campi che riguardano sia la rete che il software.\\ Punti chiave:
\begin{itemize}
\item Access networks and perimetral security: Ethernet, VLAN, IPv6, 802.11x, firewall, packet classification algorithms
\item Core networks: BGP, MPLS, DDos e Botnets, VPNs con BGP
\item End to End security: PKIs, DNS security, HTTPS, Overlay VPNs
\item Sw and Operating System
\item Virtualization \& Cloud 
\end{itemize}
Piattaforme da utilizzare per la parte da 6 CFU (Hardware):
\begin{itemize}
\item GNS3
\item Tinycore Linux
\item VMware o Virtual Box: Virtual Box sembra lavorare meglio con GNS3
\item Cumulus Linux: serve per emulare le funzionalità di livello 2 di uno switch. Net OS, \textbf{scaricare la versione 4.1} e non le ultime. C'è un insieme di VM pronte per essere installate su un virtualizzatore
\item Immagini CISCO (da cercare online, if you know what I mean)
\item Lubuntu
\item Ubuntu server
\end{itemize}

\chapter{Introduzione alla cybersecurity}
Ci sono 3 fattori principali quando si parla di cyberscurity:
\begin{itemize}
\item Cosa bisogna proteggere?
\item Da quali minacce? Ci sono diverse fonti di attacco da cui proteggersi
\item Come fare per contrastare queste minacce? Integrità, encryption, etc...
\end{itemize}
\paragraph{Definizione di computer security:}le misure ed i controlli che garantiscono confidenzialità, integrità e disponibilità degli assets dei sistemi informativi, incluso hardware, software etc...
\begin{itemize}
\item confidenzialità: i dati sono privati, le informazioni confidenziali non sono note ad entità non autorizzate. Esempi: usare SSH o HTTPS per scambiare credenziali con un server.
\item Privacy: assicurarsi che gli individui controllino o influenzino quale informazione è legata a loro potrebbe essere raccolta e a chi potrebbe essere rivelata;
\item Integrità: se dei dati, indica che essi sono cambiati solo in maniera autorizzata. Se voglio traferire un messaggio con cui trasferisco del denaro, non voglio che il contenuto sia compromesso. Se si parla di integrità dei sistemi, si intende che il sistema performi, che sia libero da una manipolazione dello stesso;
\item disponibilità: si assicura che il sistema lavori in maniera corretta e che i servizi non siano negati ad utenti autorizzati. In infrastrutture critiche, la disponibilità può diventare di gran lunga l'obiettivo di sicurezza più importante
\end{itemize}
Altri aspetti importanti:
\begin{itemize}
\item Authenticity: dell'utente, indica che sia possibile verificare che l'utente è chi dice di essere. Della sorgente: capacità di verificare che un messaggio arriva da una sorgente effettivamente affidabile.
\item Controllo d'accesso (autorizzazione), ovvero l'abilità di verificare che l'utente ha il permesso di effettuare alcune attività;
\item accountability, ovvero poter tenere traccia delle azioni di qualche entità, incluso la possibilità di salvare tali attività in un file di log per analisi forensi successive;
\item adversary: individuo, gruppo governo... che vuole condurre attività dannose
\item contromisure: un dispositivo o tecnica che ha come obiettivo la prevenzione di spionaggio etc...
\item rischio, ovvero una misura di quanto una entità sia minacciata da potenziali eventi
\item policy di sicurezza, quindi un insieme di criteri per fornire servizi di sicurezza
\item risorsa di sistema: un'applicazione major, un supporto generale al sistema, un programma ad alto impatto ...
\item minaccia (threat): una qualunque circostanza o evento col potenziale di impattare in maniera avversa delle organizzazioni
\item vulnerabilità: debolezza di un sistema informativo, procedura di sicurezza ...
\end{itemize} 
Quali sono le diverse superfici di attacco:
\begin{enumerate}
\item la rete stessa: i sistemi sono Internet-based, quindi è sempre possibile attaccare la rete. Attaccare la rete implica sfruttare vulnerabilità della rete Internet: ARP, WEP, link fisici, intrusioni nella rete etc...
\item attacchi al software: vulnerabilità nell OS, server web, code injection etc...
\item attacchi all'umano: molte vulnerabilità create da noi stessi, social engineering, errori umani, mancanza di competenze etc...
\end{enumerate}
È importante non pensare che la sicurezza sia data solo dalla tecnologia, ma anche dalle persone.\\ \paragraph{CIS critical security controls:} 20 applicazioni pratiche che la SANS suggerisce di applicare per verificare la sicurezza di una corporation (anche piccola Enterprise). Ogni controllo ha dei sotto-controlli più "concreti"\\ Quindi, non avendo il tempo di fare tutto, ci concentrare su alcuni aspetti fondamentali di tutto l'insieme visto.

\end{document}
